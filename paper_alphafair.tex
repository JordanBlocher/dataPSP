\documentclass[12pt]{article}
 
\usepackage[text={6in,8.1in},centering]{geometry}

\usepackage{enumerate}
\usepackage{amsmath,amsthm,amssymb}
\usepackage{mathrsfs} % to use mathscr fonts

\usepackage{multicol}

\usepackage{epstopdf}
\usepackage{caption,subcaption}
\usepackage{pstricks}
\usepackage{pst-solides3d}
\usepackage{pstricks-add}
\usepackage{graphicx}
\usepackage{pst-tree}
\usepackage{pst-poly}
\usepackage{calc,ifthen}
\usepackage{float}\usepackage{multicol}
\usepackage{multirow}
\usepackage{array}
\usepackage{longtable}
\usepackage{fancyhdr}
\usepackage{algorithmicx}
\usepackage[noend]{algpseudocode}
\usepackage{changepage}
\usepackage{color}
\usepackage{listings}
\usepackage{fancyvrb}
\usepackage{verbatim,moreverb}
\usepackage{courier}
\usepackage{algorithm}

\lstset{ %
language=C++,               
basicstyle=\footnotesize,
numbers=left,                  
numberstyle=\tiny,     
stepnumber=1,         
numbersep=5pt,         
backgroundcolor=\color{white},  
showspaces=false,               
showstringspaces=false,         
showtabs=false,                 
columns=fullflexible,
frame=single,          
tabsize=2,          
captionpos=b,       
extendedchars=true,
xleftmargin=17pt,
framexleftmargin=17pt,
framexrightmargin=17pt,
framexbottommargin=4pt,
breaklines=true,       
breakatwhitespace=false, 
escapeinside={\%*}{*)}       
}

\newenvironment{block}{\begin{adjustwidth}{1.5cm}{1.5cm}\noindent}{\end{adjustwidth}}

\newtheorem{proposition}{Proposition}[section]
\newtheorem{theorem}{Theorem}[section]
\newtheorem{lemma}{Lemma}[section]
\newtheorem{corollary}{Corollary}[section]
\theoremstyle{definition}
\newtheorem{definition}{Definition}[section]

 
\def\verbatimtabsize{4\relax}
\def\listingoffset{1em}
\def\listinglabel#1{\llap{\tiny\it\the#1}\hskip\listingoffset\relax}
\def\mylisting#1{{\fontsize{10}{11}\selectfont \listinginput[1]{1}{#1}}}
\def\myoutput#1{{\fontsize{9}{9.2}\selectfont\verbatimtabinput{#1}}}

\newcommand{\vs}{\varsigma}
\newcommand{\mcL}{\mathcal{L}}
\newcommand{\mcI}{\mathcal{I}}
\DeclareMathOperator*{\argmax}{arg\,max}
\DeclareMathOperator*{\argmin}{arg\,min}
 
\headsep25pt\headheight20pt
 
 
\pagestyle{fancyplain}
\rhead{\fancyplain{}{\small\bfseries Blocher, Jordan}}
\cfoot{\ \hfill\tiny\sl Draft printed on \today}
 
 
\setlength{\extrarowheight}{2.5pt} % defines the extra space in tables
 
\begin{document}
\begin{multicols}{2}

\section{Introduction}
China Mobile Hong Kong (CMHK) recently introduced such a secondary market.
CMHK’s 2cm (2nd exchange market) data exchange platform allows users to submit
bids to buy and sell data, with CMHK acting as a middleman both to match buyers
and sellers and to ensure that the sellers’ trading revenue and buyers’
purchased data are reflected on customers’ monthly bills. \cite{zheng}

In this work, we propose a distributed progressive second price (PSP) auction in
order to maximize social utility in this secondary data-sharing market. We show that for cellular data allocated between multiple users there
exists an $\epsilon$-Nash market equilibria when all users bid their real marginal
valuation of the mobile data offered in the secondary market described in
\cite{zheng}. A quality of the PSP auction is that
demand information is not known centrally, rather, it is distributed in the
buyers' valuations. The mechanism for an auction is defined
as \emph{distributed} when the allocations at any element depend only on
\emph{local} state: the quantity offered by the seller at that element, and the
bids for that element only. In this work, the proposed mechanism allows sellers submit bids to buyers
directly; there is no entity that holds a global market knowledge.

We suppose that each seller (resp. buyer) can submit a bid to the secondary
market consisting of (i) an available (required) quantity and (ii) a unit-price (calculated
using its own demand functions). Buyers submit bids cyclically until an
($\epsilon$-Nash) equilibrium is reached where $\epsilon$ corresponds to a bid
fee to be paid to the ISP on completion of the transaction. (FEE IS FIXED OR
PER-UNIT?)(DO WE MODEL ISP PROFIT? FUTURE WORK)

The form of the auction mechanism presented here is (CAN BE? NEED TO SHOW TO
CLAIM 'IS') described as a
pure-strategy progressive game with incomplete, but perfect information. (WHAT
DOES NASH SAY ABOUT THIS?) (TRY MIXED? CAN ONLY HAVE MIXED WITH A DISTRIBUTED
VALUATION.. FUTURE WORK) 

\subsection{Distributed Progressive Second Price Auctions}

\subsubsection{Bandwidth Allocation using PSP}

We begin with a brief introduction to the distributed PSP auction for the
bandwidth allocation problem, first introduced by Lazar and Semret \cite{lazar}.
The distributed PSP auction from \cite{lazar} forms a part of the
overall market based allocation model. They consider a noncooperative game
where a set of $\mcI = \lbrace 1,\cdots,I\rbrace$ users buy a fixed
amount of resource $D_i$ from a set of resources $\mcI = \lbrace 1,\cdots,L\rbrace$. 

Suppose each user $i \in \mcI$ makes a bid $s_i^l = (p_i^l, d_i^l)$ to the
seller of resource $l$,
where $p_i^l$ is the unit-price the user is willing to pay and $d_i^l$ is the
quantity the user desires. The \emph{bidding profile} forms a grid, $s \equiv
[s_i^l] \in \mcI \times \mcI$ (BAD NOTATION?
SEEMS OK)  and $s_{−i}^l \equiv [s_1^l , \cdots , s_{i−1}^l , s_{i+1}^l , \cdots
, s_I^l]_{l\in\mcI}$ is
the profile of user $i$’s opponents. In addition, the user type now includes a
\emph{routing variable}, $r_i\in\lbrace 0, 1\rbrace^L$ to manage throughput in
an arbitrary collection of nodes where $r_i^l >0$ if and only if $l(i)$ is on
$i$'s path, i.e. $l(i) \in \lbrace l\in \mcI : r_i^l >0\rbrace$. 
The market now shares $L$ resources, with allocation:
$$
    \hat{D_i^l}(y,s_{-i}^l) = \bigg\lbrack D^l -
\displaystyle\sum_{p_k^l > y, k\ne i} d_k^l\bigg\rbrack^+.
$$
The market price function (MPF) of user $i$ is defined as:
% MARKET PRICE 
\begin{align*}
    \tilde{P_i}(z, s_{-i}) &= \displaystyle\sum_{l\in\mcI}P_i^l(z_i^l,
s_{-i}^l)r_i^l \\
    = \displaystyle\sum_{l\in\mcI}&\bigg(\inf\bigg\lbrace y\ge 0 :
\hat{D_i^l}(y,s_{-i}^l)\bigg\rbrace r_i^l\bigg),
\end{align*}
which is interpreted as the aggragate minimum price a user bids over its route in order to obtain the
resource $z$ given the opponents’ profile $s_{−i}$. Its inverse function
$\tilde{D_i}$ is defined as follows:
% INVERSE PRICE (ALLOCATION)
\begin{align*}
    \tilde{D}_i(y, s_{-i}) &= \sup\bigg\lbrace z\in \bigg( 0,
\min_l\hat{D}_i^l/r_i^l\bigg) : \\
    &\tilde{P_i}(z,s_{-i}^l) < y\bigg\rbrace,
\end{align*}
which means the maximum available quantity at a bid price of $y$ given
$s_{−i}^j$. With this notation, the modified PSP allocation rule \cite{tuffin},
\cite{qpsp} is defined
as:
% ALLOCATION RULE
\begin{align*}
    \tilde{a_i}(s) &= \min\big\lbrace d_i^{l(i)}, \\
 &\frac{d_i^{l(i)}}{\sum_{k:p_k^{l(k)}=p_i^{l(i)}}d_k^{l(k)}} \tilde{D_i}(p_i^{l(i)},
s_{-i}^l)\big\rbrace,
\end{align*}
\begin{align*}
    \tilde{c_i}(s) &= \displaystyle\sum_{j\ne i}\bigg\lbrack\sum_{l\in\mcI}
p_j^{l(j)} \bigg(\tilde{a_j}(0; s_{-i}^l)\\
    &-\tilde{a_j}(s_i^l;s_{-i}^l)\bigg) r_i^l\bigg\rbrack,
\end{align*}
 $a_i^{l(i)}$ denotes the quantity user $i$ obtains by a bid price
$p_i^{l(i)}$ (when the
opponents bid $s_{−i}^l$) and the charge to user $i$ by the seller is denoted
$c_i^{l(i)}$. $\sum_{l\in\mcI} c_i^{l(i)}$ is interpreted to be the total cost incurred in the system if
user $i$ is removed from the auction. Note that the allocation rule is modified
from \cite{lazar} according to \cite{tuffin}, so that buyers with identical unit-price
$p_i$ are not rejected.
%%%%%%%%%%%%
\iffalse
An allocation rule is feasible \cite{lazar} if $\forall \ s$,
$$
    \displaystyle\sum_{i\in\mcI} a_i(s) \le Q
$$
and $\forall \ i\in\mcI$,
\begin{align*}
    a_i(s) &\le d_i \\
    c_i(s) &\le p_id_i.
\end{align*}
\fi
%%%%%%%%%%%%%%%%%%

Using this classic PSP mechanism, \cite{lazar} shows that given the opponents bids $s_{-i}$,
user $i$'s $\epsilon$-best response $s_i = (w_i, v_i)$ as a Nash move
(where $s_i$ is chosen to maximize $i$'s utility with $s_{-i}$ held constant), is
given by:
\begin{align}
\begin{split}
    v_i &= \sup\bigg\lbrace d \ge 0 : \theta '(d) > P_i(d), \\ 
&\displaystyle\int_0^d P_i(\eta) \ d\eta \le b_i\bigg\rbrace -
\frac{\epsilon}{\theta_i'(0)} \\
&\qquad\qquad \text{(best quantity reply)} \\
\end{split}\\
    w_i &= \theta_i'(v_i) \quad \text{(best unit-price reply)},
\end{align}
where $\epsilon > 0$ is the bid fee, $b_i$ is user $i$'s budget, and every user
has an elastic demand function.

\subsubsection{Data Allocation using PSP}

The main contribution of this work is to show that the rules of PSP auctions are applicable to the
secondary market described in \cite{zheng}. We find that the auction of data effects the
strategies of the users (buyers and sellers) and results in a (somewhat) different
auction from the original bandwidth problem. We intend to show that our auction
is rational and achieves the desired VCG properties, as does the original
formulation. Using \cite{zheng} as a basis for our model, and \cite{lazar} as
both a theoretic and notational template (BAD SENTENCE), we define optimal strategies for
buyers and sellers which allows for the development of a PSP auction mechanism
to achieve a network equilibrium for cellular data. (EXPAND)

We will introduce a user type called an \emph{opt-out buyer} in order to perform our analysis. The
opt-out buyer restricts its pool of sellers by only submitting bids to sellers
who are able to meet their requirements.
We intend to show that our PSP constraints are
sufficient to attain the desirable property of truthfullness through incentive compatibility. We
reason, due to our pricing mechanism, that our formulation upholds the \emph{exclusion-compensation
principle}, and is a valid progressive second price auction to the extent that buyer $i$
pays for its allocation so as to exactly cover the ``social opportunity cost" which is given by the declared willingness to pay (bids) of the users who are excluded by $i$'s
presence, and thus also compensates the seller for the maximum lost potential
revenue \cite{lazar}.

We extend the P2P rules described in \cite{semret} to include a \emph{local} market
price function as determined by the subset of nodes participating in the
auction, where the seller is its own auctioneer. This implies that the influence of the greater market on the individual
auctions will be influenced only by the submission of bids from buyers to
sellers. As a buyer may have access to multiple auctions, the sellers will be
dynamically influenced by the market via the $\epsilon$-best replies from the
buyers: the
valuation function of seller $j$ is dependent on the buyers demand, and is
modeled as a function of their potential revenue \cite{semret}. 

 

%%%%%%%%%%%%%%%%%%%
\iffalse 
We use the $\alpha$-fair utility functions \cite{zheng} to model the
valuation from consuming $d$ amount of data:
% BUYER VALUATION
\begin{equation}\label{buyervaluation}
    \theta(d) = \frac{\sigma d^{1-\alpha}}{1-\alpha}
\end{equation}
(AUGH! NEED X TO BE SO BIG COMPARED TO ALPHA! can we assume a minimum ask?)
where $\sigma$ is a positive constant representing (the scale) of the usage
utility and we take $\alpha \in [0, 1)$.
We verify that the user valuation above satisfies the conditions for an
\emph{elastic demand function}: (NOTE: this part seems in the wrong place)

\definition{\cite{lazar}}
A real valued function $\theta(\cdot)$ is an \emph{(elastic) valuation
function} on $[0, D]$ if 
\begin{itemize}
    \item $\theta(0) = 0$; \\
        \emph{Verification:} (obvious)
    \item $\theta$ is differentiable; \\
        \emph{Verification:} The derivative, $\sigma d^{-\alpha}$, is positive assuming non-negative
data requirements.
    \item $\theta ' \ge 0$, and $\theta '$ is non-increasing and continuous; \\
        \emph{Verification:} $U$ is differentiable for all $d$, and therefore
continuous. Its derivative is a negative exponential, and so is non-increasing.
    \item There exists $\gamma > 0$, such that for all $z \in [0,D]$, $\theta
'(z) > 0$ implies that for all $\eta \in [0, z), \theta '(z) \le \theta '(\eta)
- \gamma(z - \eta)$. \\
        \emph{Verification:} Without loss of generality, we may set the scaling
constant $\sigma=1$, and compute the curvature $\gamma(\xi)$, where by definition,
$$
    \gamma = \frac{\theta''}{(1+\theta')^{3/2}} = \frac{-\alpha
\xi^{-\alpha-1}}{(1+\xi^{-2\alpha)^{3/2}}}.
$$
Using a Taylor theorem approximation,
\begin{align*}
    z^{-\alpha} &\le \eta^{-\alpha} + \frac{-\alpha
d^{-\alpha-1}}{(1+\xi^{-2\alpha)^{3/2}}}(z-\eta) \\
    & \le \eta^{-\alpha} +
\frac{\alpha}{2\sqrt{2} \xi}(z-\eta)  \\
    &\le \eta^{-\alpha} +
\frac{\alpha}{\xi}(z-\eta).
\end{align*}
Now, using Taylor repeatedly, simplifying and taking the limit as $\eta \rightarrow z$,
\begin{align*}
    z^{-\alpha} - \eta^{-\alpha} &\le -\alpha\eta^{-\alpha-1}(z-\eta) \\
    & \le \frac{-\alpha}{\xi}(z-\eta).
\end{align*}
And so, since $\xi \le \eta^{\alpha+1}$,
we may set
$$
    \gamma \ge
\frac{-\alpha\eta^{-(\alpha+1)^2}}{\big(1+(\eta^{-2\alpha(\alpha+1)})^{3/2}},
$$
which holds in the case that $z > 1$, and so assuming that there must be at least one
unit of data required for a user to have
a valuation,  we have that the concavity of $\theta'$ is shown by Squeeze theorem.
\end{itemize}
\fi
%%%%%%%%%%%%%%%%%%%%%%


\section{Related Work}

\section{The Problem Model}
\subsection{The Secondary Market}

We consider the set of $\mcI = \lbrace 1, \cdots, I\rbrace$ users who purchase or sell
data from other users. A buyer submits bids directly to sellers who enough leftover data
to satisfy their demand, and will submit bids in order to maximize their
(private) valuation. We assume (SUPPOSE?) that the public information in the
secondary market consists of a set of offers that are published by users
wishing to sell their data overage. 
A user's identity $i \in \mcI$ as a subscript indicates that the user
is a buyer, and a superscript indicates the seller.
Suppose user $i$ is buying from user $j$. A bid $s_i^j = (d_i^j, p_i^j)$,
meaning $i$ would like to buy from $j$ a quantity $d^j_i$ and is willing to pay
a unit price $p^j_i$. Without loss of generality, we assume that all users bid in all
auctions; if a user $i$ does not submit a bid to $j$, then this means that
the user has the exact amout of data they require, or that seller $j$ does not have
enough data to satisfy the buyers' demand, and we simply set $s_i^j = (0, 0)$.
A seller $j$ places an ask $s^j = (d^j, p^j)$, meaning $j$ is offering a
quantity $d^j$ , with a reserve unit price of $p^j$ . In other words, for a
superscript, the bid is understood as an offer in the secondary
market; we assume that data is a unary resource belonging to the seller, and
therefore can identify the data (for sale) with the identity of the user.
To further clarify this restriction, we note that since $i$ is not a seller,
the superscript notation will not be used, as $d^i = 0$ and
$a^i = 0$. In our current formulation, we do not allow a seller to submit
multiple bids to the secondary market (FUTURE WORK).

Based on the profile of bids $s^j = (s^j_1, \cdots s^j_I)$, seller $j$ computes
an allocation $(a^j, c^j) = A^j(s^j)$, where $a^j_i$ is the quantity given to
user $i$ and $c^j_i$ is the total cost charged to user $i$. $A^j$ is the
allocation rule of seller $j$. It is feasible if $a^j_i \le d_i^j$, and $c^j_i \le  p^j_i d_i^j$.

\section{Distributed PSP Analysis}
\subsection{User Behavior}

\subsubsection{Buyer Strategy}
Absent the cost or revenue from trading data, CMHK users gain utility from consuming
data. 
We will assume that the buyer valuation satisfies the conditions for an
\emph{elastic demand function}: 
\definition{\cite{lazar}}
A real valued function for buyer $i$, $\theta(\cdot)_i$, is an \emph{(elastic) valuation
function} on $[0, b_i]$ if 
\begin{itemize}
    \item $\theta_i(0) = 0$,
    \item $\theta_i$ is differentiable,
    \item ${\theta_i}' \ge 0$, and ${\theta_i}'$ is non-increasing and continuous,
     \item There exists $\gamma > 0$, such that for all $z \in [0,D]$,
${\theta_i}
'(z) > 0$ implies that for all $\eta \in [0, z), {\theta_i}'(z) \le {\theta_i}'(\eta)
- \gamma(z - \eta)$. 
\end{itemize}
% OPT-OUT BUYER
We define a \textbf{opt-out buyer} as a user $i\in\mcI$ with utility
function,
\begin{equation}\label{opt-utility}
    u_i = \theta_i \circ e_i(a) - c_i,
\end{equation}
where $e_i : [0, \infty) \rightarrow [0,\infty)$ is %the expectation 
a function that maps buyer $i$'s data quantity (requirement) to matching
seller(s). We note that the opt-out buyers' valuation depends only on a scalar $e_i(a)$ which is a function of the
quantities of all the available data for sale in the secondary market. 

Suppose the total amount of seller $j$'s data on the network at the instance that
user $i$ joins the auction is $b^j$. 
The data transfer from each seller cannot exceed the total amount they have available,
i.e. $a_i^j \le b^j$. This will hold simultaneously for each $i \in
\mcI$ if and only if $b^j \ge \max_i \ a_i^j$. Therefore a seller $j$ is
restricted to subset of buyers $\in\mcI$.
Note that with our formulation, if a seller $j$ does not meet a buyer $i$'s data requirements, a
rational (utility-maximizing) buyer will set $s_i^j = 0$, i.e. they will not
place a bid, as in \cite{zheng}. The seller, in our analysis, is a functional
extension of the buyer, its valuation $\theta^j$ is dependent on the buyers. We will assume that buyers and sellers
are separated (a seller does not also buy data and vice versa).

Although it is possible for a seller to fully satisfy a buyer $i$'s demand, it
is also reasonable to expect that many sellers will come close to using their
entire data cap, and only sell the fractional overage. In this case, we
determine that a buyer must coordinate their bids among multiple sellers. (SAY
MORE?) Buyer $i$'s valuation is interpreted as a unit valuation $\theta_i$,
moderated
by a function of quantities desired from the distributed market. 
We define, in addition to the valuation $\theta_i$ and budget $b_i$, a generic
\textbf{data-provisioning vector} $\vs_i$, held by buyer $i$ as part of its type. We propose the
following strategy,
{
\proposition{(Opt-out buyer strategy)}\label{buyerstrategy}
Define, for any allocation $a$,
% OPT-OUT BUYER STRATEGY
\begin{equation}\label{opt-out}
    %e_i^j(a) \triangleq \frac{a_i^j}{\vs_i^j} + a_j^i,
    e_i^j(a) \triangleq \frac{a_i^j}{\vs_i^j},
\end{equation}
and define,
% MIN SET
$$
    \ell_i =\argmax_{\mcI' \subset \mcI, \vert\mcI'\vert =
n}\sum_{j\in\mcI'} d^j.
$$
Buyer $i$ chooses its seller pool by determining $n$, where
\begin{equation}\label{buyercoordinate}
    n = \argmin_{\ell_i}(j \in \ell_i : \displaystyle\sum_{n} \frac{d^n}{d_i} = d_i\rbrace,
\end{equation} 
which produces the minimal subset 
\begin{equation}\label{sellers}
    \mcI_i = \lbrace j \in \mcI : j < n\rbrace \subset \mcI.
\end{equation} 
Now let $j^* = n < I$, and define, 
\begin{equation}\label{opt-quantity}
    e_i(a) \triangleq e_i^{j^*}(a).
\end{equation}
Then, we have that $e_i$
is an optimal feasible strategy for buyer $i$.
}\\
\textbf{Proof:}
In the case that there exists a seller who can completely satisfy a buyers'
demand, $j^*=1$, $\vert \mcI_i\vert =1$ and (\ref{buyercoordinate}) holds. If such a buyer does not exist,
as the set $\ell_i$ is an ordered set, $i$ may discover 
$j^*$ by computing $\ell_i$. In the case that $d_i >
\sum_{j\in\mcI}d^j/d_i$, then $j^* > I$ and we consider the buyers' demand
infeasible (CAN DO BETTER!). 
We also note that $\mcI_i$ is not the only
possible minimum subset $\in\mcI$ able to satisfy $i$'s demand, it is the minimal subset where a
coordinated bid is possible. We will show optimality through the
analysis in Section \ref{analysis}.

\subsubsection{Buyer Influence}
To determine a strategy for the sellers, we examine seller incentive. The potential revenue for $j$ at unit price $y$ is determined by the
demand of the buyers. $\forall \ y\ge 0$, we determine that the demand is given by,
% BUYER DEMAND 
\begin{equation}\label{datademand}
    \rho^j(y) = \sum_{i\in\mcI : p_i^j\ge y} d_i^j, 
\end{equation}
with ``inverse"
% SELLER REVENUE 
\begin{equation}\label{revenue}
    f^j(z) \triangleq \sup\big\lbrace y\ge 0:
        \rho_i^j(y) \ge z \ \forall \ i \in \mcI\big\rbrace.
\end{equation}
For a given demand $\rho^j$, $f^j$ maps a unit of data to the highest price at
which it \emph{could} be sold to any buyer $i\in \mcI$ (as in \cite{semret}). However, $j$ is restricted by
the bid strategy of the buyers, as well as some natural constraints, and so we have the following Lemma.

{
% SELLER CONSTRAINTS
\lemma{(Seller constraints)}\label{constraints}
Let $j$ be a seller with budget $b^j$.
Formalizing the quantity threshold, the seller must satisfy the quantity
constraint, 
\begin{equation}\label{quantity}
    d^j \ge e_i(a)
\end{equation}
and
\begin{equation}\label{budget}
    \displaystyle\sum_{i\in\mcI} e_i^j(a) \le \sum_{i\in\mcI} d^j_i \le b^j.
\end{equation}
In addition, for a seller who does not sell at a loss, the reserve price must
satisfy 
\begin{equation}\label{reserveprice}
   p^j \ge \min_{i\in\mcI}\big({\theta_i}'(d_i^j)\big).
\end{equation}
Finally, we must have,
\begin{equation}\label{sumallocation}
    \displaystyle\sum_{k\in \mcI, k\ne i} e_k^j(a) \le a_i^j.
\end{equation}
}\\
\textbf{Proof:}
The first statement is an assumption, which we may enforce by (\ref{opt-out}),
and as a seller cannot sell more data than their bid, (\ref{budget}) enforces
the budget constraint for the seller. (\ref{reserveprice})
follows from the assumption that $j$ does not sell at a loss, and finally, (\ref{sumallocation})
follows from Proposition \ref{buyerstrategy}. 

% SELLER VALUATION
We may now define the seller $j$'s valuation (potential). 
{
\lemma{(Seller valuation)}\label{sellervaluation}
For any $i\in\mcI$, 
\begin{equation}\label{singlevaluation}
    \theta_i^j = \int_0^{e_i^j(a)} f^j(z) \ dz.
\end{equation}
it follows that 
\begin{equation}\label{valuation}
    \theta^j \circ e = \displaystyle\sum_{i\in\mcI}
 \int_0^{e_i^j(a)} f^j(z) \ dz.
\end{equation}
}
\textbf{Proof:} 
We assume that a seller will try to maximize profit for any given allocation $a$, and
as such there is an incentive for $j$ to sell all of its data, and so from Lemma
\ref{sellervaluation},  $\sum_{i\in{\mcI}} d_i^j = 
\sum_{i\in\mcI} e_i$,
The remainder of the proof follows as in \cite{semret}.

The sellers' natural utility is the potential
profit $u^j = \theta^j\circ e$, where $\theta^j$ is the
potential revenue from the sale of data composed with each buyers' opt-out value, $e_i(a)$. 
This is intuitive as a seller who does not meet the demand of \emph{any} buyer will not
participate in any auction. We have chosen to omit the original cost of the data
paid to the ISP, as a discussion of mobile data plans is outside the scope of this
paper. 

\subsubsection{Progressive User Strategy}
In order to to develop the seller strategy, it is necessary we determine 
that a seller has an incentive to
accept fractional bids (i.e. sell a fraction of their data $b^j$). Reasonably,
there may not exist a buyer such that $d_i = b^j$. In addition, a seller may accept
fractional bids in order to
maximize the revenue gained per unit of data by increasing market competition
\cite{???}.
Finally, in the case that the seller does not know the exact amount of
leftover data available, then they may only sell enough data to ensure that
they will not become a buyer while they submit their total data overage to the
secondary market. (FUTURE WORK? FROM \cite{zheng})

We describe the sellers' \emph{local} auction strategy for allocating its
data according to the constraints formed by the buyer strategy. As
local auction is progressive, and influenced by the $\epsilon$-best replies of
the buyers, we will need the following Lemma. (SHOULD I COUPLE THESE
ALGORITHMS? LESS OR MORE CONFUSING?)

% BUYER LOCAL RESPONSE
{
\lemma{(Buyer response)}\label{buyerresponse}
Define any auction duration to be $\tau \in [0,\infty)$. 
Fix any seller $j$, and fix all buyers bids $s_i^j=(d_i^j,p_i^j)$ at time $t\in\tau$. Let the
winner be determined and $s^j$ updated as in the seller strategy. As we cannot
assume that a buyers' bid will remain fixed at time $(t+1)$,
we address the following special cases:\\
(Budget limited allocation)
\begin{equation}\label{partialallocation}
    d_i^j > b^j - \sum_{k\in\mcI^j, k\ne i} d_k^j.
\end{equation}
(Equal price allocation)
There exists $k,i \in \mcI^j$ where $k\ne i$, such that
\begin{equation}\label{equalprice}
    \quad p_i^j = p_k^j.
\end{equation}
(Price-dominant allocation)
There exists $i \in \mcI^j$, $k \ni \mcI^j$, where
\begin{equation}\label{newbuyer}
    p_k^j < p_i^j \qquad \forall \ i \in \mcI^j.
    %p_i^j < p_k^j \quad \text{and} \quad d_i^j >= b^j, \quad \forall \ k \in \mcI^j.
\end{equation}
(Empty allocation)
\begin{equation}\label{emptyset}
    n > \vert \mcI\vert \Rightarrow \ell^j = \emptyset.
\end{equation}
In any of the above cases, seller $j$'s local auction strategy does not change and
Proposition \ref{sellerstrategy} holds.
}
\textbf{Proof:} \\
The first two cases, (\ref{partialallocation}) and (\ref{equalprice}) deal with
the problem of a buyer receiving a partial allocation:
In (\ref{partialallocation}), the winner with the lowest unit price $p_i^j$ results in a partial
allocation. In this case, using (\ref{buyercoordinate}), buyer $i$ will decrease its
quantity $d_i^j$ by increasing $n$, and will either remain a winner in $j$'s local
auction, or drop out of $\mcI^j$. (WILL USE LATER, DEVELOP MORE HERE?)
(\ref{equalprice}) considers that there may exist multiple buyers with the same unit price. The allocation rule (\ref{dataallocation}) determines that the data will be split proportinally
between all buyers with the same unit price. Therefore, again by
(\ref{buyercoordinate}), the buyers will increase $n$ until their demand is
met. (SHOULD I ALLOW FOR SPECIALIZED SPLIT BIDS FROM THE BUYERS? WILL IT WORK
WITH COORDINATED BIDS? MORE WORK NEEDED HERE...)\\
The third case considers that a new buyer $k$ may enter the market with a better
price. There cannot be a buyer where $d_k^j > b^j$, as this violates
(\ref{budget}) imposed by (\ref{buyercoordinate}). When $d_k^j =
b^j$, the optimal solution at time $t$ is the single allocation, therefore at
time $(t+1)$, we have $\mcI^j = k$. It follows that in the case $d_k^j < b^j$, buyers $i\in\mcI^j$ will
lose their position in $\ell^j$ as the auction progresses and the minimal
subset $\mcI^j$ is computed. 
(BUYERS ARRIVE AS A POISSOIN PROCESS? FUTURE WORK)
Lastly, to resolve (\ref{emptyset}), $j$ chooses $n=I$, and sets its reserve price to $p_I^j$. 

We now define the local auction, which we describe, when coupled with the buyer
responses, as a progressive game of strategy with incomplete, but perfect
information (SAY MORE?).
% SELLER LOCAL STRATEGY
{
\proposition{(Localized seller strategy (i.e. fractional allocation))}\label{sellerstrategy}
Define any auction duration to be $\tau \in [0,\infty)$. For any seller $j$,
For any seller $j$, fix all buyers bids $s_i^j=(d_i^j,p_i^j)$ at time $t\in\tau$.

let the winner be determined by,
\begin{equation}\label{winner}
    \bar{i} = \displaystyle\argmax_{I^j}\sum_{i\in I^j} p_i^j.
\end{equation}
Also define
$$
    \ell^j =\argmax_{{\mcI}' \subset \mcI, \vert{\mcI}'\vert =
n}\sum_{i\in{\mcI}'} p_i^j,
$$
where,
\begin{equation}\label{sellercoordinate}
    n = \argmin_{\ell^j}(i \in \ell^j : \displaystyle\sum_{i\in\mcI^j} d_i^j >
b^j),
\end{equation} 
which gives $j$ the minimal subset 
\begin{equation}\label{buyers}
    \mcI^j = \lbrace i \in \mcI : i < n\rbrace \subset \mcI
\end{equation} 
of buyers that maximizes $j$'s revenue.
Define buyer $i^* = n \le I$ in the ordered set $\ell^j$.
Then, for time $(t+1)$, set $j$'s reserve price as 
\begin{equation}\label{newprice}
    p^j = \theta_{i^*}'(d_{i^*}^j) + \epsilon.
\end{equation}
and update $j$'s budget to reflect the (tentative) allocation,
\begin{equation}\label{newbudget}
    \bar{b}^{j(t+1)} = b^{j(t)} - d_{i^*}^{j(t)}.
\end{equation}
Allowing $t$ to range over $\tau$, with buyer responses given as in Lemma
\ref{buyerresponse}, we have that (\ref{winner}) - (\ref{newbudget}) produces a local
%Assuming that no new buyers enter the market, we have that (\ref{winner}) - (\ref{newprice}) produces a local
market equilibrium.% from time $t$ to $(t+1) \in \tau$.
}\\
\textbf{Proof:}
We will assume that the seller has enough data to satisfy at least one buyer,
in the case of multiple buyers $i^*$ is the \emph{losing} buyer with the highest unit
price offer, determined by (\ref{sellercoordinate}). As the set of buyers is
computed at each iteration, we are guaranteed a subset $\mcI^j \subset \mcI$ that is 
at equilibrium as it is designed so that the data offered by
seller $j$ equals the aggregate demand of the buyers \cite{???}
Although the buyers' valuation $\theta_i$ is not known to the seller, we assume
that the buyer is bidding truthfully, and so $p^j = {\theta_i}' + \epsilon
=p_i^j + \epsilon$, therefore $p^j$ is known, and as $\mcI^j\subset \mcI$, we
note that (\ref{budget}) and (\ref{reserveprice}) hold. Now, using
(\ref{singlevaluation}), we have, $\forall \ z\ge 0$,
\begin{align*}
    \int_0^{e_{i^*}^{j}(a)} f^{j}(z)\ dz &\le\int_0^{e_i^{j}(a)}
f^{j}(z) \ dz 
\end{align*}
and so,
$$
    \theta^j\circ e_{i^*}(a) \le \theta^j\circ e_i(a),
$$
which holds $\forall \ i \in \mcI^j$.
It follows that, using
%, using (\ref{buyercoordinate}), 
the definition of an $\epsilon$-best reply $s_i^j = (v_i^j, w_i^j)$,  
for any $\epsilon$-best reply, 
$$
    p^j \le {\theta_i}'(v_i^j) + \epsilon,
$$
$\forall i \in \mcI^j$.
Therefore the choice of $p^j$ does not force any buyers out of the local
equilibrium. Thus we determine the valuation between seller
$j$ and buyer $i$ is well-posed, the reserve price (\ref{newprice}) is justified, and that the local
equlibirium created by $j$ is stable from time $t$ to $(t+1)$. 
(NEED TO SAY MORE HERE... INDUCTION? HOW DOES THE TENTATIVE AWARD WORK? SAVE
FOR NASH EQ PROOF? KEEP WITH t TO t+1?)

% MECHANISM
\subsection{Data Auction Mechanism}\label{mechanism}
We may now proceed to formally define the PSP auction as a composition of the
buyers and sellers in the secondary market. 
The market price function (MPF) for a buyer in the secondary market
can now be described as follows:
% NEW MARKET PRICE
\begin{align}\label{dataprice}
\begin{split}
    \bar{P}_i&(z, s_{-i}) =\displaystyle\sum_{j\in\mcI_i}P_i^j(z_i^j,
s_{-i}^j) \\
    &= \sum_{j\in\mcI_i}\bigg(\inf\bigg\lbrace y\ge 0 : 
    {\bar{D}_i^j}(y,s_{-i}^j)\bigg\rbrace \bigg),\\
\end{split}
\end{align}
where
% NEW ALLOCATION RULE
\begin{align}
\begin{split}\label{datacomposed}
    \bar{D}_i^j(y,s_{-i}^j) &= D_i^j(y,s_{-i}^j)\ \circ\ e_i\\
    &= \bigg\lbrack d^j - \sum_{p_k^j> y} d_k^j\bigg\rbrack^+/\vs_i^j,
\end{split}
\end{align}
and is interpreted as the aggragate of minimum bid prices of buyer $i$ for bids
over $\mcI_i$ given an opponent profile $s_{-i}$. In this expression, we note that
the price is an aggragation of the \emph{individual} prices of the buyers as it is
possible that $\mcI_i \ne \mcI^j$, and so the reserve prices of the individual sellers
may vary. 
The inverse function
% NEW INVERSE DEMAND
$\bar{D}_i$ is defined as
\begin{align}\label{datapriceinverse}
\begin{split}
    \bar{D}_i(y, s_{-i}) &= \displaystyle\sum_{j\in\mcI_i}\bigg(\sup\bigg\lbrace z\in \big( 0,
\bar{D}_i^j \big) : \\
    &\bar{P_i}(z,s_{-i}^j) < y\bigg\rbrace\bigg).
\end{split}
\end{align}
We may interpret $\bar{D}_i(y, s_i)$ as the maximum quantity provided by the subset
$\mcI_i$ at a bid price of $y$ given $s_{-i}$. 
% DATA ALLOCATION RULE
The data allocation rule is given as,
\begin{align}\label{dataallocation}
\begin{split}
    \bar{a}_i(s) &= a_i(s) \circ e_i =
\displaystyle\sum_{j\in\mcI_i}\bigg(\min\bigg\lbrace d_i^j/\vs_i^j,
\\
    &\frac{d_i^j}{\sum_{k:p_k^j=
p_i^j}d_k^j}
\bar{D}_i^j(p_i^j,s_{-i}^j)\bigg\rbrace\bigg)
\end{split}
\end{align}
with cost to the buyer,
% DATA COST
\begin{align}\label{datacost}
\begin{split}
    \bar{c}_i(s) &= \displaystyle\sum_{j\in\mcI_i} 
p^j \bigg(\bar{a}_j(0; s_{-i}^j)
    -\bar{a}_j(s_i^j;s_{-i}^j)\bigg).
\end{split}
\end{align}


The main contribution of this work is comprised by the modified PSP rules
along with the user strategies, which we use to
describe an auction mechanism inspired by the classic PSP throughput problem. In order to apply PSP to the data-sharing secondary market described in
\cite{zheng} the idea of a ``route" is repurposed, and by the use of the
opt-out buyer, we are abel to achieve a market equilibrium. We claim that in
the secondary market our
formulation not only holds the desired VCG qualities, but minimizes the message
space and auction duration, resulting in a convergence time (FIND COMPETITVE RATIO?)
with respect to the classic throughput problem.


\subsection{VCG Formulation}
Consider a user seeking to prevent
data overage by purchasing enough data from a subset of other network users.
This user $i$ can be modeled as a opt-out buyer where, as in \cite{semret},
$\vs_i^j$ corresponds to a fraction of user $j$'s data aquired by user $i$. In order to form the
distributed auction, we set $1$ if seller $j$ has enough data to
satisfy $i$'s demand, and $\vs_i^j=d_i$ otherwise. We intend to show that this
does not affect their valuation, and indeed, in this network setting, results in a shared network optima (a
global optimum). The formulation is inspired to the thinnest allocation route for
bandwidth given in \cite{lazar}. We note that if a single seller $j$ can
satisfy $i$'s demand, then
(\ref{opt-utility}) reduces to the original form, defined in
\cite{semret} as ``a simple buyer at a single resource element".

The sellers' auction will function as follows: at each bid iteration all buyers
submit bids, and the winning bid is the buyer $i$ that has the highest price
$p_i^j$. The seller allocates
data to this winner, at which point all other buyers are able to bid again, and
the winner leaves the auction (with the exception where multiple bidders bid
the same price, where (\ref{dataallocation}) determines they will not fully
satisfy their demand, and so we will assume they remain in the auction). The auction progresses as such until all the
sellers' data has been allocated. We design an algorithm based on the sellers'
fractional allocation strategy.

% SELLER ALGORITHM
\begin{algorithm}[H]
\caption{(Seller fractional allocation)}
\begin{algorithmic}[1]
\State $p^{j(0)} \gets \min_{i\in\mcI^j} p_i^j$
\State $s^{j(0)} \gets (p^j, b^j)$
\While{$b^j > 0$}
\State $\mcI^{j(t+1)} = \mcI^{j(t)}\setminus \lbrace i \in \mcI^{j(t)}: d_i^j >
\bar{b}^{j(t)}\rbrace$
\State $ i^* \gets \displaystyle\argmax_{I^j}\sum_{i\in I^j} p_i^j$ 
\State $\bar{b}^{j(t+1)} \gets \bar{b}^j - d_{i^*}^{j}$
%\State $p^j \gets \theta_{i^*}'(d_{i^*}^j)\circ e_i$
\State $p^j \gets p_{i^*}^j+\epsilon/2$ and $d^j \gets \bar{b}^j$
\State $s^{j(t+1)} \gets (d^j, p^j)$
\If{$\ \exists \ i : s_i^{j(t+1)} \ne s_i^{j(t)}$}
\State $\bar{b}^{j(t+1)} = \bar{b}^{j(t)}$
\State Go to 4.
\Else
\State $i^* \gets \bar a_i(s)$
%\State $\beta^{j(t+1)} \gets \bigg\lbrace i\in\mcI^j: 
%        d_i \le \bar{b}^j \bigg\rbrace$
\State Go to 4.
\EndIf
\EndWhile
\end{algorithmic}
\end{algorithm}
We assume that each time that $s^j$ is updated it is shared with all
participating buyers. At this point buyers have the opportunity to bid again,
where a buyer that does not bid again is assumed to hold the same bid, since a
buyer dropping out of the auction will set their bid to $s_i^j=(0,0)$. As we will show in our analysis, the buyers are bidding truthfully; the
algorithm makes use of the fact that the sellers' valuation is determined by
the buyers' market and upholds the PSP mechanism. (CHECK)

(BUYER ALGORITHM? WHY NOT..., CLEANER)

% VCG ANALYSIS
\section{VCG Analysis}\label{analysis}
\subsection{Equilibrium}

Consider an opt-out buyer $i\in\mcI$. 
(I THINK WE NEED PRICE TO COME IN SOON... MODIFIED ALLOCATION RULE!)
Due to (\ref{opt-utility}), $i$ only has an incentive to change its bid
quantity if it increases its opt-out value $e_i$. 
We show that, without loss of utility, $i$ can coordinate its
bid quantities $d_i$ to the level where the opt-out value $e_i$ 
 is the same for each qualifying seller $j \in\mcI_i$. We argue
that without loss of utility, $i$ may choose a seller pool where a minimal subset
of sellers guaranteed (CHECK) to satisfy demand where $i$ is able to play a
``consistent" strategy and still have feasible best replies, and buyer
coordination holds in the secondary data market under our assumptions.

% BUYER COORDINATION
{
\lemma{(Opt-out buyer coordination)}\label{coordinationlemma} 
Let $i\in\mcI$ be a opt-out buyer.
For any profile $s_i = (d_i, p_i)$, let $a_i \equiv \sum_j a_i^j(s)$ be the resulting data
tranfer. For a fixed $s_{-i}$, a better reply for $i$ is $x_i =
(z_i,p_i)$, where $\mcI_i$ is computed as in the buyer strategy,
$$
    z_i^j = e_i(a)
$$
and
\begin{equation}\label{coordination}
    a_i^j(z_i,p_i) = z_i^{j^*},
\end{equation}
}\\
\textbf{Proof:}
As $s_{-i}$ is fixed, we omit it, in addition, we will use $u\equiv u_i \equiv u_i(s_i) \equiv
u_i(s_i;s_{-i})$. 
We have, by (\ref{opt-out}) and (\ref{datacomposed}), $\forall \ j\in\mcI_i$,
\begin{align*}
    z_i^{j^*} &= z_i^j = e_i(a) = e_i^{j^*}(a) \\
    &\le \bigg\lbrack d^{j^*} - \sum_{p_k^j> y}
d_k^{j^*}\bigg\rbrack^+/\vs_i^{j^*},
\end{align*}
and so
$$
     a_i^j(z_i,p_i) = a_i^{j^*}(z_i,p_i) = z_i^{j^*}= e_i(a).
$$
In order to determine that $i$ has no loss of utility, we will show that
$$
    u_i(d_i,p_i) \le u_i(z_i,p_i).
$$
We address two cases:\\
1. \emph{There exists a seller who can fully satisfy $i$'s demand.} \\
In this case, $\vert\mcI_i\vert = 1$, and the case is trivial as no
coordination is necessary for a single bid.\\
2. \emph{Buyer $i$'s demand can only be satisfied by a minimal subset of sellers.} \\
Buyer $i$ maintains ordered set $\ell_i$ where the sellers with the
largest bid are considered first, the seller $j^*$ defines the minimal subset $\mcI_i$ where a
coordinated bid is possible. 
From (\ref{dataallocation}) and (\ref{coordination}), we have that, $\forall \ j\in \mcI_i$,
\begin{align*}
    e_i^j(z_i,p_i) &= \bigg\lbrack d^j - \sum_{p_k^j> y}
d_k^j\bigg\rbrack^+/\vs_i^j 
\end{align*}
We have $e_i^j(a(z_i,p_i)) = a_i^{j^*}(z_i,p_i)/\vs_i^{j^*} = e_i(a)$, which
implies that 
$$
\theta_i\circ e_i(a(z_i,p_i)) = \theta_i\circ e_i(a).
$$ 
Therefore, by the definition of utility (\ref{buyerutility}),
\begin{align*}
    \theta_i\circ& e_i(a(z_i,p_i)) - \theta_i(a)) \\
    &=  c_i^j(s) -c_i^{j^*}(z_i,p_i).
\end{align*}
Now using the definition of
the buyers' valuation (\ref{buyervaluation}), we have $\forall \ i \in\mcI$,
\begin{align*}
    u_i(z_i,p_i) &- u_i \\
    &=\displaystyle\sum_{\mcI_j}\bigg(\theta_i(a_i^j)-
\theta_i(a_i^{j^*}(z_i,p_i))\bigg) 
\end{align*}
Now, from (\ref{dataallocation}), $\forall \ j\in\mcI_i$, $a_i(z_i,p_i) \le
z_i^j \le a_i^j$ and $\theta_i\ge 0 \Rightarrow u_i(z_i,p_i) - u_i \ge 0,
\ \forall \ i\in \mcI$, as is shown by the definition of the buyers' utility,
(CAN USE THIS! VERY STRONG)
$$
     \sum_{\mcI_j}\bigg(\frac{\sigma (a_i^j)^{1-\alpha}}{1-\alpha}
-\frac{\sigma (a_i^j(z_i,p_i))^{1-\alpha}}{1-\alpha}\bigg)
     \ge 0.
$$

Additionally, since (\ref{opt-out}) does
not increase the demand of seller $i$, $\sum_j c_i^j(s) \le b_i$, i.e. $x_i$ is feasible.
In effect, we are using the buyer demand
to partition the auction space, thereby 
controlling (optimizing?) the message space (EXPLAIN) for the
ISP, and providing an optimal market space to host the buyers and sellers based
on their type. \\

\subsubsection{Incentive Compatibility}

We proceed to claim that the optimality of truth-telling holds in our
formulation, where the market functions as a hybrid of \cite{semret} and
\cite{zheng}. The opt-out buyers' market is comprised of the minimal subset of
sellers with the largest budgets, described in the buyer strategy as $\mcI_i$.
To achieve incentive compatibility, we find that the opt-out buyer must choose this subset so that
its overall marginal value is greater than its market price.
The buyers' market price is calculated as the sum of reserve prices
of the sellers in the opt-out buyers' pool.
the market prices at the %different auctions, weighted by the data provisioning vector values.
The actual bids are
obtained from the opt-out buyers' strategy. The quantity to bid is given
by the auction mechanism, i.e. (\ref{dataprice}) and (\ref{datacomposed}), as
the maximim possible quantity of data that a buyer $i$ can bid over its seller
pool while maintaining a marginal valuation greater than the aggragate of
minimum prices maintained by the sellers in $i$'s pool.
As with a single resource, \cite{semret} and
\cite{lazar}, we show that truth-telling is optimal for the buyer, i.e. in each auction, the buyer sets
the bid price to the marginal value.
% BUYER INCENTIVE COMPATIBILITY
{
\proposition{(Buyer incentive compatibility)}\label{buyerincentivecompatibility}
Let $i\in\mcI$ be an opt-out buyer, and fix all other buyers' bids
$s_{-i}$, as well as the sellers' bids $s^j$ (so $a_i$ is fixed). \\
Let 
\begin{align}
\begin{split}
    z_i &= \sup\bigg\lbrace h\ge 0 : 
 {\theta_i}'(h) > \bar{P}_i^j(h)\bigg\rbrace, 
\end{split}\\
\begin{split}
    \chi_i &= \sup\bigg\lbrace h\ge 0: 
\displaystyle\int_0^h 
    \bar{P}_i(h) \ dh \le b_i\bigg\rbrace,
\end{split}
\end{align}
$e = \min(z_i, \chi_i - \epsilon / \theta_i'(0))^+$, and for each $j \in
\mcI_i$, 
$$
    v_i^j = e = v_i^{j^*}
$$
and 
$$
    w_i^j = w_i^{j^*} = \theta_i'(e).
$$
Then a (coordinated) $\epsilon$-best reply for the opt-out buyer is $t_i =
(v_i,w_i)$, i.e., $\forall \ s_i, u_i(t_i;s_{-i}) + \epsilon \ge u_i(s_i;
s_{-i})$.
} \\ \\
\textbf{Proof:} For ease of notation, we may assume that bid
$s_i^j=s_i^{j^*}$ and that each $j\in\mcI_i$.
First suppose $e = z_i$. Since $\theta_i'$ is non-increasing
and, $P_i^j$ is non-decreasing, (\ref{dataprice}) implies
$\theta_i'(e) >\sum_j P_i^j(v_i^j)$, due to $e$ being supremum, i.e.
greater than the marginal value, same as in \cite{semret}. This,
along with $i$'s strategy, gives $\forall \ y,z \ge 0$,
\begin{align*}
    \displaystyle\sum_{j} y &> \bar{P_i}(z,s_{-i}) \\
    &\Rightarrow y > P_i^j(z,s_{-i})\\
    &\Rightarrow
    z \le D_i^j(y,s_{-i})/\vs_i^j,
\end{align*}
and so,
\begin{align}\label{figureout}
\begin{split}
    w_i^j & > {P}_i^j(v_i^j) \\
    &\Rightarrow {D}_i^j = \frac{\lbrack d^j - d^j(w_i^j)\rbrack}{\vs_i^j}.
\end{split}
\end{align}
Now, let $y_i = {\theta_i}'(z_i)$ and suppose $z_i = 0$, then $v_i=0 \Rightarrow
a_i(t_i; s_{-i})=0$ and $i$'s coordinated bids (Lemma \ref{coordinationlemma})
gives $\bar{D}_i(0,s_{-i}) =0$. 
By choice of $i$'s valuation function $\theta$,
$$
w_i = {\theta_i}'(v_i) =\sigma v_i^{-\alpha} >  \sigma
z_i^{-\alpha} = {\theta_i}'(z_i) = y_i,
$$
and since $D_i^j$ is non-decreasing, $D_i^j(y_i,s_{-i}) \ge z_i \ge v_i$.
Thus, by (\ref{dataallocation}) and (\ref{figureout}),
\begin{align*}
    a_i^j(t_i; s_{-i}) = v_i^j \\
    e_i^j \circ a(t_i;s_{-i}) = e.
\end{align*}
Therefore,
\begin{align*}
    &\qquad u_i(t_i;s_{-i}) \\
    &= \displaystyle\int_0^\epsilon {\theta_i}'(\eta) \ d\eta -
\sum_j \int_0^{v_i^j} P_i^j(z) \ dz \\
    &= \int_0^\epsilon {\theta_i}'(\eta) \ d\eta - \sum_j\int_0^\epsilon
P_i^j(\eta/\vs_i^j) \ d\eta.
\end{align*}

Now suppose $\exists \ s_i = (d_i, p_i)$ such that $u_i(s_i;s_{-i}) > u_i(t_i;
s_{-i}) + \epsilon$. Let $\xi = \min_k e_i^k\circ a_i^k(s)$, and $\forall \ j,
\zeta_i^j = \xi/\vs_i^j$ and $s_{i^*} = (\zeta_i,p_i)$. Then from
(\ref{coordination}), $a_i^j(s_{i^*}; s_{-i}) = \zeta_i^j$, therefore
\begin{align*}
    &\qquad u_i(s_{i^*};s_{-i}) \\
    &= \displaystyle\int_0^\xi {\theta_i}'(\eta)\ d\eta -\sum_j\int_0^\xi P_i^j(\eta/\vs_i^j) \ d\eta.
\end{align*}
By Lemma \ref{coordinationlemma}, $u_i(s_{i^*}, s_{-i}) > u_i(t_i; s_{-i}) +
\epsilon$, which is equivalent to 
$$
    \int_\epsilon^\xi {\theta_i}'(\eta) \ d\eta - \sum_j\int_0^\xi
P_i^j(\eta/\vs_i^j) \ d\eta > \epsilon.
$$
The rest of the calculation follows as in \cite{semret} with the modified
framework, i.e. both $\bar\epsilon = \epsilon + \epsilon/{\theta_i}'(0)$ and
$\xi > \bar\epsilon$ show a contradiction.
(DO I NEED TO SHOW THE CONTRADICTION?)

We address the case where $e=\chi_i$, that is, the seller $j^*$'s data price is
equal to the budget of buyer $i$ under the strategy given in
(\ref{opt-out}), similarly to \cite{semret}, in this case any bid $s_{i^*}$
where $u_i(s_{i^*}) > u_i(t_i)$ is not feasible, as the buyer would go over its
budget.

Finally, suppose that $j^* = I$, that is, $\mcI_i = \mcI$. 
In this case, if $u_i(s_{i^*}) > u_i(t_i)$, then
the demand cannot be satisfied by the sellers and the bid is not feasible. 

We proceed to examine the strategy of the seller. As we have shown, the seller
is a functional extension of the buyer, with rules determined by the buyers'
behavior. As we have shown, the buyers are bidding truthfully, according to the
rule of incentive compatibility. Seller $j$'s reserve price is determined by a
buyer $i \in \mcI \setminus \mcI^j$, and therefore, even if this price is zero,
then $p^j = \epsilon \ge 0$.
We argue that if truthfullness holds \emph{locally} for both buyers and sellers, i.e. $p_i ={\theta_i}' \
\forall \ j \in \mcI_i$ and $p^j = {\theta^j}' \ \forall \ i \in\mcI^j$, then there exists a market
equilibrium. We have the following Lemma.

{
\lemma{(Incentive compatibility in local auctions)}
For any seller $j$, let time $t \in \tau$ be fixed, as well as bids from $j$'s
seller pool, so that $s_i$ is fixed $\forall \ i \in\mcI^j$. The
``$\epsilon$-best" reply for seller $j$ is,
$$
    s^j = (b^j, p^j),
$$
defined by Proposition \ref{sellerstrategy}.
%and fix $j$'s reserve price $p^j$ as is defined by (\ref{newprice}). 
The sellers' ``truthful" strategy, combined with Proposition
\ref{buyerincentivecompatibility}, forms a
``truthful" local game embeddeded within $j$'s auction, and an equilibrium of
the embedded game is an equlibrium point for the local auction. 
}\\
\textbf{Proof:}
We have that $p^j$ is the lowest price that $j$
will accept to perform any allocation. By the definition of ``reserve
price", $j$ is willing to buy its own budget $b^j$ at price $p^j$. This forms a
``truthful" local game embeddeded within $j$'s auction with strategy space
restricted to $\epsilon$-best replies from buyers $\in \mcI^j$. Therefore we have that a
fixed point in the ``truthful" local game is a fixed point for the auction.
To see this, we observe that from (\ref{datacomposed}), $\forall \ i \in \mcI^j$, $\bar{D}_i^j(y, s_{-i}) =
0 \ \forall \ y < p^j$, and so $y=0 \le \epsilon \Rightarrow e_i^j(a) = 0$. We
further argue that by Lemma \ref{buyerresponse}, and as the set $\mcI^j$ is
computed at each bid iteration, that our result holds for time $(t+1) \in \tau$.
(NEED MORE? EACH CASE IN 4.3?)

{
\lemma{(Continuity of $\epsilon$-best reply)} (NEED A LEMMA?)
}

% DATA NASH EQ
{
\theorem{(Data Nash Equilibrium)}
Using the rules of the data auction mechanism, the secondary market described in
\cite{zheng} converges to a $\epsilon$-Nash equilibrium. In the network auction
game with the data-PSP rules applied independently by each user, and all users applying
their respective strategies, the secondary market converges to an $\epsilon$-Nash
equilibrium. 
}\\
\textbf{Proof:}
1. selling off data piecewise over time (MAYBE?), \\
2. using the min price of sellers in the auction i.e. ${\theta_i}'(d_{i^*}^j) =
p^j$ is OK, \\
3. that bids are still feasible AND optimal \\
4. the algorithm achieves global economic equilibrium)\\
\textbf{NEED TO COVER:}\\
1. Change in buyer valuation \\
2. New buyers\\
3. Not enough buyers \\
4. Not enough data\\
\textbf{TRY:}\\
Sellers only act when the resources obtained by the buyers influence their
respective reserve prices, which agrees with the seller stragety of attempting
to sell their data in the first iteration. Therefore we claim there exists a
market stability and therefore, the existence of a Nash equilibrium. As the
valuation of the sellers is derived by the demand of the buyers, who are
bidding equivalent bids over a minimum subset of buyers, we claim that the seller
strategy, along with the seller constraint (\ref{??}) results in a global
market equilibrium.
We have shown that the local
equlibirium created by $j$ is stable from time $t$ to $(t+1)$. 
Now, suppose that buyer $i^*$ computes its best response $s_i^j = (v_i^j, w_i^j)$
Finally, suppose that
a buyer $k$ enters the market such that for some buyer $l\in\mcI^j$,
$$
    \displaystyle\sum_{i\in\mcI^j} p_i^je_i^j(a) + p_k^je_k^j(a) \ge
\sum_{i\in\mcI^j} p_i^je_i^j(a) - p_l^je_l^j(a),
$$
that is,



NOTES: (today)\\
1. switch from defined valuation function\\
2. finish seller incentive compat\\
3. work on progression\\
4. check reserve price = monopoly price


\subsection{Efficiency}
(NEED OWN WORDS)
The objective in designing the auction is that, at equilbrium, resources al-
ways go to those who value them most. Indeed, the PSP mechanism does
have that property. This can be loosely argued as follows: for each player,
the marginal valuation is never greater than the bid price of any opponent
who is getting a non-zero allocation. Thus, whenever there is a player j
whose marginal valuation is less than player i 's and j is getting a non-zero
allocation, i can take some away from j , paying a price less than i 's marginal
valuation, i.e. increasing u i , but also increasing the total value, since i 's
marginal value is greater. Thus at equilibrium, i.e. when no one can unilat-
erally increase P their utility, the total value is maximized.

\subsection{Convergence}

\end{multicols}

\begin{thebibliography}{9}

\bibitem{zheng}
L. Zheng, C. Joe-Wong, C. W. Tan, S. Ha and M. Chiangs, 
\textit{Secondary markets for mobile data: Feasibility and benefits of traded
data plans}, 2015 IEEE
Conference on Computer Communications (INFOCOM), Kowloon, 2015, pp. 1580-1588.

\bibitem{lazar}
A. A. Lazar and N. Semret, 
\textit{“Design and Analysis of the Progressive Second Price Auction for Network
Bandwidth Sharing,”} Telecommunication Systems, Special Issue on Network Economics, 2000.

\bibitem{semret} 
N. Semret, 
\textit{“Market Mechanisms for Network Resource Sharing,”
Ph.D. thesis}. 
Columbia University, 1999.

\bibitem{tuffin}
Bruno Tuffin,
\textit{Revisited Progressive Second Price Auction for Charging
Telecommunication Networks}.
[Research Report] RR-4176, INRIA, 2001.

\bibitem{qpsp}
Clare W. Qu, Peng Jia, and Peter E. Caines,
\textit{Analysis of a Class of Decentralized Decision Processes: Quantized
Progressive Second Price Auctions},
2007 46th IEEE Conference on Decision and Control, New Orleans, LA, 2007, pp.
779-784.

 
\end{thebibliography}


\end{document}

