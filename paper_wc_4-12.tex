\documentclass[12pt]{article}
 
\usepackage[text={6in,8.1in},centering]{geometry}

\usepackage{enumerate}
\usepackage{amsmath,amsthm,amssymb}
\usepackage{mathrsfs} % to use mathscr fonts

\usepackage{multicol}
\usepackage{url}

\usepackage{epstopdf}
\usepackage{caption,subcaption}
\usepackage{pstricks}
\usepackage{pst-solides3d}
\usepackage{pstricks-add}
\usepackage{graphicx}
\usepackage{pst-tree}
\usepackage{pst-poly}
\usepackage{calc,ifthen}
\usepackage{float}\usepackage{multicol}
\usepackage{multirow}
\usepackage{array}
\usepackage{longtable}
\usepackage{fancyhdr}
\usepackage{algorithmicx}
\usepackage[noend]{algpseudocode}
\usepackage{changepage}
\usepackage{color}
\usepackage{listings}
\usepackage{fancyvrb}
\usepackage{verbatim,moreverb}
\usepackage{courier}
\usepackage{algorithm}

\lstset{ %
language=C++,               
basicstyle=\footnotesize,
numbers=left,                  
numberstyle=\tiny,     
stepnumber=1,         
numbersep=5pt,         
backgroundcolor=\color{white},  
showspaces=false,               
showstringspaces=false,         
showtabs=false,                 
columns=fullflexible,
frame=single,          
tabsize=2,          
captionpos=b,       
extendedchars=true,
xleftmargin=17pt,
framexleftmargin=17pt,
framexrightmargin=17pt,
framexbottommargin=4pt,
breaklines=true,       
breakatwhitespace=false, 
escapeinside={\%*}{*)}       
}

\newenvironment{block}{\begin{adjustwidth}{1.5cm}{1.5cm}\noindent}{\end{adjustwidth}}

\newtheorem{proposition}{Proposition}[section]
\newtheorem{theorem}{Theorem}[section]
\newtheorem{lemma}{Lemma}[section]
\newtheorem{corollary}{Corollary}[section]
\theoremstyle{definition}
\newtheorem{definition}{Definition}[section]

 
\def\verbatimtabsize{4\relax}
\def\listingoffset{1em}
\def\listinglabel#1{\llap{\tiny\it\the#1}\hskip\listingoffset\relax}
\def\mylisting#1{{\fontsize{10}{11}\selectfont \listinginput[1]{1}{#1}}}
\def\myoutput#1{{\fontsize{9}{9.2}\selectfont\verbatimtabinput{#1}}}

\newcommand{\vs}{\varsigma}
\newcommand{\mcL}{\mathcal{L}}
\newcommand{\mcI}{\mathcal{I}}
\DeclareMathOperator*{\argmax}{arg\,max}
\DeclareMathOperator*{\argmin}{arg\,min}
 
\headsep25pt\headheight20pt
 
 
\pagestyle{fancyplain}
\rhead{\fancyplain{}{\small\bfseries Blocher, Jordan}}
\cfoot{\ \hfill\tiny\sl Draft printed on \today}
 
 
\setlength{\extrarowheight}{2.5pt} % defines the extra space in tables
 
\begin{document}
\begin{multicols}{2}

\section{Abstract}

We investigate the 2cm app, a data-exchange platform published for China Mobile Hong
Kong 4G Pro Service Plan customers. Telecommunications ISPs' revenue is typically
gained by charging users a fixed fee for a maximum amount
of data usage in a month, i.e., a monthly data cap \cite{???}. 
2cm's (2nd exchange market) data exchange platform allows users to submit
bids to buy and sell data.
This usage model, is, as far as our knowledge, the first data trading
platform that allows customers to buy and sell their own data.
We describe a distributed auction mechanism for data exchange inspired by the
classic PSP throughput problem,
and prove that our distributed data exchange mechanism provides incentive
compatibility (social choice function), and that we have efficiency
using only partial valuation information of each participant in an exchange market.

In applying a distributed PSP implementation to CMHK's secondary market, we
find that
the market is able to achieve an equilibrium as the sellers and buyers have an
incentive for a collaborative exchange, and design our mechanism tp provide the functions
for effective communication between the connected users. We claim that in
this secondary market our formulation holds the desired VCG qualities through the construction of a
probable equilibrium \cite{???}. We further provide bounds on the auction duration, 
with respect to the classic throughput problem.
and provide simulated results on convergence time to support our (FIND
COMPETITVE RATIO!), and a bound
on the convergence of our mechanism. We extend the works of cite!cite!
i.e. (market influence/EQ,social EQ,payment/allocation models) OR (bandwidth, data bundles, distributed
market algorithms) and show the existence of a dynamic global market equilibrium, allowing for a unique set of market dynamics.



\section{Introduction}

In this work, we propose \emph{a distributed progressive second price (PSP) auction in
order to maximize social utility in this secondary market}. Using the
distributed PSP mechanism on CMHKs data
exhange platform, we show that for cellular data allocated between multiple users there
exists an $\epsilon$-Nash market equilibria. A quality of the PSP auction is that
demand information is not known centrally, rather, it is distributed in the
buyers' valuations. The mechanism for an auction is defined
as \emph{distributed} when the allocations at any element depend only on
\emph{local} state: the quantity offered by the seller at that element, and the
bids for that element only \cite{lazar}. In this work, the proposed mechanism
allows the distribution of bids, where there are many ISPs each holding
thier own local auction; there is no entity that holds a global market knowledge.

In a PSP mechanism, bids consist of (i) an available (required) quantity and (ii) a unit-price (calculated
using its own demand functions). Buyers submit bids cyclically until an
($\epsilon$-Nash) equilibrium is reached and a local auction is concluded. 

(FEE IS FIXED OR
PER-UNIT?)(HOW DO WE MODEL ISP REVENUE? IMMEDIATE FUTURE)

The form of the auction mechanism presented here is (CAN BE? NEED TO SHOW TO
CLAIM 'IS') described as a
pure-strategy progressive game with incomplete, but perfect information. (WHAT
DOES NASH SAY ABOUT THIS?) (TRY MIXED? CAN ONLY HAVE MIXED WITH A DISTRIBUTED
VALUATION.. FUTURE WORK) 

The paper is organized as follows...

\subsection{Distributed Progressive Second Price Auctions}

Progressive second price auctions (PSPs) were proposed
in \cite{lazar}, \cite{diffserve} to provide a dynamic network service pricing
scheme to provide consistent services for network bandwith users.
\cite{diffserve} conducts a game theoretic analysis, deriving
optimal strategies for buyers and brokers, and further shows the existence
of networkwide market equilibria based on their game-theoretic
model. Constructing necessary and sufficient conditions for the stability
of the game allows the sustainability of any set
of service level agreement configurations between Internet service
providers to be determined.

We begin with a brief introduction to the distributed PSP auction for bandwith
sharing, first introduced by Lazar and Semret \cite{lazar}.
We define a set of $\mcI = \lbrace 1,\cdots,I\rbrace$ network bandwidth users.
Suppose each user $i \in \mcI$ makes a bid $s_i^j = (p_i^j, d_i^j)$ to the
seller of resource $j$, where $p_i^j$ is the unit-price the user is willing to
pay and $d_i^j$ is the quantity the user desires. The \emph{bidding profile} forms a grid, $s \equiv
[s_i^j] \in \mcI \times \mcI$, and $s_{-i} \equiv [s_1^j , \cdots , s_{i-1}^j , s_{i+1}^j , \cdots
, s_I^j]_{j\in\mcI}$ is the profile of user $i$'s opponents. 
Using this classic PSP mechanism, \cite{lazar} shows that given the opponents
bids $s_{-i}$,
user $i$'s $\epsilon$-best response to seller $j$ is $s_i^j = (w_i^j, v_i^j)$
and is a Nash move
where $\epsilon > 0$ is the bid fee, $B_i =\sum_{j\in\mcI} b_i^j$ is user $i$'s
budget, and every user has an elastic demand function.
Based on the profile of bids $s^j = [s^j_1, \cdots , s^j_I]$, the seller applies
an allocation rule $a(s^j) = [a_1^j, \cdots , a_I^j]$, where $a^j_i$ is the quantity allocated
by $j$ to each user $i\in\mcI$ and $c^j_i$ is the cost charged to $i$ for
allocations awarded in auction $j$. 
An allocation is considered feasible if $a^j_i \le d_i^j$, and $c^j_i \le  p^j_i d_i^j$.

% MECHANISM
\subsubsection{The Mechanism}\label{mechanism}
The PSP auction as given in \cite{lazar} and \cite{semret} is designed for the
problem of network bandwidth allocation, and is analyzed as a noncooperative
game where $i\in\mcI$ agents buy the
fixed amount of bandwidth $d_i^j$ from sellers $j\in\mcI$.
The market price function (MPF) for a buyer-seller pair is,
% NEW MARKET PRICE
\begin{align}\label{dataprice}
\begin{split}
    P_i^j(z, s_{-i})= \inf\bigg\lbrace y\ge 0 : 
    d_i^j - \sum_{p_k^j>y,k\ne i} d_k^j \ge z \bigg\rbrace,\\
\end{split}
\end{align}
and is the of minimum prices a user bids in
order to obtain bandwidth $z$ given opponent profile $s_{-i}$. 
The maximum available quantity of data in auction
$j$ at unit price $y$ given $s_{-i}^j$ is,
% NEW INVERSE DEMAND
\begin{align}\label{datapriceinverse}
\begin{split}
    D_i^j(y, s_{-i}) &= \bigg\lbrace D^j - \sum_{p_k^j>y,k\ne i} d_k^j  \bigg\rbrace,
\end{split}
\end{align}
where $D^j$ is the total amount of bandwith that user $j$ has to offer.
% DATA ALLOCATION RULE
For each $i \in \mcI$, the allocation from auction $j$ is,
% and so for profile $s_i$, for any $i \in \mcI$,
\begin{align}\label{dataallocation}
    a_i^j(s) &= \min\bigg\lbrace d_i^j, 
    \frac{d_i^j}{\sum_{p_k^j= p_i^j}d_k^j} D_i^j(p_i^j,s_{-i}^j)\bigg\rbrace.
\end{align}
Finally, we have the cost of the allocation,
% DATA COST
\begin{align}\label{datacost}
    c_i^j(s) = \displaystyle\sum_{k\ne i} 
p_k^j \big[a_k^j(0; s_{-i}^j)
    -a_k^j(s_i^j;s_{-i}^j)\big].
\end{align}

It was shown, in \cite{lazar}, that the mechanism may converge to a Nash market
equilibria for differentiated services
allocated between multiple agents when all players bid their real marginal valuation
of the bandwidth resource. In other words, the PSP constraints are
sufficient to attain the desirable property of truthfullness through incentive
compatibility. The pricing mechanism upholds the \emph{exclusion-compensation
principle}, user $i$
pays for its allocation so as to exactly cover the ``social opportunity cost"
which is given by the declared willingness to pay (bids) of the users who are
excluded by $i$'s presence, and thus also compensates the seller for the maximum lost potential
revenue \cite{lazar}.
We have the definition of an elastic valuation function. In the PSP market, a
user is considered truthful if their bid price equals their marginal valuation,
i.e. $p_i^j = {\theta_i^j}'$.
\definition{\cite{lazar}}
A real valued function, $\theta(\cdot): [0,\infty) \rightarrow [0,\infty)$, is an \emph{(elastic) valuation
function} on $[0, D]$ if 
\begin{itemize}
    \item $\theta(0) = 0$,
    \item $\theta$ is differentiable,
    \item ${\theta}' \ge 0$, and ${\theta_i}'$ is non-increasing and continuous,
     \item There exists $\gamma > 0$, such that for all $z \in [0,D]$,
${\theta}'(z) > 0$ implies that for all $\eta \in [0, z), {\theta}'(z) \le
{\theta}'(\eta)
- \gamma(z - \eta)$. 
\end{itemize}


\section{Related Work}

\section{The Problem Model}
\subsection{The CMHK Market}

We begin to construct the model for a PSP data auction for  mobile users
particpating in CMHK's secondary data-sharing market.
Define $\mcI = \lbrace 1, \cdots, I\rbrace$ as the set of users who purchase or sell
data from other users. We again define the bidding profile for any user to be 
$s \equiv
[s_i^j] \in \mcI \times \mcI$, and $s_{-i} \equiv [s_1^j , \cdots , s_{i-1}^j , s_{i+1}^j , \cdots
, s_I^j]_{j\in\mcI}$ as the profile of user $i$'s opponents. 
As a cellular network is fully connected, we assume that a buyer submits bids directly to sellers via the
cellular network. We assume that users are selfish, and therefore
rational. 
 Users prefer to participate in the secondary market as it allows them to purchase
additional data for a cost less than the overage fee set by the ISP. 
Thus we assume that all users submit bids in order to maximize their
(private) valuation functions. 
In general, user preferences are defined by a utility function, $u$, which typically
represents a users' valuation of an allocation minus the price.
Absent the cost or revenue from trading data, CMHK users gain utility from consuming
data. 
We will assume that the user valuation satisfies the conditions for an
\emph{elastic demand function}, as in \cite{lazar}. 
As the CMHK market does not allow for brokers \cite{cmhk}, we determine that
the bid profiles must adhere to some additional restrictions, which we will
imply using the PSP bid profile notation.
A user's identity $i \in \mcI$ as a \textbf{subscript} indicates that the bid
belongs to a \textbf{buyer}, and a \textbf{superscript}, $j\in\mcI$, indicates
the bid belongs to a \textbf{seller}.
Suppose $i$ is buying from $j$. The bid is represented by $s_i^j = (d_i^j, p_i^j)$, 
meaning $i$ would like to buy from $j$ a quantity $d^j_i$ and is willing to pay
a unit price $p^j_i$. Without loss of generality, we assume that all users bid in all
auctions; if a user $i$ does not submit a bid to $j$, or vice versa, we simply set $s_i^j =
(0,0)$. Naturally, in a live auction, if a buyer does
not submit a bid to a seller, then this implies $s_i^j = 0$ for both buyer $i$
and seller $j$.
A seller $j$ places an ask $s_i^j = (d_i^j, p_i^j)$, meaning $j$ is offering a
quantity $d_i^j \in d^j = [d_i^j]_{i\in\mcI}$ 
with reserve unit price $p_i^j \in p^j = [p_i^j]_{i\in\mcI}$ to buyer $i$. We emphasize that we allow for $s_i^j$ to
stand for a buyer or sellers' bid, the \emph{direction} of the bid (vector) is determined
by the user type, whether or not they are  a buyer or a seller. 
To further clarify our analysis, we will emphasize the separation of buyers and
sellers using
$s_i$ and $s^j$, indicating if a bid is from a buyer or a seller.
In other words, a bid $s^j =
[s_i^j]_{j\in\mcI}$ is understood as an offer of data by seller $j$ in the CMHK
secondary market. The notational conventions of the bid vectors are essentially 
slices of the grid, $s^j = [s_i^j]_{i\in\mcI}$ denotes a sellers' profile, and $s_i =
[s_i^j]_{j\in\mcI}$ denotes a buyers' profile. 
Furthermore, noting that this is a simplification for ease of notation, we let $D^j = \sum_{i\in\mcI}
d_i^j$ be the total amount of data $j$ has to sell, and $D_i = \sum_{j\in\mcI}
d_i^j$ represent the
total amount of data requested by buyer $i$. 
Finally, we assume that data is a unary resource belonging to the seller, and
therefore can identify each local auction
with the identity of the seller $j\in\mcI$.

Consider the grid of bid proiles, $s$, representing the distributed PSP auction
mechanism in the CMHK market, each buyer $i$ will
have information from each seller $j$, as well as opponent profiles $s_{-i}$
from each auction in which it is participating, and therefore
in the extreme case, buyer $i$ gains access to the full grid $s$. The extreme with respect to
$i$ is the case where buyer $i$ will participates in all auctions $j\in\mcI$
(SHOULD I PROVE THIS OR TOO TRIVIAL?). However, 
sellers can only gain information about the market grid by observing buyer
behavior in their local auction.
In our current formulation, we do not allow a seller to host multiple
auctions(FUTURE WORK?). Thus, only the buyers are able to directly and globally
influence the CMHK market dynamics, with the sellers taking a secondary role
(FIND IN ANOTHER PAPER FOR SUPPORT OR OBVIOUS ENOUGH?).

The main contribution of this work 
is an auction mechanism inspired by the classic PSP throughput problem. In
order to apply a distributed PSP implementation to CMHK's secondary market,
we analize the behavior of users in a dynamical data exchange market. As both buyers and sellers are able
to change their bid strategies, and as each user only has \emph{local}
information about the bidding environment, it is clear that an unconstrained
market, even with a finite number of users, could suffer from the communication
expense from numerous local auctions trading an infinitely divisible resource.
We will assume that the cost of participatingn in the CMHK secondary market is
absorbed by the bid fee, which could represent data used in submitting bids, or
a fee charged per unit of data, or a flat rate charged at the completion of the
purchase. We perform a simple survey of these bid fee models, we
provide some idea of the expected revenue of the mobile data ISP (NEED
STATS!). It is worth mentioning that CMHK users are not allowed to resell data purchased from the secondary
market, additionally, the purchased data expires (does not carry to the next
service period). Therefore, a simple definition of market equilibrium, where
supply equals demand, is insufficent to complete a comprehensive analysis of
the CMHK data-exchange market behavior. We will make an attempt to address why
our formulation at least partially considers some of these issues (BAD), such
as the impact of the bid fee on user behavior. (MOVE SOME OF THIS UP)
Finally, we  .... (WHAT? COMPLETE THE THOUGHT)

We claim that the market is
able to achieve an equilibrium as the sellers and buyers have an
incentive for a collaborative exchange, our mechanism provides the functions
for the effective communication between the connected users. It was shown in
\cite{lazar} that a $2$-dimensional message space is sufficient for the PSP
auction. Using a restricted message space is essential for the distributed
nature of our design (EXPLAIN), however, as a given message can come from many
possible types, there is no single
way to do the transformation from the direct revelation mechanism to the
desired one. This is equivalent to guessing the
right direct-revelation-to-desired-mechanism transformation and building it
into the allocation rule from the start. (FINISH!)
We claim that in
this secondary market our formulation not only holds the desired VCG qualities,
but minimizes comminication overhead (and so possibly fees paid to the ISP)
and auction duration, resulting in a convergence time (FIND COMPETITVE RATIO?)
with respect to the classic throughput problem. 


\subsection{User Behavior}

We are not concerned with network
bottlenecks, which is purely a bandwidth problem, as in \cite{semret}, however
we reason that there remains an optimal user strategy. As a user no longer needs
to bid on a complete (but arbitrary)
cluster of nodes with minimum bandwidth allocation (defined as a route in
\cite{semret}), we reason that a buyer may opt-out of auctions, maximizing its
utility while minimizing the number of positive bids submitted to the overall
market. We define an \textbf{opt-out function}, $e$, as a function that when composed with our user type to describes
its market behavior. In a general sense, $e$ applies our user \emph{strategy}
to the PSP rules.
Our reserve price function is determined by the subset of nodes participating in the
auction, where the seller is its own auctioneer. This implies that the influence of the greater market on the individual
auctions will be influenced only by the submission of bids from buyers to
sellers. As a buyer may have access to multiple auctions, the sellers will be
dynamically influenced by the market via the $\epsilon$-best replies from the
buyers. We demonstrate that as the valuation function of seller $j$ is dependent
on the buyers demand, and further show evidence of symmetry in the strategies of
buyers and sellers.

\subsubsection{Buyer Strategy}

% OPT-OUT BUYER
We define each buyer as a user $i\in\mcI$ with quasi-linear utility
function $\theta_i$, the value of the allocation minus the cost,
\begin{equation}\label{buyerutility}
    u_i = \theta_i \circ e_i(a) - c_i,
\end{equation}
where $e_i : [0, \infty)^I \rightarrow [0,\infty), \ \forall \ i \in \mcI$. 
We extend the P2P rules described
in \cite{semret} to account for a set of \emph{local} data-exchange markets.

Suppose the total amount of seller $j$'s data on the network at the instance that
user $i$ joins the auction is $D^j$. 
A sellers' allocation cannot exceed the total amount they have available,
i.e. $\sum_{i\in\mcI} a_i^j \le D^j$. This will hold simultaneously for each $i \in
\mcI$ if and only if 
\begin{equation}\label{isp}
    D^j \ge \sum_{i\in\mcI} \ D_i.
\end{equation}
We define the only ``seller" to satisfy (\ref{isp}) to be the ISP.
We will show that in our
algorithm, sellers are restricted to subset of buyers $\in\mcI$, and provide a
buyer strategy defining when a rational (utility-maximizing) buyer will set $s_i^j = 0$. The seller, in our analysis, is a functional
extension of the buyer, with valuation $\theta^j$ constructed by buyer demand.
We assume that buyers and sellers are separated (a seller does not also buy
data and vice versa).

Although it is possible for a seller to fully satisfy a buyer $i$'s demand, it
is also reasonable to expect that a seller may come close to using their
entire data cap, and only sell the fractional overage. In this case, we
determine that buyers must split their bids among multiple sellers. 
We propose the following strategy, 
% BUYER STRATEGY
{
\proposition{(Opt-out buyer strategy)}\label{buyerstrategy}
Define any auction duration to be $\tau \in [0,\infty)$. 
Let $i\in\mcI$ be a buyer and fix all other buyers' bids $s_{-i}$ at time
$t\in\tau$. % For any allocation $a$ at time $t$, let
% OPT-OUT BUYER STRATEGY
%\begin{equation}\label{opt-out}
    %e_i^j(a) \triangleq \frac{a_i^j}{\vs_i^j} 
    %e_i^j(a) \triangleq a_i^j,
%\end{equation}
%and define,
Define the composition,
$$
     e_i^j \circ a = e_i^j(a) = \frac{a_i^j}{j},
$$
to be the buyer strategy with respect to quantity, and the set,
% MIN SET
$$
    \mcI_i(n) =\argmax_{\mcI' \subset \mcI, \vert\mcI'\vert =
n}\sum_{j\in\mcI'} D^j,
$$
where buyer $i$ chooses its seller pool by determining $n$, where
\begin{equation}\label{buyercoordinate}
    %n = \argmin_{\ell_i}(j \in \ell_i : \displaystyle\sum_{n} \frac{d^n}{d_i}
    %n = \argmin_{\ell_i}(j \in \ell_i : \displaystyle\sum_{n} d^n
    n = \min( j\ \vert\ j\in \mcI_i(n): j D^j\ge D_i),
\end{equation} 
The buyer strategy produces a minimal subset of sellers $\in \mcI$, so for any
fixed $n$ we will denote this subset,
\begin{equation}\label{sellers}
    \mcI_i \subset \mcI.
\end{equation}  
Now let $j^* = n \le I$, and define,
\begin{equation}\label{opt-minallocation}
    e_i(a) \triangleq e_i^{j^*}(a)
\end{equation}
$\forall \ j \in \mcI_i$.
As (\ref{opt-minallocation}) holds $\forall \ j\in\mcI_i$,
we have that $e$ defines an optimal feasible strategy for buyer $i$ from time $t$ to time $(t+1) \in\tau$.
}\\
\textbf{Proof:}

We assume that a buyer wants to fufill their data requirement.
In the case that there exists a seller who can completely satisfy a buyers'
demand, $j^*=1$, $\vert \mcI_i\vert =1$ and (\ref{buyercoordinate}) holds. If such a buyer does not exist,
as the set $\mcI_i$ is an ordered set, $i$ may discover 
$j^*$ by computing $\mcI_i$.
If we suppose that $D_i >
\sum_{j\in\mcI}D^j$, then $j^* > I$ and $\mcI_i = \emptyset$. 
We model the ISP as a seller $\kappa$ with bid $s^\kappa =(D^\kappa, P^\kappa)$,
where $D^\kappa > D^j, \
\forall \ j\in\mcI_i$, and
$P^k$ represents the overage fee for data set by the ISP, which is also the upper bound of the sellers'
pricing function, and so again (\ref{buyercoordinate}) holds.
Now consider some $k \ne i \in \mcI$ where $p_i^j = p_k^j$. 
The allocation rule (\ref{dataallocation}) determines that the data will be split proportinally
between all buyers with the same unit price. 
It is possible that the resulting partial allocation of data to $i$ and $k$
would not satisfy some demand. As the two cases $i$ and $k$ are the same, we consider such
a seller $i$. Suppose seller $j$ updates its
bid to reflect the new data quantity, so that $d_i^{j(t+1)} < d_i^{j(t)}$ (NOTE:
IS THE DIRECTION OF THE BID CONFUSING?, i.e. $d_i^j$ from buyer (rhs) or seller
(lhs) unclear? does it matter?) Then, by the definition of $\ell_i$, we can
only have that $j \ni \mcI_i$, or that $n$ has been increased, and new seller(s)
enter the pool. 
Additionally, we consider that at time $(t+1)$, we can have a 
new buyer $k$, where $p_k^j > p_i^j, \ \forall \ j \ \in \mcI_i$, in other
words, a new buyer $k$ may enter the market with a better price. In this case, by (\ref{buyercoordinate}), $i$ will increase the value of
$n$ so that
$$
    e_i^{j(t+1)}(a) = e_i^{j(t)}(a) - e_k^{j(t)}(a),
$$
and the subset $\mcI_i$ is large enough to balance the additional demand from
$k$. Thus, as in each case we have that $i$ is able to satisfy thier demand,
and we determine that the opt-out strategy is optimal.

Finally, we note that $\mcI_i$ is not the only
possible minimum subset $\in\mcI$ able to satisfy $i$'s demand, it is the
minimal ordered subset where a
coordinated bid is possible, the reasoning for which we will address in further analysis (Section
\ref{analysis}). 

\subsubsection{Buyer Influence}
The buyer demand is a key market influence (SILLY SENTENCE), as the buyer
valuation is elastic, even infintesimal changes in the market dynamics can be
modeled. $\forall \ y\ge 0$, we determine that the market demand for $D^j$ is given by,
% BUYER DEMAND 
\begin{equation}\label{datademand}
    \rho^j(y) = \sum_{i\in\mcI : p_i^j\ge y} d_i^j, 
\end{equation}

Buyer $i$'s valuation is interpreted as a unit valuation $\theta_i$,
which is distributed across the secondary market via bids in local auctions. 
Buyer behavior is influenced by opponent bid profiles from multiple auctions.
The valuation of any user, however, is a function of the entire marketplace.  
We first define the ``inverse" demand for any buyer $i$,
% BUYER REVENUE 
\begin{equation}\label{buyerinversedemand}
    f_i(z) \triangleq \inf\big\lbrace y\ge 0:
        \rho^j(y) \ge z, \ \forall \ j \in \mcI\big\rbrace.
\end{equation}
For a given demand $\rho^j$, $f_i$ maps the $z$th unit of data to the lowest
possible unit price. \\
\textbf{Remark:} In a coordinated bid, $f_i$ is such that
$i$ could still bid in \emph{any} auction $j\in\mcI$.
Naturally, the coordinated bid is the
lowest unit cost of the buyer to participate in all auctions, and corresponds to
the maximum reserve price amongst the sellers. 

The seller is represented by a similar function, however a seller only has a
single auction, and is indirectly influenced by the buyers in other auctions, and so from the perspective of the
seller we have a more direct interpretation of revenue.
We define the ``inverse" of the buyer demand function for seller $j$ as revenue at unit price $y$. 
% SELLER REVENUE 
\begin{equation}\label{sellerinversedemand}
    f^j(z) \triangleq \sup\big\lbrace y\ge 0:
        \rho^j(y) \ge z, \ \forall \ i \in \mcI\big\rbrace,
\end{equation}
where $f^j$ maps to the highest possible unit data price.
Thus we have the following Lemma. 
{
\lemma{(User valuation)}\label{uservaluation}
For any buyer $i$, the valuation of seller $j$'s data may be modeled as,
\begin{equation}\label{singlevaluation}
    \theta_i^j \circ e_i^j(a) = \int_0^{e_i^j(a)} f_i(z) \ dz,
\end{equation}
it follows that 
\begin{equation}\label{buyervaluation}
    \theta_i \circ e_i(a) = \displaystyle\sum_{j\in\mcI}
 \int_0^{e_i^j(a)} f_i(z) \ dz.
\end{equation}
Similarly, seller $j$'s valuation is,
\begin{equation}\label{sellervaluation}
    \theta^j \circ e_i(a) = \displaystyle\sum_{i\in\mcI}
 \int_0^{e_i^j(a)} f^j(z) \ dz.
\end{equation}
}
\textbf{Proof:} 
We assume that a buyer wants to minimize the cost of purchasing their data
requirement, at the same time ensuring they get the full allocation $e_i(a)$,
and so must minimally meet $j$'s reserve price.
A seller will try to maximize profit for any given allocation $a$,
and will try to sell all of its data, and so 
$d_i^j = e_i(a) \Rightarrow \sum_{i\in{\mcI}} d_i^j = D^j =
\sum_{i\in\mcI} e_i(a)$,
The remainder of the proof follows as in \cite{semret}. (REALLY THOUGH?)
(THERE MUST BE A CLEARER WAY TO DESCRIBE THE RELATIONSHIP)

We note that except at points of discontinuity, we have
$P_i(z) = f_i(z)$. This mechanism generalizes Vickrey
(``second-price") auctions which are for nondivisible objects.

The sellers' natural utility is the potential
profit $u^j = \theta^j \circ e_i(a) $, where $\theta^j$ is the
potential revenue from the sale of data composed with each buyers' opt-out
value, $e_i(a)$. 
We have chosen to omit the original cost of the data
paid to the ISP, as a discussion of mobile data plans is outside the scope of this
paper.  

As $j$'s behavior is restricted by
the bid strategy of the buyers, in addition to natural constraints, and we have the following Lemma.
{
% SELLER CONSTRAINTS
\lemma{(Seller constraints)}\label{constraints}
Let $j$ be a seller with total data amount $D^j$.
First, the seller must satisfy the quantity
constraint, 
\begin{equation}\label{quantity}
    d_i^j \ge e_i^j(a)
\end{equation}
and
\begin{equation}\label{budget}
    \displaystyle\sum_{i\in\mcI} e_i^j(a) \le \sum_{i\in\mcI} d^j_i \le D^j,
\end{equation}
which implies, for any $i\in\mcI$,
\begin{equation}\label{partialallocation}
    e_i^j(a) \le D^j - \sum_{k\in\mcI, k\ne i} e_k^j(a).
\end{equation}
In addition, for a rational seller, the reserve price must
satisfy, $\forall \ i\in \mcI$,
\begin{equation}\label{reserveprice}
   p_i^j \ge \min_{i\in\mcI}\big(p_i^j).
\end{equation}
}\\
\textbf{Proof:}
The first statement is obvious, 
a seller cannot sell more data than indicated in their bid, (\ref{budget}) and
(\ref{partialallocation}) enforce
a data constraint for the seller. Finally, (\ref{reserveprice})
follows from the assumption that $j$ is rational, and so utility-maximization
acts as revenue maximization; a rational seller will not sell its
data at a price less than the lowest offer. (ASSUMPTIONS MADE HERE)


\subsubsection{Seller Strategy}
In order to to develop the seller strategy, it is necessary we determine 
that a seller has an incentive to
accept fractional (CHANGE TO PARTIAL? PICK ONE) bids (i.e. sell a fraction of
their data $D^j$). Reasonably,
there may not exist a buyer such that $D_i = D^j$. \cite{zheng} reasons that the seller does not know the exact amount of
leftover data available, and so they may only sell enough data to ensure that
they will not become a buyer while they submit their total data overage to the
secondary market. Buyers are allowed to
bid both dynamically and asychronously, and as a seller determines allocations using only bids
in its local market. Therefore, in order to
maximize the revenue gained per unit of data the seller must respond to the
varation of competitive bids in its market (MORE? FUTURE WORK? CITE?).

We describe the sellers' \emph{local} auction strategy for allocating its
data according to the constraints formed by the buyer strategy. As
local auction is progressive, and influenced by the $\epsilon$-best replies of
the buyers, we will need the following Lemma. 
We now define the local auction, which we describe, when coupled with the buyer
responses, as a progressive game of strategy with incomplete, but perfect
information (SAY MORE?).
(BUYERS ARRIVE AS A POISSOIN PROCESS? FUTURE WORK)

% SELLER LOCAL STRATEGY
{
\proposition{(Localized seller strategy (i.e. fractional allocation))}\label{sellerstrategy}
Define any auction duration to be $\tau \in [0,\infty)$. For any seller $j$,
For any seller $j$, fix all buyers bids $s_i^j=(d_i^j,p_i^j)$ at time $t\in\tau$.
Define
$$
    \mcI^j(n) =\argmax_{{\mcI}' \subset \mcI, \vert{\mcI}'\vert =
n}\sum_{i\in{\mcI}'} p_i^j,
$$
where,
\begin{equation}\label{sellercoordinate}
    n = \min( i \ \vert \ i \in\mcI^j(n) : \displaystyle\sum_{i\in\mcI} d_i^j \ge
D^j),
\end{equation} 
We have, for any fixed $n$ at time $t$, a minimal subset 
of buyers that maximizes $j$'s revenue, which we will
denote,
\begin{equation}\label{buyers}
    \mcI^j \subset \mcI.
\end{equation} 
Define buyer $i^* = n \le I$.
Then, for time $(t+1)$, set $j$'s reserve price as 
\begin{equation}\label{newprice}
    p_i^j = \theta_{i^*}'(d_{i^*}^j) + \epsilon,
\end{equation}
Let the winner at time $t$ be determined by,
\begin{equation}\label{winner}
    \bar{i} = \displaystyle\max_{i\in I^j} p_i^j,
\end{equation}
and update $j$'s total data to reflect the (tentative) allocation,
\begin{equation}\label{newbudget}
    D^{j(t+1)} = D^{j(t)} - e_{\bar{i}}^{j(t)}(a),
\end{equation}
Allowing $t$ to range over $\tau$, we have that (\ref{winner}) - (\ref{newbudget}) produces a local
%Assuming that no new buyers enter the market, we have that (\ref{winner}) - (\ref{newprice}) produces a local
market equilibrium. (BLAST I RUINED THE EQ) % from time $t$ to $(t+1) \in \tau$. 
}\\
\textbf{Proof:}
We assume that the seller has enough data to satisfy at least one buyer,
and that they want to maximize their revenue.
In the case of multiple buyers $i^*$ is the \emph{losing} buyer with the highest unit
price offer, determined by (\ref{sellercoordinate}), where $i^*$ loses by: (1)
$i^*$ is excluded from $I^j$ or (2) $i^*$s demand is not met, noting as well
that any $i\ni \mcI^j$ is also a loser. 
In this case, by (\ref{quantity}) the seller must notify the buyer of a
fractional allocation.

%As the set of buyers is
%computed at each iteration, we are guaranteed a subset $\mcI^j \subset \mcI$ that is 
%at equilibrium as it is designed so that the data offered by
%seller $j$ equals the aggregate demand of the buyers \cite{???}
With this caveat, we have that the aggragate demand of subset $\mcI^j$ is satisfied by seller $j$.
Although the buyers' valuation $\theta_i$ is not known to the seller, we will assume
that the buyer is bidding truthfully, and so ${\theta_i}' + \epsilon
=p_i^j + \epsilon$, and as $\mcI^j\subset \mcI$, we
note that (\ref{budget}) and (\ref{reserveprice}) hold. Now, using
(\ref{singlevaluation}), we have, $\forall \ z\ge 0$,
\begin{align*}
    \int_0^{e_{i^*}^j(a)} f_{i}(z)\ dz &\le\int_0^{e_i^{j}(a)}
f^{j}(z) \ dz 
\end{align*}
and so,
$$
     \theta_i \circ e_{i^*}(a) \le \theta_i \circ e_i(a),
$$
which holds $\forall \ j \in \mcI_i$.
It follows that, using
%, using (\ref{buyercoordinate}), 
the definition of an $\epsilon$-best reply $s_i^j = (v_i^j, w_i^j)$,  
for any $\epsilon$-best reply, 
$$
    p_i^j \le {\theta_i}'(v_i^j) + \epsilon,
$$
$\forall i \in \mcI^j$.
Therefore the choice of $p_i^j$ does not force any buyers out of the local
auction. Thus we determine the valuation between seller
$j$ and buyer $i$ is well-posed, the reserve price (\ref{newprice}) is justified, and the local
equlibirium created by $j$ is stable from time $t$ to $(t+1)$. 
We note the special case where for winner $\bar{i}$, $\ \exists \ k\in\mcI^j$
such that $p_{\bar{i}}^j =
p_i^k$. In this case the seller again notifies the buyers of a fractional
allocation by changing $d_i^j$ and $d_k^j$ to reflect the proportional
division. Finally, in the case where winning buyer(s) \emph{opts-out} of the local
auction, where $j\in\mcI_i$ at time $t$, then at time $(t+1)$ $j\ne \mcI_i \Rightarrow s_i^{j(t+1)} = (0,0) \Rightarrow i \ne
\mcI^j$, which is mathematically analogous to the definition of opting-out in our
scenario. The seller may simply return the tentatively allocated data to $D^j$,
and recompute $\mcI^j$ and $p_{i^*}^j$, 
$$
    D^{j(t+1)} = D^{j(t)} + \displaystyle\sum_{i\in\Gamma} d_i^{j(t)}
$$
where $\Gamma$ represents the buyers who have opted-out from time $t$ to 
$(t+1)$, indicated by a zero bid from any buyer $\in \mcI^j$:
$$
    \Gamma = \lbrace i\in \mcI^j : s_i^{j(t+1)} = 0\rbrace.
$$

(NOTE - NEED TO SAY MORE HERE... INDUCTION? HOW DOES THE TENTATIVE AWARD WORK IN
FRACTIONAL ALLOCATIONS? SAVE FOR NASH EQ PROOF? KEEP WITH t TO t+1?)


We conclude this section by examining the relationship between the strategies of buyers and
sellers in local auctions. As we have shown, the seller
is a functional extension of the buyer, with rules determined by the buyers'
behavior. Consequentially, we have the following Lemma.

% USER UNION
{
\lemma{(Valuation across local auctions)}\label{userunion}
For any $i,j\in \mcI$, 
\begin{equation}\label{jtoi}
j\in\mcI_i \Leftrightarrow i\in\mcI^j.
\end{equation}
Fix an auction $j\in\mcI$ and define the set of users $\Lambda$, where
\begin{equation}\label{lambda}
    \lambda = \bigg(\displaystyle\bigcap_{i\in\mcI^j} \mcI_i \bigg) \cup \mcI^j,
\end{equation}
\begin{equation}\label{lambda}
    \Lambda = \bigcap_{i\in\mcI^j} \bigg(\bigcap_{k\in\mcI_i} \mcI^k \bigg),
\end{equation}
\begin{equation}\label{lambda}
    \Lambda = \mcI^j \cap \bigg(\displaystyle\bigcap_{i\in\mcI^j} \mcI_i \bigg),
\end{equation}
and for each $i\in\mcI^j$ define,
\begin{equation}\label{ilambda}
    \Lambda_i = \mcI^j \cap \bigg(\displaystyle\bigcap_{k\in\mcI_i} \mcI^k
\bigg).
\end{equation}
(WANT TO SAY $s_i^j > 0$ iff $i,j \in\Lambda \Rightarrow s_i^j \in \Lambda
\equiv s_i^j \in \mcI$)
Fixing all bids $s_i^j \in \mcI$, and time $t\in\tau$,
\begin{equation}\label{sym}
    (\theta_i \circ \theta^j) = (\theta^j \circ \theta_i).
\end{equation}
(ONLY HAVE ONE HALF OF THE SYMMETRY SO FAR! REALLY WANT TO MAKE $s_i = s^j$
HAPPEN)
}\\
\textbf{Proof:}
A local auction $j\in\mcI$, is determined by the collection of buyer bid
profiles, where $s_i^j > 0 \Rightarrow j\in\mcI_i$, and we have that $i\in\mcI^j$,
if and only if $p_i^j \ge p_{i^*}^j$.
Suppose $p_i^j < p_{i^*}^j$. By Proposition \ref{sellerstrategy}, $i^*$ is either included or excluded from $\mcI^j$. 
If $i^* \ni \mcI^j$, then (\ref{jtoi}) holds; in the latter case, by
Proposition \ref{buyerstrategy}, $i^*$
will add sellers to its pool until its demand is satisfied. Therefore for any
$j\in\mcI_i$, $\not\exists \ s_i^j > 0$ where $i \ni \mcI^j$, and so (\ref{jtoi})
holds.

The subset $\mcI^j \subset \mcI$
determines $j$'s reserve price $p_{i^*}^j$. Assuming the buyer submits a
coordinated, truthful bid, $\mcI_i\subset \mcI$ 
will determine the unit price $p_{i}$ in buyer $i$'s bid.
(\ref{sellercoordinate}) defines $i^*$ as the losing buyer with the highest bid
price, which in turn defines $j$'s reserve price, $p_{i^*}^j$,
the lowest price that $j$ will accept to perform any allocation.
Using (\ref{buyerinversedemand}), (\ref{sellerinversedemand}), and Proposition
\ref{sellerstrategy},
we have that for each $i \in \mcI^j$, 
$p_{i}^j \ge p_{i^*}^j \ge p_i^k > 0$, which holds $\forall \ k \in\mcI_i$.
Let $p_*^j=p_{i^*}^j$ denote the reserve
price of auction $j$, noting that $p_i^j=0, \ \forall \ i \in
[p_i^j]_{i\ni\mcI^j}$.
Now, we have,
%Fix any buyer $i\in\mcI^j$.
\begin{equation}\label{rho}
    \rho^j(p_*^j) \le \displaystyle\sum_{i\in\mcI^j} e_i(a(s_i^j)),
\end{equation}
and, $\forall \ i \in \mcI^j$, 
\begin{equation}\label{invrho}
    f_i \circ e_i^j(a) \ge p_*^j,
\end{equation}
and, $\forall \ k \in \mcI_i$, 
\begin{equation}\label{invrho}
    p_*^j \ge f^k \circ e_i^k(a) ,
\end{equation}
from (\ref{datademand}), (\ref{buyerinversedemand}),
(\ref{sellerinversedemand}), and so $p_i^j \ge p_{*}^j \ge p_i^k, \ \forall \ i \in
[p_i^j]_{i\in\mcI^j}$ and $\forall \ k \in [p_i^k]_{k\in\mcI_i}$

The sellers $\in\Lambda$ are competing to sell
their respective resources to buyers whose valuations are distributed across multiple auctions. 
%By (\ref{jtoi}), $j\in \mcI_i, \ \forall \ i\in \mcI^j$. 
Consider a disjoint local market $j$, where $\forall \ i\in\mcI^j$, $s_i^k = 0$
for any $k\ne j \in \mcI_i$. 
Using (\ref{buyerinversedemand}) and  (\ref{sellerinversedemand}), we have
$\forall \ i \in \mcI^j$,
\begin{equation}\label{feq}
    p_*^j = f^j \circ e_i(a) = f_i\circ e_i(a) = p_i^*,
\end{equation}
where we denote $p_i^*$ as the coordinated bid price of buyer $i$.
Now, suppose $\exists \ i\in\mcI^j$ such that $p_i^* \ge p_*^j$.
Since,
$$
    p_i^* = \max_{k\in\Lambda_i}(f^k) \circ e_i(a)= \max_{k\in\Lambda_i}(p_*^k),
$$ 
there must be some $k \ \in\mcI_{i}$, where $p_i^k > p_*^j$, and so, 
we have that $ \Rightarrow p_i^k > p_*^j$. Also, $\not\exists$(HERE) 
It follows that buyer $i$'s bid price is given by,
\begin{equation}\label{iprice}
    p_i^j = \max_{j\in\Lambda}(f^j) \circ e_i(a)= \max_{j\in\Lambda}(p_*^j),
\end{equation}



%where $p_i^*$ represents $i$'s coordinated bid price,
%i.e., $p_i^j = p_i^*, \ \forall \ j$ such that $s_i^j > 0$.
This holds for any
$i\in\mcI^j$ and similarly if $p_i^k < p_i^j$. 
Now, from (\ref{feq}), we have that $f_i=f^k=f^j, \ \forall \ i,j,k \in \Lambda$, as such, we may
denote each as $f$ in the remainder of the proof.

Now, using (\ref{rho}), and intersection of subspaces, we determine the aggregate demand from buyer
$i\in\mcI^j$ to be, (HERE)
(SHOW THAT $\Lambda$ IS DISJOINT!)
\begin{equation}\label{sumrho}
    \displaystyle\sum_{k\in\mcI_i\cup\mcI^j}\rho^k(p_*^k) =
\sum_{k\in\mcI_i\cup\mcI^j}
e_i^k(a),
\end{equation}
where $p^k$ is the highest unit price in the feasible set $\mcI_i\cup\mcI_k$.
Thus we determine the aggregate demand from buyer
$i\in\mcI^j$ over the set of intersecting markets $\Lambda$ to be, 
\begin{equation}\label{sumrho}
    \displaystyle\sum_{k\in\mcI_i\cup\mcI^j}\rho^k(p^*) =
\sum_{k\in\mcI_i\cup\mcI^j}
e_i^k(a),
\end{equation}
where $p^* = \max(p^k), \ \forall \ k \in\Lambda$, and corresponds to the
highest reserve price over all the sellers. 
Now, by the continuity of $\theta'$, (AND)
\begin{align*}
     \displaystyle\sum_{i\in\mcI^j} &\bigg(\sum_{k\in\mcI_i\cup\mcI^j}\int_0^{e_i^k(a)}
f(z) \ dz\bigg)\\
     &=\displaystyle\sum_{i\in\mcI^j} \theta_i \circ e_i(a) \\
     &= \sum_{k\in\mcI_i\cup\mcI^j}\bigg(\sum_{i\in\mcI^j}\int_0^{e_i^k(a)}
f(z) \ dz\bigg)\\
     &= \sum_{k\in\mcI_i\cup\mcI^j}\theta^j \circ e_i(a)\\
     &= (\theta_i \circ \theta^j) \circ e_i(a).
\end{align*}


\iffalse
************************
Proposition \ref{buyerstrategy} defines $e_i^j(a) = D_i/ 2$, and so $0 <
e_i^j(a)$.
************************
\fi
\iffalse
Suppose $\exists \ i \in\mcI^j$ such that $\mcI_i = j$, so that $j$ is the only
seller in $\Lambda$.
, and $i$'s bid quantity,
\begin{equation}\label{iquantity}
    d_i^* = \circ e_i(a) = \frac{a_i^j}{j},
\end{equation}
where $j=j^* = \argmin$ for each $i\in\mcI^j$, thus $d_i^*$ represents $i$'s coordinated bid quantity, and we have
$s_i^j = (p_i^*,d_i^*), \ \forall \ j$ such that $s_i^j > 0$,
\fi
\iffalse

*****************************************************************************************************
From (\ref{jtoi}), $i\in\mcI^j \Rightarrow
j\in \mcI_i$, so $i\in \mcI^j \cup \mcI^j$, 
Using (\ref{buyerinversedemand}), we have the valuation of $j\in\mcI_i$,
%$i\in\bigcap_{i\in\mcI^j}\mcI_i$,
\begin{align*}
      \theta^j \circ e = \sum_{k\in\mcI_i\cup\mcI^j}}&\int_0^{e_i^k(a)} f(z) \ dz,
\end{align*}
and by (\ref{sellerinversedemand}), we have the valuation of $j$ fixed, 
\begin{align*}
      \theta^j \circ e = \sum_{i\in\mcI^j}&\int_0^{e_i^j(a)} f(z) \ dz,
\end{align*}
and thus we have the valuation of $\Lambda^j$,
\begin{align*}
     \displaystyle\sum_{i\in\mcI^j} \bigg(\sum_{k\in\Lambda}&\int_0^{e_i^k(a)}
f(z) \ dz\bigg).
\end{align*}
\\(FINISH!)\\
$$
     \displaystyle\sum_{j\in\mcI_i} \bigg(\sum_{k\in\Lambda}\int_0^{e_k^j(a)}
f(z) \ dz\bigg).
$$
----------------HERE--------------\\
(HOW TO MAKE SETS THE SAME???)
Now we consider $\Lambda_i$. For each buyer $i\in\mcI^j$, we have that $p_i^j
\ge p_i^k$

Now, for $\Lambda_i \cap \Lambda^j$, and by the continuity of $\theta'$,
\begin{align*}
     \displaystyle\sum_{i\in\mcI^j} &\bigg(\sum_{k\in\Lambda_i}\int_0^{e_i^k(a)}
f(z) \ dz\bigg)\\
     &= \sum_{k\in\Lambda_i}\bigg(\sum_{i\in\mcI^j}\int_0^{e_i^k(a)}
f(z) \ dz\bigg)\\
      &= \sum_{k\in\Lambda^j}\bigg(\sum_{j\in\mcI_i}\int_0^{e_k^j(a)}
f(z) \ dz\bigg)\\
      &=\sum_{j\in\mcI_i} \bigg(\sum_{k\in\Lambda^j}\int_0^{e_k^j(a)}
f(z) \ dz\bigg).
\end{align*}

(WHAT?)\\
Therefore, $\Lambda$ is such that $j\ni \mcI_i, \ \forall \ i \ni \mcI^j$. In other words, $\Lambda$ is a
disjoint set in the global marketplace with respect to $j$.
****************************************************************************************************
\fi

For completeness, in the case where the ISP $\kappa$ 
does not adhere to the market dynamics, so $P^\kappa > p^j +
\epsilon, \ \forall \ j \in\mcI$, then we may absorb the overage (difference) as part of the
bid fee.

(ADD TIME? INDUCTION GOES HERE? DID NOT SHOW!)

(OWN
WORDS) Since the sellers are driven by the buyers' demands, and the buyers are
competing in multiple local auctions hosted by the sellers, the two games are
inter-dependent, and may be played on the same or on a different scale in
valuation, time (AND....). 


PSP relies on the \emph{relevation prinicple}; the
PSP mechanism is able to demonstrate equlibrium by showing that PSP is
incentive-compatible. (MORE?)
Arrow's paradox is an impossibility theorem stating that when voters have three
or more distinct alternatives (options), no ranked voting electoral system can
convert the ranked preferences of individuals into a community-wide (complete
and transitive) ranking while also meeting a specified set of criteria:
unrestricted domain, non-dictatorship, Pareto efficiency and independence of
irrelevant alternatives.

Thus, (WHAT? FINISH, OWN WORDS)
The designer of a mechanism generally hopes either to design a mechanism
y() that "implements" a social choice function to find the
mechanism y() that maximizes some value criterion (e.g. profit)
Independence of irrelevant alternatives (IIA)
The IIA condition has three purposes (or effects)

Normative
Irrelevant alternatives should not matter.
Practical
Use of minimal information.
Strategic
Providing the right incentives for the truthful revelation of individual
preferences. 

\subsection{PSP Formulation}

We intend to show that our auction
is rational and achieves the desired VCG properties, as does the original
formulation. Using \cite{zheng} as a basis for our model, and \cite{lazar} as
realistic, theoretic, and notational templates, we define optimal strategies for
CMHK users, and demonstrate that the development of a set PSP auction mechanisms
in a data exchange setting is able achieve a network equilibrium for cellular data. 


% MECHANISM
\subsubsection{Data Auction Mechanism}\label{datamechanism}
We now proceed to formally define the PSP auction, which determines the actions 
buyers and sellers in the secondary market, which we will denote the \emph{data}
PSP rules. The rules presented here incorporate of the opt-out function with the
mechanism as in \cite{lazar}, which we note greatly simplifies our
analysis. 
The market price function (MPF) for a buyer in the secondary market
can be described as follows:
% NEW MARKET PRICE
\begin{align}\label{dataprice}
\begin{split}
    &\bar{P}_i(z, s_{-i}) =\displaystyle\sum_{j\in\mcI}P_i^j(z_i^j,
s_{-i}^j) \circ e_i^j \\
    &= \sum_{j\in\mcI}\bigg(\inf\bigg\lbrace y\ge 0 : 
    {D_i^j}(y,s_{-i}^j) \ge e_i^j(z) \bigg\rbrace \bigg),\\
\end{split}
\end{align}
and is interpreted as the aggragate of minimum prices that buyer $i$ bids in
order to obtain
data amount $z$ given opponent profile $s_{-i}$. We note that
the total minimum price for the buyer must be an aggragation of the
\emph{individual} prices of the buyers as it is possible that the reserve prices of the individual sellers may vary.
%To simplify the notation, we define $z_i^j = d^j / \vs_i^j$ opt-out
%buyer $i$'s bid quantity. 
The maximum available quantity of data in auction
$j$ at unit price $y$ given $s_{-i}^j$ is: 
% NEW ALLOCATION RULE
\begin{align}
\begin{split}\label{datacomposed}
    D_i^j(y,s_{-i}^j) = \bigg\lbrack D^j - \sum_{p_k^j> y} d_k^j\bigg\rbrack^+,
\end{split}
\end{align}
it follows that the inverse price function is aggregated over all local auctions
$j\in \mcI$,
% NEW INVERSE DEMAND
\begin{align}\label{datapriceinverse}
\begin{split}
    \bar{D}_i(y, s_{-i}) &= \displaystyle\sum_{j\in\mcI}\bigg(\sup\bigg\lbrace z\in \big[ 0,
    D^j \big] : \\
    &\bar{P_i}(z,s_{-i}^j) < y\bigg\rbrace\bigg).
\end{split}
\end{align}
Finally we must have that the cost to the buyer adheres to the second price
rule for each local auction, with total cost to buyer $i$,
% DATA COST
\begin{align}\label{datacost}
\begin{split}
    \bar{c}_i(s) &= \displaystyle\sum_{j\in\mcI} 
p^j \bigg(\bar{a}_i^j(0; s_{-i}^j)
    -\bar{a}_i^j(s_i^j;s_{-i}^j)\bigg).
\end{split}
\end{align}
% DATA ALLOCATION RULE
The resulting data allocation rule is a function of the local market
interactions between buyers and sellers over all local auctions, as is composed with $i$'s opt-out value, so that for each $i \in \mcI$, the
allocation from auction $j$ is,
% and so for profile $s_i$, for any $i \in \mcI$,
\begin{align}\label{dataallocation}
    \bar{a}_i^j(s) &= a_i^j(s) \circ e \\
    &= \min\bigg\lbrace e_i^j \circ D_i, \\
    &\frac{d_i^j}{\sum_{p_k^j= p_i^j}d_k^j} e_i \circ
D_i^j(p_i^j,s_{-i}^j)\bigg\rbrace,
\end{align}
noting that for the full allocation from all auctions we may simply aggregate
over the seller pool, as in the previous definitions.


\subsubsection{Mechanism Realization}\label{realization}

Consider a user seeking to prevent
data overage by purchasing enough data from a subset of other network users.
This user $i$ can be modeled as a opt-out buyer where, as in \cite{semret},
We intend to show that this network setting results in a shared network optima (a
global optimum). The formulation is inspired to the thinnest allocation route for
bandwidth given in \cite{lazar}. We note that if a single seller $j$ can
satisfy $i$'s demand, then
(\ref{buyerutility}) reduces to the original form, defined in
\cite{semret} as ``a simple buyer at a single resource element".

The sellers' auction will function as follows: at each bid iteration all buyers
submit bids, and the winning bid is the buyer $i$ that has the highest price
$p_i^j$. The seller allocates
data to this winner, at which point all other buyers are able to bid again, and
the winner leaves the auction (with the exception where multiple bidders bid
the same price, where (\ref{dataallocation}) determines they will not fully
satisfy their demand, and so we will assume they remain in the auction). The auction progresses as such until all the
sellers' data has been allocated. We design an algorithm based on the sellers'
fractional allocation strategy.

(NOTE - FIX TO UPDATED NOTATION!)
% SELLER ALGORITHM
\begin{algorithm}[H]
\caption{(Seller fractional allocation)}
\begin{algorithmic}[1]
\State $p^{j(0)} \gets \epsilon$
\State $s^{j(0)} \gets (p^j, D^j)$
\While{$D^j > 0$}
\State $\mcI^{j(t+1)} = \mcI^{j(t)}\setminus \lbrace i \in \mcI^{j(t)}: d_i^j >
D^{j(t)}\rbrace$
\State $ \bar{i} \gets \displaystyle\argmax_{I^j}\sum_{i\in I^j} p_i^j$ 
\State $D^{j(t+1)} \gets D^{j(t)} - e_{\bar{i}}^{j(t)}(a)$
%\State $p^j \gets \theta_{i^*}'(d_{i^*}^j)\circ e_i$
\State $p^j \gets p_{i^*}^j+\epsilon$ and $d^j \gets D^{j(t+1)}$
\State $s^{j(t+1)} \gets (d^j, p^j)$
\If{$\ \exists \ i : s_i^{j(t+1)} \ne s_i^{j(t)}$}
\State $D^{j(t+1)} = D^{j(t)}$
\State $t \gets t+1$
\State Go to 4.
\Else
\State $\bar{i} \gets e_{\bar{i}}^j(a)$
%\State $\beta^{j(t+1)} \gets \bigg\lbrace i\in\mcI^j: 
%        d_i \le \bar{b}^j \bigg\rbrace$
\State $t \gets t+1$
\State Go to 4.
\EndIf
\EndWhile
\end{algorithmic}
\end{algorithm}
We assume that each time that $s^j$ is updated it is shared with all
participating buyers. At this point buyers have the opportunity to bid again,
where a buyer that does not bid again is assumed to hold the same bid, since a
buyer dropping out of the auction will set their bid to $s_i^j=(0,0)$. As we will show in our analysis, the buyers are bidding truthfully; the
algorithm makes use of the fact that the sellers' valuation is determined by
the buyers' market and upholds the PSP mechanism. (CHECK) (ALSO $j$ NEEDS TO
UPDATE FOR NEW BUYERS, $j$'S CONTROL)

(BUYER ALGORITHM? WHY NOT..., CLEANER)

% VCG ANALYSIS
\section{PSP Analysis}\label{analysis}
\subsection{Equilibrium}

Consider an opt-out buyer $i\in\mcI$. A PSP auction
allows for a buyer to react to opponents bids, and so its incentive is based on
the opponent profile.

Due to (\ref{buyerutility}), $i$ only has an incentive to change its bid
quantity if it increases its opt-out value $e_i$, and therefore its utility. 
We will show that $i$ can coordinate its
bids over $\mcI_i$ so that the opt-out value $e_i$ is the same for each
$j \in\mcI_i$, and therefore, without loss of utility, $i$ may choose a seller pool 
using a ``consistent" strategy, where for any $j\in \mcI_i$, $d_i^j = d_i^k, \
\forall \ k \in \mcI_i$, and still have feasible best replies. 
Our result shows that a buyer may select $\mcI_i$ in order to maximize its
utility while maintaining a coordinated bid strategy. 
It is intuitive that, if $j^*
< I$, a buyer may increase the size of $\mcI_i$, thereby lowering its bid
quantity while obtaining the same (potential) allocation $a_i$. 
(OWN WORDS!)
One important question to ask is why a bidder should bid with identical unit
prices on (j) all auctions and not reduce the unit price to a level where he still wins the
amount q i . The reason for this can be found in the pricing rule of the PSP auction. 
% BUYER COORDINATION
{
\lemma{(Opt-out buyer coordination)}\label{coordinationlemma} 
Let $i\in\mcI$ be a opt-out buyer and fix all sellers' profiles $s^j$.
For any profile $S_i = (D_i, P_i)$, let $a_i \equiv \sum_j a_i^j(s)$ be the resulting data
allocation. For any fixed $S_{-i}$, a better reply for $i$ in any auction is $x_i =
(z_i,y_i)$, where $\forall \ j \in \mcI_i$,
\begin{align*}
    z_i^j &= e_i^{j^*}(a), \\
    y_i^j &= \theta_i'(z_i^j).
\end{align*}
Furthermore,
\begin{equation}\label{allocationcoordination}
    a_i^j(z_i,y_i) = z_i^{j},
\end{equation}
and
\begin{equation}\label{costcoordination}
    c_i^j(z_i,y_i) = y_i^{j},
\end{equation}
where $i$'s strategy is as in Proposition \ref{buyerstrategy}.
}\\
\textbf{Proof:}
As $s_{-i}$ is fixed, we omit it, in addition, we will use $u\equiv u_i \equiv u_i(s_i) \equiv
u_i(s_i;s_{-i})$. In full notation, we intend to show
$$
    u_i((d_i,p_i);s) \le u_i((z_i,y_i);s_{-i}).
$$
If there exists a seller who can fully satisfy $i$'s demand,
then $\vert\mcI_i\vert = 1$, and the case is trivial as no
coordination is necessary for a single bid. (SPECIAL CASE OF MONOPOLY? THINK!)\\
Otherwise, buyer $i$'s demand can only be satisfied by purchasing data from
multiple sellers.
We will show that $i$ may increase $\vert\mcI_i\vert$, and so decreasing
$d_i^j, \ \forall \ j\in\mcI_i$, without decreasing $u_i$.
Buyer $i$ maintains ordered set $\ell_i$ where the sellers with the
largest bid quantities are considered first; the index of seller $j^*$ defines
a minimal subset $\mcI_i$, satisfying (\ref{buyercoordinate}). By construction, 
$d_i^{j^*}$ is the minimum quantity offered by any $j\in \mcI_i$,
so $d_i^{j^*} \le d_i^j, \ \forall \ j \in \mcI_i$; $\mcI_i$ also defines the
maximum quantity bid of any $k \ni \mcI_i$.
Thus by (\ref{buyercoordinate}) and (\ref{opt-minallocation}), $\forall \ j\in \mcI_i$,
$k\ni\mcI_i$,
$$
    e_i^k(a) \le z_i^j = e_i^{j^*}(a) \le e_i^{j}(a),
$$
and so, 
\begin{equation}\label{minimaleq}
    e_i^{j^*}(a)  \le \bigg\lbrack D^{j} - \sum_{k\in \mcI^j : p_k^{j}> y_{i}^{j}} d_k^{j}\bigg\rbrack^+,
\end{equation}
The buyer valuation function (\ref{buyervaluation}), guarantees
that $\forall \ j\in\mcI_i$, $y_i^j \ge p_{i^*}^j$, where $p_{i^*}^j$ is the reserve price of seller $j$, defined in Proposision
\ref{sellerstrategy}, and is by definition the minimum price for a buyer bid to
be accepted. 
As $D_i$ is non-decreasing, $\forall \ j\in \mcI_i$,
$k\ni\mcI_i$,
$$
    D_i^j(y_i^j) \ge D_i^j(p_i^{j^*})\ge D_i^j(p_i^{k}).
$$
Furthermore, suppose there exists buyer $k\in \mcI^j$, such that
$e_k^{j^*}(a) \ge e_i^{j^*}(a)$ and so $d^j - d_k^j < z_i^{j}$, resulting in a
partial allocation. Bid $s_k \in S_i$, and so is considered in buyer $i$'s strategy. 
As buyer $i$ is allowed to choose subset $\mcI_i$, and $\mcI_i$ is a minimal
set, (\ref{buyercoordinate}) states that $n$ is such that $e_i^{j*}(a)
\ge e_k^{j^*}(a)$ for any $k$, therefore such a buyer $k$ cannot exist.
\iffalse
We have, $\forall \ j\in \mcI_i$, $k\ni\mcI_i$,
$$
    {\theta_i}'(e_i^{k}(a)) \ge {\theta_i}'(e_i^{j^*}(a)) \ge y_i^j.
$$
Now, suppose that $p_k^j = p_i^j$, however $d_k^j > d_i^j$...
\fi
Thus (\ref{minimaleq}) holds and so, by (\ref{dataallocation}),
\begin{align*}
     a_i^j(z_i,p_i) &=\min_{i\in\mcI^j}\bigg(z_i^j, \bigg\lbrack D^{j} - \sum_{p_k^{j}> y_i^{j}}
d_k^{j}\bigg\rbrack^+\bigg) \\
    &= z_i^{j}= e_i^{j^*}(a)
\end{align*}
where the last equality is by definition, and so (\ref{allocationcoordination}) is proven.
Now, since a buyer is only charged with the cost of excluding other players from the market, the
unit price does not influence the final charges. This unit price reflects the
valuation of the total resources gained from the multi-auction market, and so a buyer uses this price on all
auctions.
We have, $\forall \ j\  \in \mcI_i, \ \forall \ k \ni \ \mcI_i$,
$$
    y_i^j \ge p_i^j \ge p_{i^*}^j\ge p_i^k,
$$
and we observe that from (\ref{datacomposed}), $\bar{D}_i^j(y, s_{-i}) =
0 \ \forall \ y < p_{i^*}^j$, and so $y=0 \le \epsilon \Rightarrow e_i^j(a) = 0$,
and so clearly $z_i^k = 0, \ \forall \ k \ni \mcI_i$, and therefore,
\begin{align*}
    \displaystyle\sum_{j\in\mcI_i} c_i^j(z_i,y_i) =\sum_{j\in\mcI_i}
c_i^j(z_i,p_i),
\end{align*}
thus (\ref{costcoordination}) simply shows that changing the price $p^j_i$ to $y^j_i$ does not exclude any
additional buyers, as the bid $p^j_i$ was already above the reserve price of any
seller $j \in \mcI_i$.
We proceed to demonstrate that $x_i$ does not result in a loss of utility for
buyer $i$; we will show that
$$
    u_i \le u_i(z_i,y_i).
$$
From (\ref{allocationcoordination}), we have $a_i^{j}(z_i,y_i) = z_i^{j} = e_i^j(a(z_i,y_i))$, 
and so,
$$
    \theta_i\circ e_i^j(a(z_i,y_i)) = \theta_i\circ e_i^j(a),
$$ 
which holds $\forall \ j\in\mcI_i$.
Therefore, by the definition of utility (\ref{buyerutility}),
and the buyers' valuation (\ref{buyervaluation}), 
\begin{align*}
    &\theta_i\circ e_i(a(z_i,y_i)) - \theta_i(a)\circ e_i(a) \\
    &= u_i(z_i,y_i) - u_i =  \displaystyle\sum_{j\in\mcI_i} c_i^j -c_i^{j}(z_i,y_i)\\
    &= \sum_{j\in\mcI_i}\int_{a_i^j(z_i,p_i)}^{a_i^j} f_i(d_i^j - z) \ dz.
\end{align*}
Then, as $a_i(z_i,p_i) \le z_i^j \le a_i^j $, and noting that
$\theta_i\ge 0\Rightarrow f_i\ge 0$, we have $u_i(z_i,y_i) - u_i \ge 0$, $\forall \ j\in\mcI_i$.
%1. buyers have same d but diff p
%2. " same p but diff d i.e. d1 < d2
Finally, as we do
not increase the aggragate demand of the buyer, $\sum_{j\in\mcI_i} c_i^j(s) \le 
\sum_{j\in\mcI_i} b_i^j$, and so $x_i$ is feasible.

In effect, we are using the buyer demand
to partition the auction space based on their type (EXPLAIN). \\

\subsubsection{Incentive Compatibility}

We proceed to claim that the optimality of truth-telling holds in our
formulation, where the market functions as a hybrid of \cite{semret} and
\cite{zheng}. The opt-out buyers' market is comprised of the minimal subset of
sellers with the largest amounts of available data, described in the buyer strategy as $\mcI_i$.
To achieve incentive compatibility, we find that the opt-out buyer must choose this subset so that
its overall marginal value is greater than its market price.
The buyers' market price is calculated as the maximum of the reserve prices
of the sellers in the opt-out buyers' pool.
the market prices at the %different auctions, weighted by the data provisioning vector values.
The actual bids are
obtained from the opt-out buyers' strategy. The quantity to bid is given
by the auction mechanism, i.e. (\ref{dataprice}) and (\ref{datacomposed}), as
the maximim possible quantity of data that a buyer $i$ can bid over its seller
pool while maintaining its aggragate marginal valuation greater than the aggragate of
minimum prices maintained by the sellers in $i$'s pool.
As with a single resource, \cite{semret} and
\cite{lazar}, we show that truth-telling is optimal for the buyer, i.e. in each auction, the buyer sets
the bid price to the marginal value.

Seller $j$'s reserve price is determined by a
buyer $i \ni \mcI^j$, and therefore, even if this price is zero,
then $p^j = \epsilon \ge 0$.

We argue that if truthfulness holds \emph{locally} for both buyers and sellers, i.e. $p_i ={\theta_i}' \
\forall \ j \in \mcI_i$ and $p^j = {\theta^j}' \ \forall \ i \in\mcI^j$, then
there exists a local market
equilibrium (NOTE: NOT GLOBAL! YET). We have the following Proposition. For
completeness, we use the full notation. (BAD SENTENCE)

{
\proposition{(Incentive compatibility in local auctions)}
For any seller $j$, let time $t \in \tau$ be fixed and for any buyer $i\in
\mcI^j$, let $s_{-i}$ also be fixed. )VALIDATE: FOR ALL j) Define,
\begin{align}
\begin{split}\label{eta}
    \eta_i &= \sup\bigg\lbrace x\ge 0 : 
 {\theta_i}'(x) > \bar{P}_i^j(x)\bigg\rbrace, 
\end{split}\\
\begin{split}\label{chi}
    \chi_i &= \sup\bigg\lbrace x\ge 0: 
\displaystyle\int_0^x 
    \bar{P}_i(x) \ dx \le \sum_{j\in\mcI_i}b_i^j \bigg\rbrace,
\end{split}
\end{align}
$z = \min(\eta_i, \chi_i - \epsilon / \theta_i'(0))^+$, and for each $j \in
\mcI_i$, 
$$
    v_i^j = e_i^{j^*}(z) 
$$
and 
$$
    w_i^j = \theta_i'(z).% \circ e = \theta_i'(v_i^j).
$$
Then a (coordinated) $\epsilon$-best reply for the opt-out buyer is $t_i =
(v_i,w_i)$, i.e., $\forall \ s_i, u_i(t_i;s_{-i}) + \epsilon \ge u_i(s_i;
s_{-i})$.
%and fix $j$'s reserve price $p^j$ as is defined by (\ref{newprice}). 
With reserve prices $p^j >0$, there exists a ``truthful" local game embeddeded
in each local auction, and thus an equilibrium point for the local auction. 
}\\
\textbf{Proof:}
For the buyer, we show that $t_i$ is an $\epsilon$-best reply. That is,
$$
    u_i(t_i;s_{-i}) + \epsilon \ge u_i(s_i;s_{-i}).
$$
Let $z = \eta_i^j$. We have that $i\in\mcI^j$, 
and (\ref{buyerinversedemand}) defines $\theta_i'(z)$ as being
max of the reserve prices $p_i^j,\ \forall \ j\in\mcI_i$, therefore
(\ref{eta}) is such that,
$$
    \theta_i'(z) > \bar{P}_i^j(z),
$$ 
which implies, as (MORE) $\theta_i'$ is non-increasing and $P_i^j \ge 0$,
we have $\forall \ j \in \mcI_i$, 
\begin{align*}
    w_i^j &> P_i^j(v_i^j) \\
    \Rightarrow v_i^j &\le D_i^j(w_i^j) = D^j - \rho^j(w_i^j).
\end{align*}
And so, by (\ref{dataallocation}),
\begin{align*}
    a_i^j(t_i) &= v_i^j \\
    \Rightarrow \displaystyle\sum_{j\in\mcI_i} a_i^j(t_i) &= z.
\end{align*}
Therefore,
\begin{align*}
    \displaystyle\sum_{j\in\mcI_i}&\int_0^{v_i^j} P_i^j(x) \ dx =  \int_0^{z}
    \bar{P}_i^j(x) \ dx \\
    &= \sum_{j\in\mcI_i}\int_0^{e_i^j(z)} P_i^j(x) \ dx.
\end{align*}
Suppose $\exists \ s_i = (d_i, p_i)$ such that $u_i^j(s_i;s_{-i}) > u_i^j(t_i;
s_{-i}) + \epsilon$. The buyer coordinated strategy, from Propositions
\ref{coordinationlemma} and
\ref{buyerstrategy}, gives $s_{i^*} = (e_i\circ a_i(s), \ \theta_i'(z)) =
(z_i, c_i)$, where
for each $j\in \mcI$, $a_i^j(s_{i^*}; s_{-i}) = z_i^j$,
then clearly $u_i(s_{i*}, s_{-i}) \ge u_i(s_i, s_{-i}) \Rightarrow u_i(t_i;s_{-i}) -
u_i(s_i;s_{-i}) > \epsilon$. Denoting $z_i^j$ (fixed) as $\zeta$, 
$$
    \int_z^{\zeta} {\theta_i}'(x) \ dx- \int_z^{\zeta} \bar{P}_i(x) \ dx > \epsilon.
$$
For concave valuation functions, the first-order derivative of $\theta$ at point
$0$ gives the maximum slope of the valuation function, and so the factor $\epsilon
/ \theta'(0)$ guarantees that new bids will differ by at least
$\epsilon$, and as such, buyer $i$ will remain in
any local auction with reserve price determined by (\ref{???}). We therefore
verify that,
$$
    \displaystyle\int_{z}^{z +\epsilon / \theta_i'(0)} \theta_i'(x) \
dx \le \epsilon,
$$
and as $P_i^j \ge 0$, we have that, from the construction of $\zeta$,
$$
    \displaystyle\int_{z+\epsilon / \theta_i'(0)}^\zeta \theta_i'(x) \ dx -
\int_{z+\epsilon / \theta_i'(0)}^\zeta \bar{P}_i(x) \ dx > 0.
$$
If $\zeta > z + \epsilon / \theta_i'(0)$, then for some $\delta >0$,
$\theta_i(z+\epsilon / \theta_i'(0)+\delta) > P_i^j(z+\epsilon /
\theta_i'(0)+\delta)$, contradicting (\ref{eta}).
Now, if $\zeta \le z$, then $\theta_i'(z+\epsilon / \theta_i'(0)) < P_i^j(z+\epsilon /
\theta_i'(0))$, also a contradiction of (\ref{eta}), and so buyer $s_i$ cannot
exist.

We consider the other term in the minimization; let $z=\chi_i$. We have that
For any coordinated bid $s_i$ such that $u_i(z_i, c_i) > u_i(t_i)+
\epsilon$, then $\zeta > z
\Rightarrow c_i(s_{i}) > b_i$, and so is infeasible. 
Finally, by Lemma \ref{userunion}, for a single buyer-seller interaction, we
have that $f_i = f^j \Rightarrow \theta_i \circ e_i^j = \theta^j \circ e_i^j$.
Therefore $\theta_i' = {\theta^j}'$. We have by Proposision
\ref{sellerstrategy} and (\ref{???}eqn 30), that $\theta^j = \max_{i\in\mcI^j}
\theta_i$, and so for any $i\in\mcI^j, \theta^j(z) \ge \theta_i(z)$, and so $j$
cannot set a higher reserve price, and also cannot sell more data than the buyers'
bids, so we have that $j$'s bid is both truthful and optimal.
\\----------------------REF-------------------------\\
% NEW MARKET PRICE
\begin{align}
\begin{split}
    &\bar{P}_i(z, s_{-i}) =\displaystyle\sum_{j\in\mcI}P_i^j(z_i^j,
s_{-i}^j) \\
    &= \sum_{j\in\mcI}\bigg(\inf\bigg\lbrace y\ge 0 : 
    {D_i^j}(y,s_{-i}^j) \ge e_i^j(z) \bigg\rbrace \bigg),\\
\end{split}
\end{align}
% NEW ALLOCATION RULE
\begin{align}
\begin{split}
    D_i^j(y,s_{-i}^j) = \bigg\lbrack D^j - \sum_{p_k^j> y} d_k^j\bigg\rbrack^+,
\end{split}
\end{align}
% NEW INVERSE DEMAND
\begin{align}\label{datapriceinverse}
\begin{split}
    \bar{D}_i(y, s_{-i}) &= \displaystyle\sum_{j\in\mcI}\bigg(\sup\bigg\lbrace z\in \big[ 0,
    D^j \big] : \\
    &\bar{P_i}(z,s_{-i}^j) < y\bigg\rbrace\bigg).
\end{split}
\end{align}
% DATA ALLOCATION RULE
\begin{align}
    a_i^j(s) \\
    &= \min\bigg\lbrace e_i^j(D_i),
    \frac{d_i^j}{\sum_{p_k^j= p_i^j}d_k^j} D_i^j(p_i^j,s_{-i}^j)\bigg\rbrace.
\end{align}
% DATA COST
\begin{align}
\begin{split}
    \bar{c}_i(s) &= \displaystyle\sum_{j\in\mcI} 
p^j \bigg(ai^j(0; s_{-i}^j)
    -a_i^j(s_i^j;s_{-i}^j)\bigg).
\end{split}
\end{align}
% BUYER DEMAND 
\begin{equation}
    \rho^j(y) = \sum_{i\in\mcI : p_i^j\ge y} d_i^j, 
\end{equation}
% BUYER REVENUE 
\begin{equation}
    f_i(z) \triangleq \inf\big\lbrace y\ge 0:
        \rho_i^j(y) \ge z, \ \forall \ j \in \mcI\big\rbrace.
\end{equation}
% SELLER REVENUE 
\begin{equation}
    f^j(z) \triangleq \sup\big\lbrace y\ge 0:
        \rho_i^j(y) \ge z, \ \forall \ i \in \mcI\big\rbrace,
\end{equation}
\begin{equation}
    \theta_i \circ e_i^j = \int_0^{e_i^j(a)} f_i(z) \ dz,
\end{equation}
\begin{equation}
    \theta_i \circ e = \displaystyle\sum_{j\in\mcI}
 \int_0^{e_i^j(a)} f_i(z) \ dz.
\end{equation}
\begin{equation}
    \theta^j \circ e = \displaystyle\sum_{i\in\mcI}
 \int_0^{e_i^j(a)} f^j(z) \ dz.
\end{equation}
\begin{equation}
    d_i^j \ge e_i^j(a)
\end{equation}
\begin{equation}
    \displaystyle\sum_{i\in\mcI} e_i^j(a) \le \sum_{i\in\mcI} d^j_i \le D^j,
\end{equation}
\begin{equation}
    e_i^j(a) \le D^j - \sum_{k\in\mcI, k\ne i} e_k^j(a).
\end{equation}
\begin{equation}
   p_i^j \ge \min_{i\in\mcI}\big(p_i^j).
\end{equation}
\\----------------------------------------------------\\
(SELLER FINISH)\\
This forms a ``truthful" local game embeddeded within $j$'s auction with strategy space
restricted to $\epsilon$-best replies from buyers $\in \mcI^j$. Therefore we have that a
fixed point in the ``truthful" local game is a fixed point for the auction. We
further argue that as the set $\mcI^j$ is
computed at each bid iteration, that our result holds for time $(t+1) \in \tau$.


{
\lemma{(Static Data Nash Equilibrium)} 
}

% DATA NASH EQ
{
\theorem{(Data Nash Equilibrium)}
Using the rules of the data auction mechanism, the secondary market described in
\cite{zheng} converges to a $\epsilon$-Nash equilibrium. In the network auction
game with the data-PSP rules applied independently by each user according to their respective strategies, the secondary market converges to an $\epsilon$-Nash
equilibrium. 
}\\
\textbf{Proof:}
2. using the min price of sellers in the auction i.e. ${\theta_i}'(d_{i^*}^j) =
p^j$ is OK, \\
3. that bids are still feasible AND optimal \\
4. the algorithm achieves global economic equilibrium)\\
\textbf{NEED TO COVER:}\\
1. Change in buyer valuation \\
2. New buyers\\
3. Not enough buyers \\
4. Not enough data\\
\textbf{TRY:}\\
Sellers only act when the resources obtained by the buyers influence their
respective reserve prices, which agrees with the seller stragety of attempting
to sell their data in the first iteration. Therefore we claim there exists a
market stability and therefore, the existence of a Nash equilibrium. As the
valuation of the sellers is derived by the demand of the buyers, who are
bidding equivalent bids over a minimum subset of buyers, we claim that the seller
strategy, along with the seller constraint (\ref{??}) results in a global
market equilibrium.
We have shown that the local
equlibirium created by $j$ is stable from time $t$ to $(t+1)$. 
Now, suppose that buyer $i^*$ computes its best response $s_i^j = (v_i^j, w_i^j)$
Finally, suppose that
a buyer $k$ enters the market such that for some buyer $l\in\mcI^j$,
$$
    \displaystyle\sum_{i\in\mcI^j} p_i^je_i^j(a) + p_k^je_k^j(a) \ge
\sum_{i\in\mcI^j} p_i^je_i^j(a) - p_l^je_l^j(a),
$$
that is,



NOTES: (today)\\
2. finish seller incentive compat\\
3. work on progression\\
4. check reserve price = monopoly price


\subsection{Efficiency}
(NEED OWN WORDS)
The objective in designing the auction is that, at equilbrium, resources al-
ways go to those who value them most. Indeed, the PSP mechanism does
have that property. This can be loosely argued as follows: for each player,
the marginal valuation is never greater than the bid price of any opponent
who is getting a non-zero allocation. Thus, whenever there is a player j
whose marginal valuation is less than player i 's and j is getting a non-zero
allocation, i can take some away from j , paying a price less than i 's marginal
valuation, i.e. increasing u i , but also increasing the total value, since i 's
marginal value is greater. Thus at equilibrium, i.e. when no one can unilat-
erally increase P their utility, the total value is maximized.

\subsection{Convergence}

\end{multicols}

\begin{thebibliography}{9}

\bibitem{zheng}
L. Zheng, C. Joe-Wong, C. W. Tan, S. Ha and M. Chiangs, 
\textit{Secondary markets for mobile data: Feasibility and benefits of traded
data plans}, 2015 IEEE
Conference on Computer Communications (INFOCOM), Kowloon, 2015, pp. 1580-1588.

\bibitem{lazar}
A. A. Lazar and N. Semret, 
\textit{“Design and Analysis of the Progressive Second Price Auction for Network
Bandwidth Sharing,”} Telecommunication Systems, Special Issue on Network Economics, 2000.

\bibitem{semret} 
N. Semret, 
\textit{“Market Mechanisms for Network Resource Sharing,”
Ph.D. thesis}. 
Columbia University, 1999.

\bibitem{tuffin}
Bruno Tuffin,
\textit{Revisited Progressive Second Price Auction for Charging
Telecommunication Networks}.
[Research Report] RR-4176, INRIA, 2001.

\bibitem{qpsp}
Clare W. Qu, Peng Jia, and Peter E. Caines,
\textit{Analysis of a Class of Decentralized Decision Processes: Quantized
Progressive Second Price Auctions},
46th IEEE Conference on Decision and Control, New Orleans, LA, 2007, pp.
779-784.

\bibitem{cmhk}
\textit{China Mobile Hong Kong Company Limited}.
\url{https://www.hk.chinamobile.com/en/corporate_information/Data_N_VAS/DataTopup/index.html}.
2018

\bibitem{iia}
Andreu Mas-Colell, Michael Dennis Whinston, and Jerry R. Green, 
\textit{Microeconomic Theory}. Oxford University Press. 1995. p. 794.

\bibitem{matt}
Matthias Roggendorf and Fernando Beltr{\'a}n,
\textit{Simulation Model for the Dynamic Allocation of Network Resources in a
Competitive Wireless Scenario}. Mobility Aware Technologies and Applications, Springer Berlin Heidelberg. 2005.
pp. 54 - 64.

\bibitem{diffserve}
N. Semret, R. R. F. Liao, A. T. Campbell and A. A. Lazar, 
\textit{Pricing, provisioning and peering: dynamic markets for differentiated Internet services and
implications for network interconnections}, in IEEE Journal on Selected Areas in
Communications, vol. 18, no. 12, pp. 2499-2513, Dec. 2000.

\end{thebibliography}


\end{document}

