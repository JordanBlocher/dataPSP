\documentclass[12pt]{article}
 
\usepackage[text={6in,8.1in},centering]{geometry}

\usepackage{enumerate}
\usepackage{amsmath,amsthm,amssymb}
\usepackage{mathrsfs} % to use mathscr fonts

\usepackage{multicol}

\usepackage{epstopdf}
\usepackage{caption,subcaption}
\usepackage{pstricks}
\usepackage{pst-solides3d}
\usepackage{pstricks-add}
\usepackage{graphicx}
\usepackage{pst-tree}
\usepackage{pst-poly}
\usepackage{calc,ifthen}
\usepackage{float}\usepackage{multicol}
\usepackage{multirow}
\usepackage{array}
\usepackage{longtable}
\usepackage{fancyhdr}
\usepackage{algorithmicx,algpseudocode}
\usepackage{changepage}
\usepackage{color}
\usepackage{listings}
\usepackage{fancyvrb}
\usepackage{verbatim,moreverb}
\usepackage{courier}

\lstset{ %
language=C++,               
basicstyle=\footnotesize,
numbers=left,                  
numberstyle=\tiny,     
stepnumber=1,         
numbersep=5pt,         
backgroundcolor=\color{white},  
showspaces=false,               
showstringspaces=false,         
showtabs=false,                 
columns=fullflexible,
frame=single,          
tabsize=2,          
captionpos=b,       
extendedchars=true,
xleftmargin=17pt,
framexleftmargin=17pt,
framexrightmargin=17pt,
framexbottommargin=4pt,
breaklines=true,       
breakatwhitespace=false, 
escapeinside={\%*}{*)}       
}

\newenvironment{block}{\begin{adjustwidth}{1.5cm}{1.5cm}\noindent}{\end{adjustwidth}}

\newtheorem{proposition}{Proposition}[section]
\newtheorem{theorem}{Theorem}[section]
\newtheorem{lemma}{Lemma}[section]
\newtheorem{corollary}{Corollary}[section]
\theoremstyle{definition}
\newtheorem{definition}{Definition}[section]

 
\def\verbatimtabsize{4\relax}
\def\listingoffset{1em}
\def\listinglabel#1{\llap{\tiny\it\the#1}\hskip\listingoffset\relax}
\def\mylisting#1{{\fontsize{10}{11}\selectfont \listinginput[1]{1}{#1}}}
\def\myoutput#1{{\fontsize{9}{9.2}\selectfont\verbatimtabinput{#1}}}

\newcommand{\vs}{\varsigma}
\newcommand{\mcL}{\mathcal{L}}
\newcommand{\mcI}{\mathcal{I}}
\DeclareMathOperator*{\argmax}{arg\,max}
\DeclareMathOperator*{\argmin}{arg\,min}
 
\headsep25pt\headheight20pt
 
 
\pagestyle{fancyplain}
\rhead{\fancyplain{}{\small\bfseries Blocher, Jordan}}
\cfoot{\ \hfill\tiny\sl Draft printed on \today}
 
 
\setlength{\extrarowheight}{2.5pt} % defines the extra space in tables
 
\begin{document}
\begin{multicols}{2}

\section{Introduction}
China Mobile Hong Kong (CMHK) recently introduced such a secondary market.
CMHK’s 2cm (2nd exchange market) data exchange platform allows users to submit
bids to buy and sell data, with CMHK acting as a middleman both to match buyers
and sellers and to ensure that the sellers’ trading revenue and buyers’
purchased data are reflected on customers’ monthly bills. \cite{zheng}

In this work, we propose a distributed progressive second price (PSP) auction in
order to maximize social utility. We show that for cellular data allocated between multiple users there
exists an $\epsilon$-Nash market equilibria when all users bid their real marginal
valuation of the mobile data offered in the secondary market described in
\cite{zheng}. The P2P auction's (as in all auctions),
demand information is not known centrally, rather it is distributed in the
buyers' valuations. The mechanism for an auction is defined
as \emph{distributed} when the allocations at any element depend only on
\emph{local} state: the quantity offered by the seller at that element, and the
bids for that element only. In this work, the proposed mechanism allows sellers submit bids to buyers
directly; there is no entity that holds a global market knowledge.

We suppose that each seller (resp. buyer) can submit a bid to the secondary
market consisting of (i) an available (required) quantity and (ii) a unit-price (calculated
using its own demand functions). Buyers submit bids cyclically until an
($\epsilon$-Nash) equilibrium is reached where $\epsilon$ corresponds to a bid
fee to be paid to the ISP on completion of the transaction. (FEE IS FIXED OR
PER-UNIT? (will change market incentives slightly))
(DO WE MODEL ISP PROFIT? FUTURE WORK)

\subsection{Distributed Progressive Second Price Auctions}

The distributed PSP auction first introduced in \cite{lazar} forms a part of the
overall market based allocation model. Consider a noncooperative game
where a set of $\mcI = \lbrace 1,\cdots,I\rbrace$ users buy a fixed
amount of resource $D_i$ from a set of resources $\mcL = \lbrace 1,\cdots,L\rbrace$. Suppose
each user $i \in \mcI$ makes a bid $s_i^l = (p_i^l, d_i^l)$ to the
seller of resource $l$,
where $p_i^l$ is the unit-price the user is willing to pay and $d_i^l$ is the
quantity the user desires. The \emph{bidding profile} forms a grid, $s \equiv
[s_i^l] \in \mcL \times \mcI$ (BAD NOTATION?
SEEMS OK)  and $s_{−i}^l \equiv [s_1^l , \cdots , s_{i−1}^l , s_{i+1}^l , \cdots
, s_I^l]_{l\in\mcL}$ is
the profile of user $i$’s opponents. In addition, the user type now includes a
\emph{routing variable}, $r_i\in\lbrace 0, 1\rbrace^L$ to manage throughput in
an arbitrary collection of nodes where $r_i^l >0$ if and only if $l(i)$ is on
$i$'s path, i.e. $l(i) \in \lbrace l\in \mcL : r_i^l >0\rbrace$. 
The market now shares $L$ resources, with allocation:
$$
    \hat{D_i^l}(y,s_{-i}^l) = \bigg\lbrack D^l -
\displaystyle\sum_{p_k^l > y, k\ne i} d_k^l\bigg\rbrack^+.
$$
The market price function (MPF) of user $i$ is defined as:
% MARKET PRICE 
\begin{align*}
    \tilde{P_i}(z, s_i) &= \displaystyle\sum_{l\in\mcL}P_i^l(z_i^l,
s_{-i}^l)r_i^l \\
    = \displaystyle\sum_{l\in\mcL}&\bigg(\inf\bigg\lbrace y\ge 0 :
\hat{D_i^l}(y,s_{-i}^l)\bigg\rbrace r_i^l\bigg),
\end{align*}
which is interpreted as the aggragate minimum price a user bids over its route in order to obtain the
resource $z$ given the opponents’ profile $s_{−i}$. Its inverse function
$\tilde{D_i}$ is defined as follows:
% INVERSE PRICE (ALLOCATION)
\begin{align*}
    \tilde{D}_i(y, s_i) &= \sup\bigg\lbrace z\in \bigg( 0,
\min_l\hat{D}_i^l/r_i^l\bigg) : \\
    &\tilde{P_i}(z,s_{-i}^l) < y\bigg\rbrace,
\end{align*}
which means the maximum available quantity at a bid price of $y$ given
$s_{−i}$. With this notation, the modified PSP allocation rule \cite{tuffin} is defined
as:
% ALLOCATION RULE
\begin{align*}
    \tilde{a_i}(s) &= \min_i\big\lbrace d_i^{l(i)}, \\
 &\frac{d_i^{l(i)}}{\sum_{k:p_k^{l(k)}=p_i^{l(i)}}d_k^{l(k)}} \tilde{D_i}(p_i^{l(i)},
s_{-i}^l)\big\rbrace,
\end{align*}
\begin{align*}
    \tilde{c_i}(s) &= \displaystyle\sum_{j\ne i}\bigg\lbrack\sum_{l\in\mcL}
p_j^{l(j)} \bigg(\tilde{a_j}(0; s_{-i}^l)\\
    &-\tilde{a_j}(s_i^l;s_{-i}^l)\bigg) r_i^l\bigg\rbrack,
\end{align*}
 $a_i^{l(i)}$ denotes the quantity user $i$ obtains by a bid price
$p_i^{l(i)}$ (when the
opponents bid $s_{−i}^l$) and the charge to user $i$ by the seller is denoted
$c_i^{l(i)}$. $\sum_{l\in\mcL} c_i^{l(i)}$ is interpreted to be the total cost incurred in the system if
user $i$ is removed from the auction. Note that the allocation rule is modified
from \cite{lazar} according to \cite{tuffin}, so that buyers with identical unit-price
$p_i$ are not rejected.

We further modify the allocation rule in \cite{tuffin} in order to suit our
model. We argue that as there is no capacity bottleneck in our network as in
\cite{lazar} and \cite{semret}, we are able to increase the bid quantity and
take the maximum in the allocation rule. 
We propose a \emph{data allocation rule}:
{
% DATA ALLOCATION RULE
\proposition{(Data allocation rule)}
\begin{equation}\label{allocation}
    \bar{a}_i(s) = \max_i\big\lbrace d_i, \frac{d_i}{\sum_{k:p_k= p_i}d_k}
D_i(p_i,s_{-i})\big\rbrace
\end{equation}
} \\
Intuitively, this rule indicates that a seller will allocate its data to the
buyer with the highest demand that can be satisfied (within its budget). 
An allocation rule is feasible \cite{lazar} if $\forall \ s$,
$$
    \displaystyle\sum_{i\in\mcI} a_i(s) \le Q
$$
and $\forall \ i\in\mcI$,
\begin{align*}
    a_i(s) &\le d_i \\
    c_i(s) &\le p_id_i.
\end{align*}
We claim
that (\ref{allocation}) is feasible, to be shown in our analysis (SAY
BETTER).
The modified allocation rule changes the market
dynamics by adding a demand threshold to buyer behavior; the strategy is now to
submit bids to a (thresholded) subset of users, rather than submitting bids to all sellers as in \cite{lazar}, however
we still hold the assumption that data is infinitely divisibile. 
We will introduce a player type called an \emph{opt-out buyer} in order to perform our analysis. The
opt-out buyer restricts its pool of sellers by only submitting bids to sellers
who have sufficient data to meet their requirement.
(NOTE: TAKE MAX ONLY OR NEED TO INTRODUCE
A GAP VARIABLE?), similar to \cite{zheng}. 
We intend to show that the pricing model (\ref{price}) is sufficient to attain the desirable
property of truthfullness through incentive compatibility. (MOVE THIS? NEED OWN
WORDS?) We
reason that our formulation upholds the \emph{exclusion-compensation
principle}, and is a valid progressive second price auction to the extent that buyer $i$
pays for its allocation so as to exactly cover the ``social opportunity cost" which is given by the declared willingness to pay (bids) of the users who are excluded by $i$'s
presence, and thus also compensates the seller for the maximum lost potential
revenue \cite{lazar}.

(NOTE: move this?)
We extend the P2P rules as in \cite{semret} to include a \emph{local} market
price function as determined by the subset of nodes participating in the
auction, where the user is the auctioneer. Therefore the influence of the greater market on the individual
auctions will be influenced only by the submission of bids from buyers to
sellers. As a buyer may have access to multiple auctions, the sellers will be
dynamically influenced by the $\epsilon$-best replies from the buyers. The
valuation function of seller $j$ is dependent on the buyers demand, and is
modeled as a function of their potential revenue \cite{semret}. 

Absent the cost or revenue from trading data, users gain utility from consuming
data. We use the $\alpha$-fair utility functions \cite{zheng} to model the
usage utility from consuming $d$ amount of data:
\begin{equation}
    \theta(d) = \frac{\sigma d^{1-\alpha}}{1-\alpha}
\end{equation}
(AUGH! NEED X TO BE SO BIG COMPARED TO ALPHA! can we assume a minimum ask?)
where $\sigma$ is a positive constant representing (the scale) of the usage
utility and we take $\alpha \in [0, 1)$.
We verify that the user valuation above satisfies the conditions for an
\emph{elastic demand function}: (NOTE: this part seems in the wrong place)

\definition{\cite{lazar}}
A real valued function $\theta(\cdot)$ is an \emph{(elastic) valuation
function} on $[0, D]$ if 
\begin{itemize}
    \item $\theta(0) = 0$; \\
        \emph{Verification:} (obvious)
    \item $\theta$ is differentiable; \\
        \emph{Verification:} The derivative, $\sigma d^{-\alpha}$, is positive assuming non-negative
data requirements.
    \item $\theta ' \ge 0$, and $\theta '$ is non-increasing and continuous; \\
        \emph{Verification:} $U$ is differentiable for all $d$, and therefore
continuous. Its derivative is a negative exponential, and so is non-increasing.
    \item There exists $\gamma > 0$, such that for all $z \in [0,D]$, $\theta
'(z) > 0$ implies that for all $\eta \in [0, z), \theta '(z) \le \theta '(\eta)
- \gamma(z - \eta)$. \\
        \emph{Verification:} Without loss of generality, we may set the scaling
constant $\sigma=1$, and compute the curvature $\gamma(\xi)$, where by definition,
$$
    \gamma = \frac{\theta''}{(1+\theta')^{3/2}} = \frac{-\alpha
\xi^{-\alpha-1}}{(1+\xi^{-2\alpha)^{3/2}}}.
$$
Using a Taylor theorem approximation,
\begin{align*}
    z^{-\alpha} &\le \eta^{-\alpha} + \frac{-\alpha
d^{-\alpha-1}}{(1+\xi^{-2\alpha)^{3/2}}}(z-\eta) \\
    & \le \eta^{-\alpha} +
\frac{\alpha}{2\sqrt{2} \xi}(z-\eta)  \\
    &\le \eta^{-\alpha} +
\frac{\alpha}{\xi}(z-\eta).
\end{align*}
Now, using Taylor repeatedly, simplifying and taking the limit as $\eta \rightarrow z$,
\begin{align*}
    z^{-\alpha} - \eta^{-\alpha} &\le -\alpha\eta^{-\alpha-1}(z-\eta) \\
    & \le \frac{-\alpha}{\xi}(z-\eta).
\end{align*}
And so, since $\xi \le \eta^{\alpha+1}$,
we may set
$$
    \gamma \ge
\frac{-\alpha\eta^{-(\alpha+1)^2}}{\big(1+(\eta^{-2\alpha(\alpha+1)})^{3/2}},
$$
which holds in the case that $z > 1$, and so assuming that there must be at least one
unit of data required for a user to have
a valuation,  we have that the concavity of $\theta'$ is shown by Squeeze theorem.
\end{itemize}

% USER UTILITY (7)
In the remainder of this paper we omit the bar in the data allocation rule for
simplicity of notation, so $\bar{a}_i = a_i$. We may now define the user's utitlity function as
\begin{equation}\label{utility}
    u_i = \theta_i(a_i(s)) - c_i(s).
\end{equation}

Under the PSP rule, \cite{lazar} shows that given the opponents bids $s_{-i}$,
user $i$'s $\epsilon$-best response $s_i = (w_i, v_i)$ as a Nash move
(where $s_i$ is chosen to maximize $i$'s utility with $s_{-i}$ held constant), is
given by:
\begin{align}
\begin{split}
    v_i &= \sup\bigg\lbrace d \ge 0 : \theta '(d) > P_i(d), \\ 
&\displaystyle\int_0^d P_i(\eta) \ d\eta \le b_i\bigg\rbrace -
\frac{\epsilon}{\theta_i'(0)} \\
&\qquad\qquad \text{(best quantity reply)} \\
\end{split}\\
    w_i &= \theta_i'(v_i) \quad \text{(best unit-price reply)},
\end{align}
where $\epsilon > 0$ is the bid fee, $b_i$ is user $i$'s budget, and every user
has an elastic demand function.

\emph{NOT GOOD\\
In the Secondary Market \cite{zheng}, we intend to show that it is optimal for a
buyer to fully satisfy thier demand in a single auction, that is, a buyer
will purchase its required data from a single seller. In the sub-optimal case,
the buyer will split its bid among multiple sellers. }

(ALSO NOT GOOD)
In the Secondary Market \cite{zheng}, we intend to show that it is optimal for a
buyer to satisfy thier demand by submitting equal bids to the subset of sellers
$\in \mcI$ who
meet their data threshold. (FIGURE OUT, WITH A
DATA GAP OR JUST THE MAX? PRICE DEPENDENT OR JUST DATA?)

\section{Related Work}
(HERE ADDRESS THE CENTRALIZED AUCTIONS FROM OTHER PAPERS... ISP CONTROLLED
ETC.. MENTION DISTRIBUTED NON-DATA PSP? AD-HOC? NOT SURE, IS MATCHING ALGORITHM
A NON-ISSUE? (probably)) 

\section{The Problem Model}
\subsection{The Secondary Market}

We consider the set of $\mcI = \lbrace 1, \cdots, I\rbrace$ users who purchase or sell
data from other users. A buyer is matched with sellers who enough leftover data
to satisfy thier demand, and will submit bids in order to maximize thier (private) valuation.
A user's identity $i \in \mcI$ as a subscript indicates that the user
is a buyer, and a superscript indicates the seller.
Suppose user $i$ is buying from user $j$. A bid $s_i^j = (d_i^j, p_i^j)$,
meaning $i$ would like to buy from $j$ a quantity $d^j_i$ and is willing to pay
a unit price $p^j_i$. (STILL TRUE?) Without loss of generality, we assume that all users bid in all
auctions; if a user $i$ does not submit a bid to $j$, then this means that
the user has the exact amout of data they require, or that seller $j$ does not have
enough data to satisfy the buyer's demand, and we simply set $s^j_i = (0, 0)$.
A seller $j$ places an ask $s^j = (d^j, p^j)$, meaning $j$ is offering a
quantity $d_j^j$ , with a reserve unit price of $p_j^j$ . In other words, when the subscript and
superscript are the same, the bid is understood as an offer in the secondary
market; we assume that data is a unary resource belonging to the seller, and
therefore can identify the data (for sale) with the identity of the user.
With this restriction, we note that since $i$ is not a seller, $d_i^i = 0$ and
$a^i = 0$. We may simplify the notation so that $s_j^j = (d_j^j, p_j^j) = s^j = (d^j, p^j)$.

Based on the profile of bids $s^j = (s^j_1, \cdots s^j_I)$, seller $j$ computes
an allocation $(a^j, c^j) = A^j(s^j)$, where $a^j_i$ is the quantity given to
user $i$ and $c^j_i$ is the total cost charged to user $i$. $A^j$ is the
allocation rule of seller $j$. It is feasible if $a^j_i \le d_i^j$ (THIS MAY
NEED A SLIGHT MODIFICATION), and $c^j_i \le  p^j_i d_i^j$.

\section{Distributed PSP Analysis}
\subsection{User Behavior}

% OPT-OUT BUYER
We define a \textbf{opt-out buyer} as a user $i\in\mcI$ with utility
function as in \cite{semret},
\begin{equation}\label{opt-utility}
    u_i = \theta_i \circ e_i(a) - \displaystyle\sum_j c_i^j,
\end{equation}
where $e_i : [0, \infty) \rightarrow [0,\infty)$ is the expectation (FUTURE
WORK) that user
$i$ finds a matching seller $j$. An opt-out buyer's valuation
depends only on a scalar $e_i(a)$ which is a function of the
quantities of all the available data for sale in the secondary market. 

Suppose the total amount of seller $j$'s data on the network at the instance that
user $i$ joins the auction is $\bar{b}^j$. For the service to function as desired,
the data transfer from each seller cannot exceed the total amount they have available,
i.e. $a_i^j \le \bar{b}^j$. (CHECK THE MATH!) This will hold simultaneously for each $j \in
\mcL_i$ if and only if $\bar{b}^j \ge \max_j \ a_i^j \vs_i^j = e_i(a)$.
(TERRIBLE NONSENSE) Thus $e_i(a)$
lower-bounds the amount of data that each seller $j$ has in an auction.
We define the set of \emph{qualifying sellers} for buyer $i$,
% QUALIFYING SELLERS
\begin{equation}\label{sellers}
     \mcL_i = \lbrace j \in \mcI : \bar{b}^j \ge e_i(a) \rbrace,
\end{equation}
and denote the set of buyers who submit bids to $j$ as $\mcI^j = \lbrace i\in\mcI : j \in
\mcL_i\rbrace.$

Note that with our formulation, if a seller $j$ does not meet a buyer $i$'s data requirements, a
rational (utility-maximizing) buyer will set $s_i^j = 0$, i.e. they will not
place a bid, as in \cite{zheng}. 

Buyer $i$'s valuation may now be interpreted as a unit valuation $\theta_i$, scaled
by a function of quantity desired from the market. 
We define, in addition to the valuation and budget of user $i$, a generic
\textbf{data-provisioning vector} $\vs_i$, held by buyer $i$ as part of its type. We propose the
following (novel) strategy,
{
\theorem{(Opt-out buyer strategy)}\label{strategy}
Define, for any allocation $a$,
% OPT-OUT FUNCTION
\begin{equation}\label{opt-out}
    %e_i^j(a) \triangleq \frac{a_i^j}{\vs_i^j} + a_j^i,
    e_i^j(a) \triangleq a_i^j\vs_i^j,
\end{equation}
and let
\begin{equation}
    e_i(a) \triangleq \min\bigg(\max_j e_i^j(a), \delta\bigg),
\end{equation}
where $\delta = \max_k d_k$ where $k\in \mcL_i$ and $s_i = (\delta,
\theta'(\delta)) \in S_i(s_{-i})$, where 
$$
    S_i(s_{-i}) = \bigg\lbrace s_i : \displaystyle\sum_j c_i^j(s_i;s_{-i}) \le b_i\bigg\rbrace,
$$
is given as in \cite{lazar} and $b_i$ is user $i$'s budget. In other words, the
buyer will always make a maximum feasible bid (WHY IS IT FEASIBLE? BETTER MIN
AND STAY IN AUCTION? PRICE OR QUANTITY?). 
Under this rule, the secondary market described in \cite{zheng} converges to a $\epsilon$-Nash equilibrium.
}\\
\textbf{Proof:}
We will prove the theorem by asserting that our formulation upholds the VGC
mechanism that is given in the properties of modern PSP. Namely,
\begin{itemize}
    \item Incentive compatibility
    \item Efficiency 
    \item Convergence
\end{itemize}

The seller, in our analysis, is an extension of the buyer, where the valuation
$\theta^j$ is dependent on the buyers, we will assume that buyers and sellers
are separated (a seller does not also buy data and vice versa). 

The natural utility is the potential
profit $u^j = \theta^j\circ e$, where $\theta^j$ is the
potential revenue from the sale of data composed with opt-out value
$e_i(a)$ for each buyer $i$ that has $j$ in its set of qualified buyers
$\mcI^j$. Note that we have chosen to omit the original cost of the data
paid to the ISP, as a discussion of mobile data plans is outside the scope of this
paper. This is intuitive as a seller who does not meet the demand of \emph{any} buyer will not
participate in any auction. Additionally, as the
value of the secondary market is its ability to fufill the data demands of its
users, we determine the
valuation of the transaction between seller $j$ and buyer $i$ is well-posed,
and the form of (\ref{opt-out}) is justified. 
\iffalse
We may now define the maximum
quantity of \emph{data} at bid price $y$ given $s_{-i}$:
% NEW DEMAND
\begin{align}
\begin{split}
    \bar{D}_i(y,s_{-i}) &= \max_j \bigg(D_i^j(y,s_{-i}^j)\ \circ\ e_i\bigg)\\
    &= \max_j\bigg\lbrack D^j - \sum_{p_k^j> y} d_k^j\vs_k^j\bigg\rbrack^+.
\end{split}
\end{align}
\fi
(MOVE?)
Define, for a buyer $i$,
$$
    \ell_i =\min_{n\in\mcI} \bigg( \argmax_{\mcL_i' \subset \mcL_i, \vert\mcL_i'\vert =
n}\sum_{j\in\mcL_i'} d^j\bigg),
$$
and
\begin{equation}\label{minset}
    \bar{\mcL_i} = \bigg\lbrace j \in  \ell_i:
\displaystyle\sum_{j\in \ell_i} d^j = d_i \bigg\rbrace.
\end{equation} 
where $\bar{\mcL_i} \subset \mcL_i$ is the
minimal subset of sellers (which may be empty or unary) whose data can sum to \emph{exactly}
(DO I WANT EXACTLY? ... YES) the demand $d_i$.
We note that as we assume $\forall \ j \in \mcL_i, d^j \ge
d_i$, the subset definition is expressed here for notational ease.
Additionally, we define, for a seller $j$,
$$
    \beta^j = \bigg\lbrace i\in\bar\mcI^j: \sum_i
d_i^j \le \bar{b}^j \bigg\rbrace,
$$
and so, similarly to (\ref{minset}),
\begin{equation}\label{maxset}
    \bar{\mcI}^j = \max_{n\in\lbrace 1,\cdots, \vert \beta_j\vert\rbrace}\bigg(\argmax_{{\mcI^j}' \subset \mcI^j,
\vert{\mcI^j}'\vert = n}\sum_{i\in{\mcI^j}'} d_i\bigg).
\end{equation}
as the maximum subset of buyers $\in\mcI^j$ that $j$ can satisfy with budget
$\bar{b}^j$. The set $\beta^j$ is computed at each bidding iteration,
i.e., for $\tau >0$, 
\begin{align*}
    \beta^{j(\tau+1)} &= \bigg\lbrace i\in\bar\mcI^j: \sum_i
        d_i^j \le \bar{b}^j \bigg\rbrace, \\
    \text{where}&\\
    \bar{b}^j &= b_j - \sum_{t\in
\tau} d_i^{j(t)}
\end{align*}
and $b^j$ is $j$'s original budget.


\subsection{Data Auction Mechanism}
We may now proceed to formally define the PSP auction.
The market price function (MPF) for a user in the secondary market
can now be described as follows:
% NEW MARKET PRICE
\begin{align}
\begin{split}
    \bar{P}_i(z, s_i) &= \displaystyle\sum_{j\in\bar\mcL_i}P_i^j(z_i^j,
s_{-i}^j) \\
    &= \sum_{j\in\bar\mcL_i}\bigg(\inf\bigg\lbrace y\ge 0 : 
    {\bar{D}_i^j}(y,s_{-i}^j)\bigg\rbrace \bigg),\\
\end{split}
\end{align}
where
\begin{align}
\begin{split}
    \bar{D}_i^j(y,s_{-i}^j) &= D_i^j(y,s_{-i}^j)\ \circ\ e_i\\
    &= \bigg\lbrack D^j - \sum_{p_k^j> y} d_k^j\vs_k^j\bigg\rbrack^+,
\end{split}
\end{align}
which is interpreted as the minimum bid price, as in \cite{lazar}.
The inverse function
$\bar{D}_i$ is defined as
\begin{align}
\begin{split}
    \bar{D}_i(y, s_i) &= \sup\bigg\lbrace z\in \bigg( 0,
\sum_{j\in\bar\mcL_i} \bar{D}_i^j \bigg) : \\
    &\bar{P_i}(z,s_{-i}^j) < y\bigg\rbrace.
\end{split}
\end{align}
The rules defined above describe the mechanism for the PSP auction given in
\cite{lazar}, however, in order to apply PSP to the data-sharing secondary market described in
\cite{zheng} the idea of a ``route" is repurposed. We claim that in the
data-sharing model, it is not necessary to bid on a minimal route, and that our
formulation not only holds the desired VCG qualities, but minimizes the message
space and auction duration, resulting in a faster convergence time than the
classic throughput problem (EEEK!).

The potential revenue for $j$ at unit price $y$ is determined by the
demand of the buyers. $\forall \ y\ge 0$, we determine that the demand is given by,
% BUYER DEMAND 
\begin{equation}\label{datademand}
    \rho^j(y) = \sum_{i\in\bar\mcI_j : p_i^j\ge y} d_i^j, 
\end{equation}
with ``inverse"
% SELLER REVENUE 
\begin{equation}\label{revenue}
    f^j(z) \triangleq \sup\big\lbrace y\ge 0:
        \rho_i^j(y) \ge z \ \forall \ i \in \bar\mcI_j\big\rbrace.
\end{equation}
For a given demand $\rho^j$, $f^j$ maps a unit of data to the highest price at
which it could be sold to any buyer $i\in \bar{\mcI}^j$.
Now, in order to derive the seller's valuation, we have the following,
{
% SELLER CONSTRAINTS
\proposition{(Seller constraints)}\label{constraints}
Let $j\in\mcL_i$ be a seller, and fix its allocation to $(a^j,c^j)$.
Formalizing the quantity threshold, the seller must satisfy the quantity
constraint, $\forall \ i \in\bar\mcI^j$,
\begin{equation}\label{quantity}
    d^j \ge d_i = e_i(a),
\end{equation}
and
\begin{equation}\label{budget}
    \displaystyle\sum_{i\in\bar\mcI^j} e_i^j(a) \le \sum_{i\in\bar\mcI^j} d^j_i \le b^j.
\end{equation}
In addition, for a seller who does not sell at a loss, the reserve price must
satisfy, $\forall \ i \in \bar\mcI^j$,
\begin{equation}\label{reserveprice}
   p^j \ge \min_{i\in\bar\mcI_j}\big({\theta_i}'(d_i^j)\big).
\end{equation}
}\\
\textbf{Proof:}
The first statement is an assumption, which we may enforce by (\ref{opt-out}),
and as a seller cannot sell more data than thier bid, (\ref{budget}) enforces
that there is a natural budget constraint for the seller. (\ref{reserveprice})
follows from the assumption that $j$ does not sell at a loss.

We may now define the seller $j$'s valuation. 
{
\proposition{(Seller valuation)}
% SELLER VALUATION
For any $i\in\bar\mcI^j$, 
\begin{equation}
    \theta_i^j = \int_0^{e_i^j(a)} f^j(z) \ dz.
\end{equation}
defining the function
$$
\mu_i^j = \frac{1}{\sum_{k\in\mcI_j:p_k^j=p_i^j} k},
$$
and an even smaller subset $\varrho\subset\bar\mcI^j$:
$$
    \varrho = \bigg\lbrace i \in \argmax_{\bar\mcI_j'\subset\bar\mcI_j}
\displaystyle\sum_{i\in\bar\mcI'} p_i^j : p_i^j \ge p_k^j \ \forall \ k\ne
i\bigg\rbrace,
$$
it follows that 
\begin{equation}\label{valuation}
    \theta^j = \displaystyle\sum_{i\in\varrho}
 \int_0^{e_i^j(a)} \mu_i^j f^j(z) \ dz.
\end{equation}
}
(COULD SIMPLIFY A LOT..)
\textbf{Proof:} 
We have that $\sum_{i\in\bar{\mcI_j}} d_i^j = 
\sum_{i\in\bar{\mcI}^j} e_i$ from (\ref{quantity}) and (\ref{budget}),
i.e. a seller will try and sell all of its data. The remainder of the proof
follows as in \cite{semret}.

\subsection{Formulation?}
Consider a user seeking to prevent
data overage by purchasing enough data from a subset of other network users.
This user $i$ can be modeled as a opt-out buyer where, as in \cite{semret}, $\vs_i^j$ denotes the
fraction of user $j$'s data aquired by user $i$. In order to form the
distributed auction, we set $\vs_i^j=1$ for all sellers who offer the enough data to meet the needs required by buyer $i$,
$\vs_i^j = 0$ for all other $j$. We intend to show that this does not affect thier valuation, and
indeed, in this network setting, results in a shared network optima (a
global optimum). The formulation is inversely analogous to the thinnest allocation route for
bandwidth given in \cite{lazar}. Reasonably, if only a single seller is available, then
(\ref{opt-utility}) reduces to the original form (\ref{utility}), defined in
\cite{semret} as ``a simple buyer at a single resource element", in other
words, (\ref{valuation}) where $i$ is fixed (MAKES SENSE?).

\section{VCG Analysis}
\subsection{Equilibrium}

Consider an opt-out buyer $i\in\mcI$. 
Due to (\ref{opt-out}), $i$ only has an incentive to change its bid
quantity if it increases its opt-out value $e_i$. 
(I THINK WE NEED PRICE TO COME IN SOON... MODIFIED ALLOCATION RULE!)
We show that, without loss of utility, $i$ can increase its
bid quantities $d_i$ to the level where the opt-out value $e_i$ 
 is the same for each qualifying seller $j \in\bar\mcL_i$. Therefore buyer
coordination holds in the secondary data market under our assumptions.
We note that in our analysis, by convention, $\infty\cdot 0 = 0$.
{
\lemma{(Opt-out buyer coordination)} Let $i\in\mcI$ be a opt-out buyer.
For any profile $s_i = (d_i, p_i)$, let $a \equiv a(s)$ be the resulting data
tranfer. For a fixed $s_{-i}$, a better reply for $i$ is $x_i =
(z_i,p_i)$ where $\forall \ j \ne i$,
$$
    z_i^j = \big\lbrack e_i(a) \big\rbrack},
$$
and
\begin{equation}\label{coordination}
    a_i^j(z_i,p_i) = z_i^j.
\end{equation}
}\\
\textbf{Proof:} (NOTE: DID NOT ADDRESS CLUTTERED NOTATION... DO LATER) We will show that
\begin{align*}
    u_i(s_i;s_{-i}) &\equiv u_i(d_i,p_i) \le u_i(z_i,p_i) \\
   &\equiv \theta_i(a_i((d_i,p_i);s)\vs_i) - c_i(s)\\
    & \le \theta_i(a_i((z_i,p_i);s_{-i})\vs_i) \\
    &-c_i((z_i,p_i);s_{-i}).
\end{align*}
Using (\ref{allocation}), we have $\forall \ l \in \mcL_i$, 
$$
    z_i^l \ge a_i^l(s) \ge \bar{D}_i^l((d_i, p_i^l);s), \\
$$
therefore,
\begin{align*}
    a_i^l((z_i,p_i);s)) &= \max\bigg\lbrace z_i^l, \\
    &\frac{z_i^l}{\sum_{k:p_k^l\ge p_i^l} d_k^l}
D_i^l((z_i,p_i^l);s)\bigg\rbrace
\end{align*}
implies that $\forall \ l \in \mcI$, and where $k \in \mcL_i$,
\begin{align}
\begin{split}
    z_i^l &= e_i(a) /\vs_i^l = \bigg[\min_l\big(\max_{l\ne i} e_i^j(a), \delta_i\big)\bigg]/ \vs_i^l\\
    &= \bigg[\min\big(\max_{j\ne i} e_i^j(a),\max_k(d_k : d_k < b_i)\big)\bigg]/ \vs_i^l\\
    & \ge  e_i^l(a) /\vs_i^l = a_i^l\vs_i^j/\vs_i^l = a_i^l\\
    & \ge \frac{d_i}{\sum_{k:p_k=p_i} d_k^l}d_i(p_i;s_i) \\
    & = \frac{d_i^l}{\sum_{k:p_k=p_i} d_k^l}\bigg[d_l^l-\displaystyle\sum_{p_k>y,k\ne
i} d_k^l\bigg]^+.
\end{split}
\end{align}
In other words, the buyer will bid the minimum of: 1) the entirety of the
sellers offer, and 2) the highest amount possible with respect to its budget $b_i$
such that it matches highest quantity from any bid in the auction. Using the allocation rule (\ref{allocation}) again, we have
\begin{align*}
    a_i^l(z_i,p_i) &= \max\bigg(z_i^l, \\
\frac{d_i^l}{\sum_{k:p_k=p_i} d_k^l}&\bigg[d_l^l-\displaystyle\sum_{p_k>y,k\ne
i} d_k^l\bigg]^+\bigg) \\
    &= z_i \\
    &= e_i(a) / \vs_i^l,
\end{align*}
and (\ref{coordination}) is proven.
We have that $e_i^l(a(z_i,p_i)) = a_i^l(z_i,p_i)\vs_i^l
= e_i(a) \ \forall \ l\ne i$, and so
$\theta_i\circ e_i(a(z_i, p_i)) = \theta_i\circ e_i(a)$. 
By the definition of utility,
\begin{align*}
    u_i(z_i,p_i) - u_i &= \theta_i\circ e_i(z_i,p_i)  - \displaystyle\sum_{l\ne
i}c_i(z_i,p_i) \\
    &- \theta_i(a_i(d_i,p_i)) - \displaystyle\sum_{l\ne i} c_i^l(s) \\
\end{align*}
(CHECK MATH!) Now, by definition of
the seller's valuation, $\forall \ l$,
\begin{align*}
    &=\displaystyle\sum_{l\ne i}\int_{a_i^l(z_i,p_i)}^{a_i^l} f^l(d_l^l
-z) \ dz \\
    e_i(a) &\ge e_i^l(a) \Rightarrow z_i^l\vs_i^l \ge a_i^l\vs_i^l \\
    &\Rightarrow a_i^l \le z_i^l\le a_i^l(z_i,p_i),
\end{align*}
Additionally, we comment that $x_i$ is feasible, i.e. $\sum_l c_i^l(s) \le b_i$
as the factor $\vs$ separates the auction space, restricting the number of bids
a user may place. In effect, we are using the buyer demand
to partition the auction space, thereby optimizing the message space for the
ISP, and providing an (BAD, DOESN'T MATTER) optimal market space to host the buyers and sellers based
on their type. 
(....AND!?) 
The rest of the proof follows as in \cite{semret}.
The idea of buyer coordination follows from \cite{lazar} (FINISH!)

In the case where no seller is offering enough data to satisfy a single user's
demand, the user is forced to submit bids to multiple sellers. According to (\ref{demand}) we claim that there is an incentive for the seller to
submit fractional bids (i.e. bid a fraction of thier leftover data) in the hopes
of maximizing the auction price in its
\emph{local} auction by including more sellers. To
formalize this we have the following lemma. (WILL NEED THIS LATER? BAD?)
{
\lemma{(Seller incentive)}\label{incentive}
Let $j$ be an opt-out buyer with fixed bid $(d^j,c^j)$. Then, in order to change the
quantity $d^j$ we must have, for some $k \in \mcI$,
\begin{equation}
       p_k^j \ge \max_i c_i^j\vs_i^j.
\end{equation}
}
\textbf{Proof:}
We will demonstrate a proof by cases:\\
(NEED TO THINK, I ONLY SAID THERE IS AN INCENTIVE TO DECREASE... STILL NEED
BOTH CASES THOUGH)\\
The idea is that as the sellers are decreasing their bid quantities, the buyers
are increasing them so that the result is distributed collaboration, resulting
in an inherent network matching $\rightarrow$ market equilibrium.
\emph{Case 1:} Suppose decrease $d$ (NEED PRICE? SHOULD NOT CHANGE FROM THEORY
BUT INCLUDE THIS FACT)\\
\emph{Case 2:} Increase $d$\\

(DOES THIS BELONG HERE?)
In addition, in the case that the seller does not know the exact amount of
leftover data available, then they may only sell enough data to ensure that
they will not become a buyer. (FUTURE WORK?)


{
\proposition{(Network incentive compatibility)}
(DOES THIS PROOF EVEN CHANGE? NEED TO START WORKING IT OUT)
Let $i\in\mcI$ be an opt-out buyer, and fix all other users' bids
$s_{-i}$, as well as the sellers' bids $s_i^i$ (so $a^i$ is fixed). \\
Let 
\begin{align}
    z_i &= \sup\bigg\lbrace h\ge 0 : \\
& \theta_i(h) > \displaystyle\sum_{j\ne i}
P_i^j\big((h-a_j^i)\vs_i^j\big)\vs^j\bigg\rbrace, 
\end{align}
\begin{align}\label{price}
    \psi_i &= \sup\bigg\lbrace h\ge 0: \\
&\displaystyle\int_0^h \displaystyle\sum_{j\ne i}
P_i^j\big((h-a_j^i)\vs_i^j\big)\vs^j \ dh \le b_i\bigg\rbrace,
\end{align}
$e = \max(z_i, \psi_i - \epsilon / \theta_i'(0))^+$, and for each $j\ne i$, 
$$
    v_i^j = (e - a_j^i)\vs_i^j
$$
and 
$$
    w_i^j = \vs_i^j\theta_i'(e).
$$
Then a (coordinated) $\epsilon$-best reply for the opt-out buyer is $t_i =
(v_i,w_i)$, i.e., $\forall \ s_i, u_i(t_i;s_{-i}) + \epsilon \ge u_i(s_i;
s_{-i})$.
} \\ \\
\textbf{Proof:} First suppose $e = z_i$. Since $\theta_i'$ is non-increasing
and $\forall j$, $P_i^j$ is non-decreasing, \ref{price} implies $\theta_i'(e) >
\sum_{j\ne i} P_i^j(v_i^j)\vs_i^j$, and so $\forall j\ne i$, 

\subsection{Efficiency}

\subsection{Convergence}

\end{multicols}

\begin{thebibliography}{9}

\bibitem{zheng}
L. Zheng, C. Joe-Wong, C. W. Tan, S. Ha and M. Chiangs, 
\textit{Secondary markets for mobile data: Feasibility and benefits of traded
data plans}, 2015 IEEE
Conference on Computer Communications (INFOCOM), Kowloon, 2015, pp. 1580-1588.

\bibitem{lazar}
A. A. Lazar and N. Semret, 
\textit{“Design and Analysis of the Progressive Second Price Auction for Network
Bandwidth Sharing,”} Telecommunication Systems, Special Issue on Network Economics, 2000.

\bibitem{semret} 
N. Semret, 
\textit{“Market Mechanisms for Network Resource Sharing,”
Ph.D. thesis}. 
Columbia University, 1999.

\bibitem{tuffin}
Bruno Tuffin,
\textit{Revisited Progressive Second Price Auction for Charging
Telecommunication Networks}.
[Research Report] RR-4176, INRIA, 2001.
 
\end{thebibliography}


\end{document}

