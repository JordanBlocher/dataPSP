\documentclass[12pt]{article}
 
\usepackage[text={6in,8.1in},centering]{geometry}

\usepackage{enumerate}
\usepackage{amsmath,amsthm,amssymb}
\usepackage{mathrsfs} % to use mathscr fonts

\usepackage{multicol}

\usepackage{epstopdf}
\usepackage{caption,subcaption}
\usepackage{pstricks}
\usepackage{pst-solides3d}
\usepackage{pstricks-add}
\usepackage{graphicx}
\usepackage{pst-tree}
\usepackage{pst-poly}
\usepackage{calc,ifthen}
\usepackage{float}\usepackage{multicol}
\usepackage{multirow}
\usepackage{array}
\usepackage{longtable}
\usepackage{fancyhdr}
\usepackage{algorithmicx,algpseudocode}
\usepackage{changepage}
\usepackage{color}
\usepackage{listings}
\usepackage{fancyvrb}
\usepackage{verbatim,moreverb}
\usepackage{courier}

\lstset{ %
language=C++,               
basicstyle=\footnotesize,
numbers=left,                  
numberstyle=\tiny,     
stepnumber=1,         
numbersep=5pt,         
backgroundcolor=\color{white},  
showspaces=false,               
showstringspaces=false,         
showtabs=false,                 
columns=fullflexible,
frame=single,          
tabsize=2,          
captionpos=b,       
extendedchars=true,
xleftmargin=17pt,
framexleftmargin=17pt,
framexrightmargin=17pt,
framexbottommargin=4pt,
breaklines=true,       
breakatwhitespace=false, 
escapeinside={\%*}{*)}       
}

\newenvironment{block}{\begin{adjustwidth}{1.5cm}{1.5cm}\noindent}{\end{adjustwidth}}

\newtheorem{proposition}{Proposition}[section]
\newtheorem{theorem}{Theorem}[section]
\newtheorem{lemma}{Lemma}[section]
\newtheorem{corollary}{Corollary}[section]
\theoremstyle{definition}
\newtheorem{definition}{Definition}[section]

 
\def\verbatimtabsize{4\relax}
\def\listingoffset{1em}
\def\listinglabel#1{\llap{\tiny\it\the#1}\hskip\listingoffset\relax}
\def\mylisting#1{{\fontsize{10}{11}\selectfont \listinginput[1]{1}{#1}}}
\def\myoutput#1{{\fontsize{9}{9.2}\selectfont\verbatimtabinput{#1}}}

\newcommand{\vs}{\varsigma}
 
\headsep25pt\headheight20pt
 
 
\pagestyle{fancyplain}
\rhead{\fancyplain{}{\small\bfseries Blocher, Jordan}}
\cfoot{\ \hfill\tiny\sl Draft printed on \today}
 
 
\setlength{\extrarowheight}{2.5pt} % defines the extra space in tables
 
\begin{document}
\begin{multicols}{2}

\section{Introduction}
China Mobile Hong Kong (CMHK) recently introduced such a secondary market.
CMHK’s 2cm (2nd exchange market) data exchange platform allows users to submit
bids to buy and sell data, with CMHK acting as a middleman both to match buyers
and sellers and to ensure that the sellers’ trading revenue and buyers’
purchased data are reflected on customers’ monthly bills. \cite{zheng}

In this work, we propose a distributed progressive second price (PSP) auction in
order to maximize social utility. We show that for cellular data allocated between multiple users there
exists a Nash market equilibria when all users bid their real marginal
valuation of the bandwidth resource. The P2P auction's (as in all auctions),
demand information is not known centrally, rather it is distributed in the
buyers' valuations. A basic goal is that the mechanism also be
\emph{distributed} in that the allocations at any element depend only on
\emph{local} state: the quantity offered by the seller at that element, and the bids for that element only.

We suppose that each seller (resp. buyer) can submit a bid to the secondary
market consisting of (i) an available (required) quantity and (ii) a unit-price (calculated
using its own demand functions). Buyers submit bids cyclically until an
($\epsilon$-Nash) equilibrium is reached where $\epsilon$ corresponds to a bid
fee to be paid to the ISP on completion of the transaction.

\subsection{Progressive Second Price Auctions}

The PSP auction first introduced in \cite{lazar} forms a part of the
overall market based allocation model. Consider a noncooperative game
where $I$ users buy the fixed amount of resource $D$ from one seller. Suppose
each user $i \in \mathcal{I}$ makes a bid $s = (p_i , d_i )$ to the seller,
where $p_i$ is the unit-price the user is willing to pay and $d_i$ is the
quantity the user desires. $s \equiv [s_i]_{i\in\mathcal{I}}$ is the bidding
profile and $s_{−i} \equiv [s_1 , \cdots , s_{i−1} , s_{i+1} , \cdots , s_N]$ is
the profile of user $i$’s opponents. The market price function (MPF) of user $i$ is defined as:
\begin{equation}
    P_i(z, s_i) = \inf\bigg\lbrace y\ge 0 : D - \displaystyle\sum_{p_k>y, k\ne i} d_k
\ge z \bigg\rbrace,
\end{equation}
which is interpreted as the minimum price a user bids in order to obtain the
resource $z$ given the opponents’ profile $s_−i$. Its inverse function $D_i$ is defined as follows:
\begin{equation}
    D_i(y, s_i) = \bigg\lbrack D - \displaystyle\sum_{p_k > y, k\ne i}
d_k\bigg\rbrack^+,
\end{equation}
which means the maximum available quantity at a bid price of $y$ given
$s_{−i}$. With this notation, the PSP allocation rule \cite{tuffin} is defined as
\begin{align}
    a_i(s) = \min\big\lbrace d_i, \frac{d_i}{\sum_{k:p_k=p_i}d_k} D_i(p_i,
s_{-i})\big\rbrace, \\
    c_i(s) = \displaystyle\sum_{j\ne i} p_j \big[a_j(0; s_{-i}) -
a_j(s_i;s_{-i}\big],
\end{align}
where $a_i$ denotes the quantity user $i$ obtains by a bid price $p_i$ (when the
opponents bid $s_{−i}$) and the charge to user $i$ by the seller is denoted
$c_i$. $c_i$ is interpreted to be the total cost incurred in the system if
user $i$ is removed from the auction. Note that the allocation rule is modified
according to \cite{tuffin}, so that buyers with identical unit-price
$p_i$ are not rejected.

We extend the P2P rules as in \cite{semret} to include a \emph{local} market
price function as determined by the subset of nodes participating in the
auction. Therefore the influence of the greater market on the individual
auctions will be influenced only by the submission of bids from buyers to
sellers. As a buyer may have access to multiple auctions, the sellers will be
dynamically influenced by the $\epsilon$-best replies from the buyers. (need to
start to explain the opt-out buyer here... is the basis for network
equilibrium, as the optimal solution is to have the opt-out function the same
for all sellers i.e. P2P theory..).

Absent the cost or revenue from trading data, users gain utility from consuming
data. We use the $\alpha$-fair utility functions \cite{tuffin} to model the
usage utility from consuming $d$ amount of data:
\begin{equation}
    \theta(d) = \frac{\sigma d^{1-\alpha}}{1-\alpha}
\end{equation}
where $\sigma$ is a positive constant representing (the scale) of the usage
utility and we take $\alpha \in [0, 1)$.
We verify that the user valuation above satisfies the conditions for an
\emph{elastic demand function}: (NOTE: this part seems in the wrong place)

\definition{\cite{lazar}}
A real valued function $\theta(\cdot)$ is an \emph{(elastic) valuation
function} on $[0, D]$ if 
\begin{itemize}
    \item $\theta(0) = 0$; \\
        \emph{Verification:} (obvious)
    \item $\theta$ is differentiable; \\
        \emph{Verification:} The derivative, $\sigma d^{-\alpha}$, is positive assuming non-negative
data requirements.
    \item $\theta ' \ge 0$, and $\theta '$ is non-increasing and continuous; \\
        \emph{Verification:} $U$ is differentiable for all $d$, and therefore
continuous. Its derivative is a negative exponential, and so is non-increasing.
    \item There exists $\gamma > 0$, such that for all $z \in [0,D]$, $\theta
'(z) > 0$ implies that for all $\eta \in [0, z), \theta '(z) \le \theta '(\eta)
- \gamma(z - \eta)$. \\
        \emph{Verification:} Without loss of generality, we may set the scaling
constant $\sigma=1$, and compute the curvature $\gamma(\xi)$, where by definition,
$$
    \gamma = \frac{\theta''}{(1+\theta')^{3/2}} = \frac{-\alpha
\xi^{-\alpha-1}}{(1+\xi^{-2\alpha)^{3/2}}}.
$$
Using a Taylor theorem approximation,
\begin{align*}
    z^{-\alpha} &\le \eta^{-\alpha} + \frac{-\alpha
d^{-\alpha-1}}{(1+\xi^{-2\alpha)^{3/2}}}(z-\eta) \\
    & \le \eta^{-\alpha} +
\frac{\alpha}{2\sqrt{2} \xi}(z-\eta)  \\
    &\le \eta^{-\alpha} +
\frac{\alpha}{\xi}(z-\eta).
\end{align*}
Now, using Taylor repeatedly, simplifying and taking the limit as $\eta \rightarrow z$,
\begin{align*}
    z^{-\alpha} - \eta^{-\alpha} &\le -\alpha\eta^{-\alpha-1}(z-\eta) \\
    & \le \frac{-\alpha}{\xi}(z-\eta).
\end{align*}
And so, since $\xi \le \eta^{\alpha+1}$,
we may set
$$
    \gamma \ge
\frac{-\alpha\eta^{-(\alpha+1)^2}}{\big(1+(\eta^{-2\alpha(\alpha+1)})^{3/2}},
$$
which holds in the case that $z > 1$, and so assuming that there must be at least one
unit of data required for a user to have
a valuation,  we have that the concavity of $\theta'$ is shown by Squeeze theorem.
\end{itemize}

We may now define the user's utitlity function as
\begin{equation}\label{utility}
    u_i = \theta_i(a_i(s)) - c_i(s).
\end{equation}

Under the PSP rule, \cite{lazar} shows that given the opponents bids $s_{-i}$,
user $i$'s $\epsilon$-best response $s_i = (w_i, v_i)$ as a Nash move
(where $s_i$ is chosen to maximize $i$'s utility with $s_{-i}$ held constant), is
given by:
\begin{align}
\begin{split}
    v_i &= \sup\bigg\lbrace d \ge 0 : \theta '(d) > P_i(d), \\ 
&\displaystyle\int_0^d P_i(\eta) \ d\eta \le b_i\bigg\rbrace -
\frac{\epsilon}{\theta_i'(0)} \\
&\qquad\qquad \text{(best quantity reply)} \\
\end{split}\\
    w_i &= \theta_i'(v_i) \quad \text{(best unit-price reply)},
\end{align}
where $\epsilon > 0$ is the bid fee, $b_i$ is user $i$'s budget, and every user
has an elastic demand function.

In the Secondary Market \cite{zheng}, we intend to show that it is optimal for a
buyer to fully satisfy thier demand in a single auction, that is, a buyer
will purchase its required data from a single seller. (WHAT?? why do I think
this? Answer: defining the routes as the sellers who have the EXACT amount of
data that the buyer requires, thus the min route in \cite{lazar} is the same as the subset of
users who have just enough data to meet the buyer's needs)

\section{Related Work}
 

\section{The Problem Model}
\subsection{The Secondary Market}

We consider the set of $\mathcal{I} = {1, \cdots, I}$ users who purchase or sell
data from other users. A buyer is matched with sellers who enough leftover data
to satisfy thier demand, and will submit bids in order to maximize thier (private) valuation.
A user's identity $i \in \mathcal{I}$ as a subscript indicates that the user
is a buyer, and a superscript indicates the seller.
Suppose user $i$ is buying from user $j$. A bid $s_i^j = (d_i^j, p_i^j)$,
meaning $i$ would like to buy from $j$ a quantity $d^j_i$ and is willing to pay
a unit price $p^j_i$. Without loss of generality, we assume that all users bid in all
auctions; if a user $i$ does not need to buy from j, then this means that the
user has the exact amout of data they require and we simply set $s^j_i = (0, 0)$.
A seller $j$ places an ask $s_j^j = (d_j^j, p_j^j)$, meaning $j$ is offering a
quantity $d_j^j$ , with a reserve unit price of $p_j^j$ . In other words, when the subscript and
superscript are the same, the bid is understood as an offer in the secondary
market; we assume that data is a unary resource belonging to the seller, and
therefore can identify the data (for sale) with the identity of the user.

Based on the profile of bids $s^j = (s^j_1, \cdots s^j_I)$, seller $j$ computes an
allocation $(a^j, c^j) = A^j(s^j)$, where $a^j_i$ is the quantity given to user
$i$ and $c^j_i$ is the total cost charged to user $i$. $A^j$ is the allocation
rule of seller $j$. It is feasible if $a^j_i \le d_i^j$, and $c^j_i \le  p^j_i
d_i^j$.

\subsection{User Behavior}

We define a \textbf{opt-out buyer} as a user $i\in\mathcal{I}$ with utility of
the form 
\begin{equation}
    u_i = \theta_i \circ e_i(a) - \displaystyle\sum_j c_i^j,
\end{equation}
where $e_i : [0, \infty) \rightarrow [0,\infty)$ is the expectation that user
$i$ finds a matching seller $j$. An opt-out buyer's valuation
depends only on a scalar $e_i(a)$ which is a function of the
quantities of all the available data for sale in the secondary market. Buyer
$i$'s valuation may now be interpreted as a unit valuation $\theta_i$, scaled
by a function of quantity desired from the market. 
We define, in addition to the valuation and budget of user $i$, a generic
\textbf{data-provisioning vector} $\vs_i$.  Define, for any allocation $a$,
\begin{equation}\label{opt-out}
    %e_i^j(a) \triangleq \frac{a_i^j}{\vs_i^j} + a_j^i,
    e_i^j(a) \triangleq a_i^j\vs_i^j,
\end{equation}
and let
\begin{equation}
    e_i(a) \triangleq \max_{j\ne i} e_i^j(a).
\end{equation}

Deviating slightly from the
bottleneck player defined in \cite{semret}, in addition to taking the inverse
of the route-provisioning vector to suit the data problem, we omit the term $a_j^i$ from the lemma, as it is
assumed that buyers and sellers are separated (a seller does not also buy data
and vice versa).
(NOTE: proof should come easily! make sure!)

Consider a user seeking to prevent
data overage by purchasing enough data from a subset of other network users.
This user $i$ can be modeled as a opt-out buyer where $\vs_i^j$ denotes the
fraction of user $j$'s data aquired by user $i$. Buyer $i$ has an incentive to
to maximize the amount of requested data from one seller, thereby minimizing
the bid fee paid to the ISP and still fufill thier data requirement.
For the sellers that do not meet a competing buyer $k$'s data requirements, a rational (utility-maximizing) buyer will
set $s_k^j = 0$. We note that since $i$ is not a seller, $d_i^i = 0$ and $a^i = 0$.

Suppose the total amount of seller $j$'s data on the network at the instance that
user $i$ joins the auction is $\chi_j$. For the service to function as desired,
the data transfer from each seller cannot exceed the total amount they have available,
i.e. $a_i^j \le \chi_j\vs_i^j$. This will hold simultaneously for all $j$ if
and only if $\chi_j \ge \max_j \ a_i^j \vs_i^j = e_i(a)$. Thus $e_i(a)$
lower-bounds the amount of data that each seller $j$ has in an auction. We
determine (explain further?) the
valuation of the transaction between seller $j$ and buyer $i$ is well-defined,
and the form of (\ref{opt-out}) is justified. (Does this make sense? and check
the MATH!)

In order to form the distributed auction, we set $\vs_i^j=1$ for all sellers
who offer enough data to meet the needs required by buyer $i$, and $\vs_i^j=0$ for
all other $j$. This restricts the number of auctions in which the user is able to
participate. We intend to show that this does not affect thier valuation, and
indeed, in this network setting, results in
(EEEK! say something else like... results in a shared network optima, or a
global optimum.) the smallest
number of auctions that the network participates in as a whole (NOTE: can I
formulate this?), and is inversely analogous to the thinnest allocation route for
bandwidth given in \cite{lazar}. Reasonably, if only a single seller is available, then
(\ref{opt-out}) reduces to the original form (\ref{utility}), defined in
\cite{semret} as "a simple buyer at a single resource element".

The seller, in our analysis, is an extension of the buyer, where the valuation
$\theta^j$ is dependent on the buyers. The natural utility is the potential
profit from buyer $i$, $u^j = \theta^j\circ e_i(a)$, where $\theta^j$ is the
potential revenue from the sale of data composed with buyer $i$'s opt-out value
$e_i(a)$. We derive the potential revenue as a function of demand $\varrho$ as in
\cite{semret}, $\forall \ y\ge 0$, the demand for auction $j$ at unit price $y$ is given by
$$
    \varrho^j(y) \triangleq \displaystyle\sum_{p_j^i\ge y} d_j^i,
$$
with inverse
$$
    f^j(z) \triangleq \sup\big\lbrace y\ge 0: \varrho^j(y) \ge z\big\rbrace.
$$
Therefore, we have the seller's valuation \cite{semret},
$$
    \theta^j(x) = \displaystyle\int_0^x f^j(z) \ dz,
$$
and 
$$
    \theta^j\circ e_i(a) = \displaystyle\int_0^{e_i(a)} f^j(z) \ dz.
$$
We note the difference in subscript/superscript notation from \cite{semret},
 we emphasize the separation of buyers and sellers, and our
claim that the seller's valuation depends on the opt-out \emph{buyer demand}.
The proof remains the same from \cite{semret}.

\section{Network Data P2P Analysis}

\subsection{Equilibrium}

Consider an opt-out buyer $i\in\mathcal{I}$, participating in many auctions
simultaneously. Due to (\ref{opt-out}), data allocations are only valuable to the
seller if they decrease the expected matching of buyer $i$ (i.e. the buyer finds another
seller). This creates an incentive for $i$ to coordinate its bids
to maximize the number of available sellers, and therefore maximize its overall
utitlity. We show that, without loss of utility, a buyer $i$ can decrease its
bid quantities $d_i$ to the level where the opt-out value is the same for all
$j$ sellers. We show that buyer coordination \cite{semret} holds in the
secondary data market.
{
\lemma{(Opt-out buyer coordination)} Let $i\in\mathcal{I}$ be a opt-out buyer.
For any profile $s_i = (d_i, p_i)$, let $a \equiv a(s)$ be the resulting data
tranfer. For a fixed $s_{-i}$, a better reply for $i$ is $x_i =
(z_i,p_i)$ where $\forall \ j \ne i$,
$$
    z_i^j = \big\lbrack e_i(a) \big\rbrack,
$$
and
\begin{equation}\label{coordination}
    a_i^j(z_i,p_i) = z_i^j.
\end{equation}
}
\textbf{Proof:} We will show that
\begin{equation}
    u_i(s_i;s_{-i}) \equiv u_i(d_i,p_i) \le u_i(z_i,p_i)
\end{equation}
using the modified allocation rule from \cite{tuffin} and our current
formulation. $\forall \ j \in \mathcal{I}$,
\begin{align}
\begin{split}
    z_i^j &= e_i(a) = \max_{j\ne i} e_i^j(a)\\
    & \le  e_i^j(a) = a_i^j\vs_i^j\\
    & \le \frac{d_i}{\sum_{k:p_k=p_i} d_k}d_i(p_i;s_i) \\
    & = \frac{d_i}{\sum_{k:p_k=p_i} d_k}\bigg[D-\displaystyle\sum_{p_k>y,k\ne
i} d_k\bigg]^+.
\end{split}
\end{align}
Using the allocation rule again, we have
\begin{align*}
    a_i^j(z_i,p_i) &= \min\bigg(z_i^j, \\
\frac{d_i}{\sum_{k:p_k=p_i} d_k}&\bigg[D-\displaystyle\sum_{p_k>y,k\ne
i} d_k\bigg]^+\bigg),
\end{align*}
and (\ref{coordination}) is proven. We may now confirm that $e_i(a_i^j(z_i,p_i)) = a_i^j(z_i,p_i)
= z_i = e_i(a) \ \forall \ j\ne i$.
$\theta_i\circ e_i(a(z_i, p_i)) = e_i(a)$, 
{
\proposition{(Network incentive compatibility)}
Let $i\in\mathcal{I}$ be an opt-out buyer, and fix all other users' bids
$s_{-i}$, as well as the sellers' bids $s_i^i$ (so $a^i$ is fixed). \\
Let 
\begin{align}
    z_i &= \sup\bigg\lbrace h\ge 0 : \\
& \theta_i(h) > \displaystyle\sum_{j\ne i}
P_i^j\big((h-a_j^i)\vs_i^j\big)\sigma_i^j\bigg\rbrace, 
\end{align}
\begin{align}\label{price}
    \psi_i &= \sup\bigg\lbrace h\ge 0: \\
&\displaystyle\int_0^h \displaystyle\sum_{j\ne i}
P_i^j\big((h-a_j^i)\vs_i^j\big)\sigma_i^j \ dh \le b_i\bigg\rbrace,
\end{align}
$e = \max(z_i, \psi_i - \epsilon / \theta_i'(0))^+$, and for each $j\ne i$, 
$$
    v_i^j = (e - a_j^i)\vs_i^j
$$
and 
$$
    w_i^j = \vs_i^j\theta_i'(e).
$$
Then a (coordinated) $\epsilon$-best reply for the opt-out buyer is $t_i =
(v_i,w_i)$, i.e., $\forall \ s_i, u_i(t_i;s_{-i}) + \epsilon \ge u_i(s_i;
s_{-i})$.
} \\ \\
\textbf{Proof:} First suppose $e = z_i$. Since $\theta_i'$ is non-increasing
and $\forall j$, $P_i^j$ is non-decreasing, \ref{price} implies $\theta_i'(e) >
\sum_{j\ne i} P_i^j(v_i^j)\vs_i^j$, and so $\forall j\ne i$, 

\subsection{Efficiency}

\subsection{Convergence}

\end{multicols}

\begin{thebibliography}{9}

\bibitem{zheng}
L. Zheng, C. Joe-Wong, C. W. Tan, S. Ha and M. Chiangs, 
\textit{Secondary markets for mobile data: Feasibility and benefits of traded
data plans}, 2015 IEEE
Conference on Computer Communications (INFOCOM), Kowloon, 2015, pp. 1580-1588.

\bibitem{lazar}
A. A. Lazar and N. Semret, 
\textit{“Design and Analysis of the Progressive Second Price Auction for Network
Bandwidth Sharing,”} Telecommunication Systems, Special Issue on Network Economics, 2000.

\bibitem{semret} 
N. Semret, 
\textit{“Market Mechanisms for Network Resource Sharing,”
Ph.D. thesis}. 
Columbia University, 1999.

\bibitem{tuffin}
Bruno Tuffin,
\textit{Revisited Progressive Second Price Auction for Charging
Telecommunication Networks}.
[Research Report] RR-4176, INRIA, 2001.
 
\end{thebibliography}


\end{document}

