%\iffalse
\section{Distributed Progressive Second Price Auctions}

Progressive second price auctions (PSPs) were proposed
in \cite{lazar}, \cite{diffserve} to provide a dynamic network service pricing
scheme to provide consistent services for network bandwith users.
\cite{diffserve} conducts a game theoretic analysis, deriving
optimal strategies for buyers and brokers, and further shows the existence
of networkwide market equilibria based on their game-theoretic
model. Constructing necessary and sufficient conditions for the stability
of the game allows the sustainability of any set
of service level agreement configurations between Internet service
providers.

We begin with a brief introduction to the distributed PSP auction for bandwith
sharing, first introduced by Lazar and Semret \cite{lazar}.
We define a set of $\mcI = \lbrace 1,\cdots,I\rbrace$ network bandwidth users.
Suppose each user $i \in \mcI$ makes a bid $s_i^j = (p_i^j, d_i^j)$ to the
seller of resource $j$, where $p_i^j$ is the unit-price the user is willing to
pay and $d_i^j$ is the quantity the user desires. The \emph{bidding profile} forms a grid, $s \equiv
[s_i^j] \in \mcI \times \mcI$, and $s_{-i} \equiv [s_1^j , \cdots , s_{i-1}^j , s_{i+1}^j , \cdots
, s_I^j]_{j\in\mcI}$ is the profile of user $i$'s opponents. 
Using this classic PSP mechanism, \cite{lazar} shows that given the opponents
bids $s_{-i}$,
user $i$'s $\epsilon$-best response to seller $j$ is $s_i^j = (w_i^j, v_i^j)$
and is a Nash move
where $\epsilon > 0$ is the bid fee, $B_i =\sum_{j\in\mcI} b_i^j$ is user $i$'s
budget, and every user has an elastic demand function.
Based on the profile of bids $s^j = [s^j_1, \cdots , s^j_I]$, the seller applies
an allocation rule $a(s^j) = [a_1^j, \cdots , a_I^j]$, where $a^j_i$ is the quantity allocated
by $j$ to each user $i\in\mcI$ and $c^j_i$ is the cost charged to $i$ for
allocations awarded in auction $j$. 
An allocation is considered feasible if $a^j_i \le d_i^j$, and $c^j_i \le  p^j_i d_i^j$.

% MECHANISM
\subsubsection{The PSP Mechanism}\label{mechanism}
The PSP auction as given in \cite{lazar} and \cite{semret} is designed for the
problem of network bandwidth allocation, and is analyzed as a noncooperative
game where $i\in\mcI$ agents buy the
fixed amount of bandwidth $d_i^j$ from sellers $j\in\mcI$.
The market price function (MPF) for a buyer-seller pair is,
% MARKET PRICE
\begin{align}\label{dataprice}
\begin{split}
    P_i^j(z, s_{-i})= \inf\bigg\lbrace y\ge 0 : D_i^j(y, s_{-i}) \ge z \bigg\rbrace,\\
\end{split}
\end{align}
and is the of minimum prices a user bids in
order to obtain bandwidth $z$ given opponent profile $s_{-i}$. 
The maximum available quantity of data in auction
$j$ at unit price $y$ given $s_{-i}^j$ is,
% INVERSE DEMAND
\begin{align}\label{datapriceinverse}
\begin{split}
    D_i^j(y, s_{-i}) &= \bigg[ D^j - \sum_{p_k^j>y,k\ne i} d_k^j  \bigg]^+,
\end{split}
\end{align}
where $D^j$ is the total amount of bandwith that user $j$ has to offer.
% DATA ALLOCATION RULE
For each $i \in \mcI$, the allocation from auction $j$ is,
% and so for profile $s_i$, for any $i \in \mcI$,
\begin{align}\label{dataallocation}
    a_i^j(s) &= \min\bigg( d_i^j, 
    D_i^j(p_i^j,s_{-i}^j)\bigg).
\end{align}
Finally, we have the cost of the allocation,
% DATA COST
\begin{align}\label{datacost}
    c_i^j(s) = \displaystyle\sum_{k\ne i} 
p_k^j \big[a_k^j(0; s_{-i}^j)
    -a_k^j(s_i^j;s_{-i}^j)\big].
\end{align}

It was shown, in \cite{lazar}, that the mechanism may converge to a Nash market
equilibria for differentiated services
allocated between multiple agents when all players bid their real marginal valuation
of the bandwidth resource. In other words, the PSP constraints are
sufficient to attain the desirable property of truthfullness through incentive
compatibility. The pricing mechanism upholds the \emph{exclusion-compensation
principle}, user $i$
pays for its allocation so as to exactly cover the ``social opportunity cost"
which is given by the declared willingness to pay (bids) of the users who are
excluded by $i$'s presence, and thus also compensates the seller for the maximum lost potential
revenue \cite{lazar}.
{
\definition{(Nash Equilibrium)}
A Nash equilibrium is defined as a strategy vector, or, in terms of PSP, a bid
profile $s$,
from which no player has a unilateral incentive to deviate (Johari, 2004)
(EXPAND?)
}\\

The PSP rules assume that an agent's valuation is represented by an elastic valuation function.
\definition{\cite{lazar}}
A real valued function, $\theta(\cdot): [0,\infty) \rightarrow [0,\infty)$, is an \emph{(elastic) valuation
function} on $[0, D]$ if 
\begin{itemize}
    \item $\theta(0) = 0$,
    \item $\theta$ is differentiable,
    \item ${\theta}' \ge 0$, and ${\theta_i}'$ is non-increasing and continuous,
     \item There exists $\gamma > 0$, such that for all $z \in [0,D]$,
${\theta}'(z) > 0$ implies that for all $\eta \in [0, z), {\theta}'(z) \le
{\theta}'(\eta)
- \gamma(z - \eta)$. 
\end{itemize}
The function $\theta'(\cdot)$ on $[0, D]$ is called an (elastic) demand
function.
In the PSP market, a
user is considered truthful if their bid price equals their marginal valuation,
i.e. $p_i^j = {\theta_i^j}'$.
%\fi


